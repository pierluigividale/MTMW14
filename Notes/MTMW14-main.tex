%%%%%%%%%%%%%%%%%%%%%%%%%%%%%%%%%%%%%%%%%
%  My documentation report
%  Objetive: Explain what I did and how, so someone can continue with the investigation
%
% Important note:
% Chapter heading images should have a 2:1 width:height ratio,
% e.g. 920px width and 460px height.
%
%%%%%%%%%%%%%%%%%%%%%%%%%%%%%%%%%%%%%%%%%

%----------------------------------------------------------------------------------------
%	PACKAGES AND OTHER DOCUMENT CONFIGURATIONS
%----------------------------------------------------------------------------------------

\documentclass[11pt,fleqn]{book} % Default font size and left-justified equations

\usepackage[top=3cm,bottom=3cm,left=3.2cm,right=3.2cm,headsep=10pt,letterpaper]{geometry} % Page margins

\usepackage{xcolor,lipsum} % Required for specifying colors by name
\definecolor{ocre}{RGB}{51,102,0} 
\definecolor{lightgray}{RGB}{229,229,229} 
% Font Settings
\usepackage{avant} % Use the Avantgarde font for headings
%\usepackage{times} % Use the Times font for headings
\usepackage{mathptmx} % Use the Adobe Times Roman as the default text font together with math symbols from the Sym­bol, Chancery and Com­puter Modern fonts

\usepackage{microtype} % Slightly tweak font spacing for aesthetics
\usepackage[utf8]{inputenc} % Required for including letters with accents
\usepackage[T1]{fontenc} % Use 8-bit encoding that has 256 glyphs

% MATHS PACKAGE
\usepackage{amsmath,tikz}
\usetikzlibrary{matrix}
\newcommand*{\horzbar}{\rule[0.05ex]{2.5ex}{0.5pt}}
\usepackage{calc}

% VERBATIM PACKAGE
\usepackage{verbatim}

% Bibliography
\usepackage[style=alphabetic,sorting=nyt,sortcites=true,autopunct=true,babel=hyphen,hyperref=true,abbreviate=false,backref=true,backend=biber]{biblatex}
\addbibresource{bibliography.bib} % BibTeX bibliography file
\defbibheading{bibempty}{}

\input{structure} % Insert the commands.tex file which contains the majority of the structure behind the template

\begin{document}
	
	\let\cleardoublepage\clearpage
	
	%----------------------------------------------------------------------------------------
	%	TITLE PAGE
	%----------------------------------------------------------------------------------------
	
	\begingroup
	\thispagestyle{empty}
	\AddToShipoutPicture*{\put(0,0){\includegraphics[scale=.075]{stars_cloud_moon4}}} % Image background
	\centering
	\vspace*{0cm}
	\par\normalfont\fontsize{30}{25}\sffamily\selectfont
	\textbf{MTMW14\\NUMERICAL MODELLING OF THE ATMOSPHERE AND OCEAN}\\
	{\LARGE }\par % Book title
	\vspace*{1cm}
	{\color{white}{\Huge P.L. Vidale \\ January 2024}}\par % Author name
	\endgroup
	
	%----------------------------------------------------------------------------------------
	%	COPYRIGHT PAGE
	%----------------------------------------------------------------------------------------
	
	\newpage
	~\vfill
	\thispagestyle{empty}
	
	%\noindent Copyright \copyright\ 2014 Andrea Hidalgo\\ % Copyright notice
	
	\noindent \textsc{Jan 2024, the University of Reading, Reading, UK}\\
	
	\noindent Original template design and graphics by Andrea Hidalgo (2014) and Hervé CARDOT (2015).\\ % License information
	
	\noindent Front page image from Apollo run by Daniel Klocke and colleagues, MPI Hamburg (2023).\\ % License information
	
	\noindent \textit{Published in Jan 2024} % Printing/edition date
	
	%----------------------------------------------------------------------------------------
	%	TABLE OF CONTENTS
	%----------------------------------------------------------------------------------------
	
	\chapterimage{2613-1477-max.jpg} % heading image	
	\pagestyle{empty} % No headers
	
	\renewcommand\contentsname{Table of Contents}
	\renewcommand{\bibname}{Bibliography}
	\tableofcontents% Print the table of contents itself
	
	%\cleardoublepage % Forces the first chapter to start on an odd page so it's on the right
	
	\pagestyle{fancy} % Print headers again
	
	%----------------------------------------------------------------------------------------
	%	CHAPTER 1
	%----------------------------------------------------------------------------------------
	
\chapterimage{2613-1477-max.jpg} % Chapter heading image	
\chapter{Introduction}
	
	\section{Module guide}
	\subsection{MTMW14 objectives}
	MTMW14 aims to take you from the basic principles of numerical modelling, mostly learned in MTMW12, to the foundations of current weather and climate models, bringing you up to speed with the vital components of state-of-the-art numerical models for atmospheres and oceans and their use in predicting weather and climate.
	
	Starting from fundamental properties of atmosphere and ocean:
\begin{itemize}
\item	Learn how to design numerical schemes, preserving key properties.
\item	Perform simple prediction using models you will build.
\item	Critically evaluate and interrogate models.
\item	Prepare you for designing, implementing and testing your own components, to be inserted and used in complex models.
\end{itemize}
	
	This competence is going to be very useful in your careers, even if you aim to analyse, albeit not develop models. All is accomplished by doing, rather than just watching a lecturer.\\
	
	
	This module is therefore mostly hands-on: you will learn by doing, and 70\% of the total marks are awarded for your work in the two projects.  \\
	
	Project 1 is about oscillatory / periodic behaviour, and is designed to explore ideas of model initialisation, chaos, ensemble forecasting. \\
	
	Project 2 is about 2D wave motion, and introduces spatial schemes and solution methods that are commonly found in today's state-of-the-art models. \\
	
	In both cases, it is important to distinguish what is expected from the analytical solutions (when available) and what is produced by the numerical solutions.
	
	\subsection{A map of MTMW14 topics}
	\begin{figure}
	\includegraphics[width=1.\textwidth]{Figures/MTMW14_mindmap.png}
	\caption{A map of MTMW14 topics, and how they are related}
	\end{figure}
	
	\subsection{MTMW14 calendar}
	
	
	\begin{figure}
	\includegraphics[width=1.5\textwidth, angle=90]{MT4YF_MTMW14 Class Calendar 2024.pdf}
	\caption{The 2024 calendar: please always refer to the eletronic version for last-minute changes of venue}
	\end{figure}
	
	\subsection{MTMW14 resources}
	
	\begin{itemize}
		\item	Dale Durran's book
		\item	Dave Randall's notes
		\item	MTMW12 notes
		\item	These notes and the papers introducing the two projects
	\end{itemize}
	
	\section{Motivation}\index{Motivation}
	
	\vspace{1em}
	
	Numerical modelling is ...
	
	\vspace{2em}
	
	\subsection{Going for numbers}
	
	\vspace{1em}
	
	There are no analytical solutions to the Primitive Equations: we mostly rely on numerical solutions, which are achieved by a number of methods. The most common method is the finite difference method, but we shall briefly look into the spectral and finite elements methods.
	
	The following is a mindmap of this ten week module.
	
	
	\section{Historical aspects of Numerical Modelling}
	
	This \href{https://journals.ametsoc.org/view/journals/amsm/59/1/amsmonographs-d-18-0018.1.xml}{Historic review by David Randall} provides a great and synthetic history of numerical modelling for the atmosphere and ocean, which has evolved from very simple box models all the way to complex, multi-scale, multi-component "Earth System Models". \\
	
		\begin{figure}[h]
		\includegraphics[width=1.\textwidth, angle=0]{Figures/ClimateModellingTimeline}
		\caption{As climate modeling has evolved over the last 120 years, increasing amounts of physical science have been incorporated into the models. This figure shows when various components of the climate system became regularly included in global climate model simulations.}
	\end{figure}

	Please read the article, which is available free of charge here:  

\href{https://journals.ametsoc.org/view/journals/amsm/59/1/amsmonographs-d-18-0018.1.xml} {\includegraphics[width=1.\textwidth, angle=0]{Figures/Randall-ESM100-title.png}}


	
	
%%%%%%%%%%%%%%%%%%%%%%%%%%%%%%%%%%%%%%%%%%%%%%%
\newpage	
\chapterimage{2613-1477-max.jpg} % Chapter heading image
\chapter{Time Schemes}
%%%%%%%%%%%%%%%%%%%%%%%%%%%%%%%%%%%%%%%%%%%%%%%
	\vspace{1em}
	
The fundamental ideas in this chapter ...
	

\subsection{Real world applications: hurricane and Typhoon simulation in climate GCMs}

\begin{center}	
	\includegraphics[width=0.8\textwidth]{Figures/TC-structures}
\end{center}

\subsection{Why the differences}
These GCMs are very different in terms of dynamical core, parameterizations etc., but they are also quite different in terms of the use of the time step. I have been experimenting with our UK model (HadGEM3), and Colin Zarzycki has been experimenting with the NCAR model, managing to double the number of hurricanes with a time step of 1/4, but I have also tested the same ideas in the ECMWF model, since it is the one that uses very long time steps.
\begin{center}	
	\includegraphics[width=0.8\textwidth]{Figures/tden_PRESENT_TC_map_STD}
\end{center}
	
	
%%%%%%%%%%%%%%%%%%%%%%%%%%%%%%%%%%%%%%%%%%%%%%%
\chapterimage{2613-1477-max.jpg} % Chapter heading image	
\chapter{Lorenz's Chaos and Predictability}
	
\vspace{1em}
	
The fundamental ideas in this chapter revolve around chaos theory, and how it affects predictability in the Earth System.
	
\vspace{1em}
	
	\section{Can we predict the predictions?}
	
	\vspace{1em}
		
%%%%%%%%%%%%%%%%%%%%%%%%%%%%%%%%%%%%%%%%%%%%%%%%%%
% 2D CFL
\newpage	
%%%%%%%%%%%%%%%%%%%%%%%%%%%%%%%%%%%%%%%%%%%%%%%%%%%%%%%%%%%%%%%%%%%%	
\chapterimage{2613-1477-max.jpg} % Chapter heading image	
\chapter{Courant-Friedrichs-Levy (CFL) condition}
%%%%%%%%%%%%%%%%%%%%%%%%%%%%%%%%%%%%%%%%%%%%%%%%%%%%

\section{Re-visiting advection and time step limitations}
\subsection{A graphical description}

	Let us start with something familiar, e.g. simple advection solved with CTCS.
	We have our beloved CFL criterion from studying the problem of advection:
	$\alpha = U \frac{\Delta t}{\Delta h} \le \alpha_{max}$
	, where $\Delta h$ is our grid spacing.\\
	
	~

\begin{figure}
	\begin{center}
\includegraphics[width=0.55\textwidth]{Figures/image_grid_cfl.jpg}
	\end{center}
\caption{CFL in the x direction}
\end{figure}
	
What is the practical meaning of CFL for Eulearian finite differences?

~

If we think of it as a method for propagating information in space, the CFL criterion is telling us that we are not allowed to transmit information from grid point $i$ to grid point $i+2$ without first making a stop at grid point $i+1$.
If our information "leapfrogs", very bad things will happen.

	\begin{center}
\includegraphics[width=0.2\textwidth]{Figures/GameOver.jpg}
	\end{center}
	

\subsection{A short reminder of the meaning of the CFL condition}
\subsection{The example of advection in 1D}

	Let us start with something familiar, e.g. simple advection solved with CTCS:
	
	\begin{equation}
		\frac {A_j^{n+1}-A_j^{n-1}}{ 2 \Delta t} +c \left(  \frac {A_{j+1}^{n}-A_{j-1}^{n}}{ 2 \Delta x}  \right) = 0
		\label{adv-CTCS-1D}
	\end{equation}
	
	We can substitute a (Von Neumann) solution of this type: ${A_j^{n}=B^{n \Delta t} e^{i \mu m \Delta x}}$, where $\mu$ is the horizontal wavenumber and $m \Delta x$ is the distance along the $x$ axis. $B$ can be any complex number. If we substitute into our finite difference (CTCS) scheme above we obtain:
	
	\begin{equation}
		\left ( B^{(n+1)\Delta t}-B^{(n-1) \Delta t} \right ) e^{i \mu m \Delta x} = - \frac {c \Delta t}{\Delta x} B^{n \Delta t} \left( e^{i \mu (m+1) \Delta x}  - e^{i \mu (m-1) \Delta x}  \right)
		\label{adv-CTCS-1D-solution}
	\end{equation}
	
	If we remember Euler's formula: $e^{i \theta}=\cos \theta +i \sin \theta$ and we multiply the expression above by $B^{\Delta t}$ we obtain a simple equation, after we cancel out the common term $A_j^{n}$:
	
	\begin{equation}
		B^{2 \Delta t}    + 2 i \sigma B^{\Delta t}  -1 = 0  
		\label{adv-CTCS-1D-stability}
	\end{equation}
	
	where $\sigma = \frac {c \Delta t}{\Delta x} \sin \mu \Delta x$, so that (\ref{adv-CTCS-1D-stability}) has the solution:
	
	\begin{equation}
		B^{\Delta t} = -i \sigma \pm \left( 1 - \sigma^2 \right)^{1/2} 
		\label{adv-CTCS-1D-eigenvalues}
	\end{equation}
	

	Two cases may be considered: \textbf{stable}, with $|\sigma| \leq 1$ and \textbf{unstable}, with $|\sigma| > 1$.
	\medskip
	
	The \underline{stable case} is very interesting from the point of view of phase, group velocity etc., but does not seem to pose major threats to the stability of the simulation (see Haltiner and Williams, pages 112-119).
	
	\medskip
	The \underline{unstable case} concerns us from the point of view of the amplitude of the signal. In fact, if $|\sigma| > 1$, then $\left( 1 - \sigma^2 \right)^{1/2}$ is imaginary and both roots will be pure imaginary:
	
	
	
	\begin{align}
		B_{+}^{\Delta t}  &= -i \left ( \sigma -S \right)  \mathrm{,where } | \sigma| > S \equiv \left(  \sigma^2 - 1 \right)^{1/2} \nonumber \\
		B_{-}^{\Delta t}  &= -i \left ( \sigma +S \right)  \nonumber
		\label{adv-CTCS-1D-unstable-solution}
	\end{align}
	
	If $\sigma$ is positive, the magnitude $R=|B_{-}^{\Delta t}| > 1$ and the solution $ B_{-}^{\Delta t} = R e^{-i \pi /2 }$ will thus grow exponentially when raised to the power of $n$ across the time steps. If $\sigma$ is positive, the other root has a magnitude exceeding $1$. \\
	In either case, the solution: $A_j^{n} = \left( M B_{+}^{n \Delta t}+ E B_{-}^{n \Delta t} \right)e^{i \mu m \Delta x}$  (with two arbitrary constants, $M,E$, to be determined later), will amplify with increasing time, which is not a desired property, since it does not correspond to the true solution of the differential equation. This phenomenon of \emph{exponential amplification of the solution} is known as \emph{computational instability} and must be avoided at all costs. 
	
\subsection{The Courant-Friedrichs-Levy (CFL) condition in 1D}

	So, we are left with having to impose a \emph{condition for a stable solution} $|\sigma| \leq 1$, that is:
	
	\begin{equation}
		\left | \frac {c \Delta t}{\Delta x} \sin \mu \Delta x \right | \leq 1
	\end{equation}	
	
	If this condition is to hold for all admissible values of the wavelength $\mu$, then the maximum value of $\sin \mu \Delta x$ will happen for the highest resolved wavenumber (that is a wavelenght $L=4 \Delta x$), which requires that:
	
	\begin{equation}
		\left | \frac {c \Delta t}{\Delta x}  \right | \leq 1
	\end{equation}	
	which is commonly referred to as the \emph{\textbf{Courant-Friedrichs-Levy (CFL)} condition for computational stability}.


\subsection{The CFL condition in 1D: plugging in some numbers}


What happens in practice? What typical time step are we contending with, in the horizontal and vertical directions?

\medskip

%\begin{minipage}{0.45\textwidth}
	{\bf Horizontal example:} $\Delta x = 10km$, $c = O(100) ms^{-1}$ what $\Delta t$ can we afford?
	
	\medskip
	
	But what happens for a contemporary grid, like this one?	
	\begin{figure}
		\includegraphics[trim={0cm 0.cm 0cm 0cm},clip,width=0.65\textwidth]{Figures/horizontal-irregular-grid}
	\end{figure}
%\end{minipage}

\hfill

%\begin{minipage}{0.45\textwidth}

{\bf Vertical example:} $\Delta z = 500m$, $c = O(1) ms^{-1}$ what $\Delta t$ can we afford?

	\medskip

But what happens for a typical vertical atmospheric grid, telescoped, like this one?
	\begin{figure}
	\includegraphics[trim={0cm 0.cm 0cm 0cm},clip,width=0.6\textwidth]{Figures/vertical-telescoped-grid}
\end{figure}
%\end{minipage}


\section{The CFL condition in 2 dimensions}

\subsection{The Courant-Friedrichs-Levy (CFL) condition in 2D}

	Same problem as before, albeit advection in 2D:
	
	\begin{equation}
		\frac{\partial A}{\partial t} + \mathbf{V}_s \cdot \grad{A} = 0 \textrm{, where } \mathbf{V}_s=U\mathbf{i}+V\mathbf{j} = \textrm{const.}
	\end{equation}
	
	Let $\Delta x = \Delta y =d$; $x=jd$; $y=kd$; $t=n \Delta t$ and apply CTCS as before:
	
	\begin{equation}
		{A_{j,k}^{n+1}-A_{j,k}^{n-1}} = - \frac{ U \Delta t}{d} \left( A_{j+1,k}^{n} - A_{j-1,k}^{n}  \right) - \frac{ V \Delta t}{d} \left( A_{j,k+1}^{n} - A_{j,k-1}^{n}  \right) 
		\label{adv-CTCS-2D}
	\end{equation}
	
	Just as before, we apply a solution following the Von Neumann method: ${A_{j,k}^{n}=B^{n \Delta t} e^{i pjd + qkd }}$ and we simplify by cancelling common terms, just as we did in the 1D case, leading to:
	
	\begin{equation}
		B^{n+1}  =  B^{n-1}  - \frac{2 i \Delta t}{d}  \left ( U \sin pd + V \sin qd \right) B_n
		\label{adv-CTCS-2D-solution}
	\end{equation}
	
	We can solve by defining $D=B^{n-1}$, re-writing (\ref{adv-CTCS-2D-solution}) in matrix form and finding its eigenvalues. This is quite similar to what is done on page 127 of Haltiner and Williams to solve the 1D case we treated before, which results in a quadratic equation ((5-55), page 127) with coefficients identical to what we found previously (equation (\ref{adv-CTCS-1D-eigenvalues}), else see (5-16) in H\&W, page 112).

\subsection{The 2D eigenvalues for stability and the 2D CFL}

	The eigenvalues are:
	
	\begin{equation}
		\lambda = -i \frac{ \Delta t}{d}  \left ( U \sin pd + V \sin qd \right) \pm \sqrt { 1-  (\frac{ \Delta t}{d})^2  \left ( U \sin pd + V \sin qd \right) ^2 }
	\end{equation}
	
	which will have (the desired) magnitude of $1$ provided that: 
	\begin{equation}
		\frac{ \Delta t}{d}  \left | U \sin pd + V \sin qd \right | \leq 1
	\end{equation}
	
	At this time we consider $U$ and $V$ as projections of the vector $\mathbf{V}_s$ onto the $x$ and $y$ directions, making use of the "angle of the wind", or \emph{direction of the wave front} $\theta$: $U=V_s \cos \theta$ and $V=V_s \sin \theta$, which gives us:
	
	
	\begin{equation}
		\frac{ V_s\Delta t}{d}  \left | \cos \theta \sin pd + \sin \theta \sin qd \right | \leq 1
	\end{equation}
	
	Which is the wind direction $\theta$ that is most likely to violate CFL? Since the wave numbers $p,q$ are independent, we can choose the maximum value, $1$, for each of the two terms: $\sin pd$ and $\sin qd$, and we are left with the task of maximising the remaining sum: $\cos \theta + \sin \theta$, which is $\sqrt{2}=0.707$, corresponding to $\theta=\pi/4=45^o$.
	
	So, \textbf{for 2D CTCS, the CFL condition} that will allow us to avoid computational instability under all circumstances (even with wind blowing at an angle of $45^o$) is: 
	\begin{equation}
		V_s \Delta t \sqrt{2} / d \leq 1 \textrm{ or } V_s \Delta t \leq 0.707 d 
	\end{equation}
	
\subsection{2D CFL: graphical interpretation}

	In summary, we have seen that in 2D the maximum value of $\Delta t$ required for computational stability is almost 30\% smaller than for the 1D case, other things being equal. In order to understand this, consider the two figures below: on the left is the grid as seen normally, while on the right is the grid as seen when aligned with the progressing wavefront, at an angle of $45^o$.
	A wave propagating at that angle (e.g from the SW to the NE) encounters an effective distance of $d/\sqrt{2}$ between gridpoints. So, the CFL stability criterion says that the wave cannot move more than the effective distance between gridpoints ($d/\sqrt{2}$ at its shortest) during time $\Delta t$, without incurring computatational instability.
	
	\medskip
	
%	\begin{minipage}{0.45\textwidth}
		\begin{figure}
			\includegraphics[trim={5.25cm 0.cm 6.25cm 1.25cm},clip,width=0.75\textwidth]{Figures/2D_CFL_drawing}
		\end{figure}
%	\end{minipage}
	
%	\begin{minipage}{0.45\textwidth}
		\begin{figure}
			\includegraphics[trim={4.25cm 0.cm 7.25cm 4.5cm},clip,width=0.75\textwidth]{Figures/2D_CFL_drawing_rotated}
		\end{figure}
%	\end{minipage}
	

\subsection{Readl world applications: CFL in current GCMs}
	
	In lectures 4,5 we started to learn about Arakawa-sensei's staggered grids, and I showed you my current C-grid configuration, for the UM's EndGame dynamical core. \\
	
	\medskip
	
	In the last row near the poles, my $\Delta x \approx 5$km GCM has a grid spacing of 3.1m. Given a "worst case scenario" for wave propagation, what should my time step be, according to the 1D ("optimistic!") CFL criterion? Please compute that now, and raise your hand as soon you have the answer.\\
	
	\medskip
	
	Here is my real time step: 
	
	\begin{figure}
		\includegraphics[width=0.75\textwidth]{Figures/Stevens_Table5}
		\tiny{\caption{DYAMOND GCMs, with typical $\Delta x \approx 5$km, from Stevens et al., 2019}}
	\end{figure}
	
	
	How can such magic be possible?
	

\subsection{CFL and coupled equations: where does $\Delta t$ matter?}

	Let us not confuse the advection process with the other processes, described in parametrizations, or with the coupling between the equations. 
	
	Also, remember that the time step may be set by the 2D CFL, but the vertical grid spacing is very likely far smaller, meaning that we must use a different numerical scheme in the vertical.
	
	As an example, take a look at our SWEs:
	\begin{eqnarray*}
		\frac{u^{n+1}-u^{n-1}}{2\Delta t}-fv^n+\frac{g}{2}
		\left( \partd{h}{x}^{n+1} +\partd{h}{x}^{n-1} \right) & = & 0 \\
		\frac{v^{n+1}-v^{n-1}}{2\Delta t}+fu^n+\frac{g}{2}
		\left( \partd{h}{y}^{n+1} +\partd{h}{y}^{n-1} \right) & = & 0 \\
		\frac{h^{n+1}-h^{n-1}}{2\Delta t}+\frac{H}{2}
		\left( \partd{u}{x}^{n+1} +\partd{u}{x}^{n-1}
		+\partd{v}{y}^{n+1} +\partd{v}{y}^{n-1} 
		\right) & = & 0. 
	\end{eqnarray*}

\subsection{Hurricane and Typhoon simulation in climate GCMs: frequency}

	\begin{center}	
		\includegraphics[width=0.8\textwidth]{Figures/Zarz1}
	\end{center}
	

\subsection{Hurricane and Typhoon simulation in climate GCMs: dynamics and physics}

	\begin{center}	
		\includegraphics[width=0.8\textwidth]{Figures/Zarz2}
	\end{center}
	



\section{Practical considerations: stability analysis in 1D and 2D for Project 2}

The goal of the second project is to solve the Stommel model on the finite difference C-grid with a FBT time integration scheme. The Stommel model is the simplest geophysical flow model able to represent a wind-driven circulation in a closed basin with a western boundary layer. The model equations read:
\begin{eqnarray}
	&&\frac{\partial \mathbf{u}}{\partial t} + (f_0+\beta y) \mathbf{e}_z \times
	\mathbf{u} = -g \bnabla \eta - \gamma \mathbf{u} + \frac{\btau^{\eta}}{\rho h},\\
	&&\frac{\partial \eta}{\partial t} + h \bnabla \cdot \mathbf{u} = 0,
\end{eqnarray}
where $f_0$ is the reference value of the Coriolis parameter, $\beta$ is the reference value of the Coriolis parameter first derivative in the $y$-direction, $\gamma$ is a linear friction coefficient, $\rho$ is the homogeneous density of the fluid and $\btau^{\eta}$ is the wind stress acting on the surface of the fluid. The model solutions look like these:
\begin{center}
	\includegraphics[width=0.9\textwidth]{Figures/sol_project2.eps}
\end{center}


\subsection{How do we decide on grid, grid spacing, time step?}

Remember that we started with the CFL criterion in 1D:

\begin{equation}
	\mu = U \frac {\Delta t}{\Delta x} \le 1
\end{equation}

There are now several ways in which we can meet the stability criteria for our particular 2D SWE problem. Thinking a bit harder about these decisions seems to lead to some circular reasoning: where do we start? \textbf{Are there tradeoffs?} 
\medskip

Questions that we must consider when we choose the numerical method to solve our problem:
\begin{enumerate}
	\item What is the "worst case scenario" distance relevant to CFL for SWEs in 2D?
	\item How many grid points are the minimum required to resolve anything?
	\item What is the size of our domain?
	\item What is the size of the phenomenon we are trying to simulate?
	\item What do all these decisions end up causing in terms of the physical realism of the solutions? For instance, compromising the accuracy of solution for one wave type.
	
\end{enumerate}


\subsection{Numerical analysis of the 1D "upstream" advection equation}
\subsection{There is more to choosing $\Delta t$ and $\Delta x $}

It is not a given that going for a small $\mu$ is always the best possible decision.
For instance, let us start with the advection equation and let us choose the \textbf{upstream scheme}, which we know is \textbf{dissipative}. That means, each time step we lose a bit of the signal.


If we increase the resolution of our model, making the time step smaller, we may think that we will obtain a better solution. However, this is not necessairly true for every $\mu$.

\medskip
Consider a domain of size $D$, resolved by a number $J$ of $\Delta x$ points, in which we are solving the advection equation:

\begin{equation}
	\frac {A_j^{n+1}-A_j^{n}}{\Delta t} +c \left(  \frac {A_j^{n}-A_{j-1}^{n}}{\Delta x}  \right) = 0
	\label{adv-upstream}
\end{equation}

$D=J \Delta x$ and {\bf every time that we decrease $\Delta x$ we increase $J$, so that the domain size D does not change}. For a signal that has wavenumber $k$ in the $x$ direction:

\begin{equation}
	k \Delta x = \frac {k D}{J}
\end{equation}


\subsection{Stability analysis of the upstream advection equation}


If you remember your Von Neuman analysis, \textbf{the amplification factor, $\lambda$} for upstream advection (equation \ref{adv-upstream})	has this form:

\begin{equation}
	|\lambda|^2 = 1+2\mu(\mu-1)(1-\cos k \Delta x) = 1+2\mu(\mu-1)[1-\cos ( \frac {k D} {J})]
\end{equation}

which tells us that $\lambda$ depends on both wavenumber ($k$) and our Courant number ($\mu$). 
In order to maintain computational stability, \emph{we keep $\mu$ fixed as $\Delta x$ decreases}, and that decision limits our time step:

\begin{equation}
	\Delta t = \frac {\mu \Delta x}{c} = \frac {\mu D}{cJ}
\end{equation}

Since we know the velocity of the fluid, $c$, we also know how long it takes for it to cross the entire domain: $T=\frac {D}{c}$. If we choose to carry out the time integration in $N$ steps:

\begin{equation}
	N= \frac {T}{\Delta t} = \frac {D}{c \Delta t} = \frac {D}{\mu \Delta x} = \frac {J}{\mu}
\end{equation}



\subsection{How damped is the signal after N time steps?}
The total amount of damping that "accumulates" throughout the time integration ($N$ time steps) is given by:

\begin{equation}
	|\lambda|^N=(|\lambda^2|)^{N/2} = \left \{ 1 -2 \mu (1-\mu) \left[ 1 - \cos ( \frac {k D} {J}) \right] \right \} ^{\frac {J}{2 \mu}}
\end{equation}

This relationship says that $\lambda$ depends on $J$ (the resolution) in two different ways, so that making $\Delta x$ smaller (increasing the resolution) can be good or bad. 

\medskip
Which is it?
Let us pick a wavelength half the domain size, so that $kD=4 \pi$. This causes the $\cos$ factor in the equation to approach 1, which \textbf{weakens the damping}; on the other hand it also causes the exponent to increase, which \textbf{strenghtens the damping}. This is shown in the figure, for two values of $\mu$.

%	\begin{minipage}{0.4\textwidth}
	\begin{figure}
		\includegraphics[trim={0cm 0cm 0cm 0cm},clip,width=5.cm]{/Users/vidale/Projects/Teaching/MTMW14/Figures/Resolution_damping.png}
	\end{figure}
	%	\end{minipage}
\hfill
%	\begin{minipage}{0.5\textwidth}
	Overall, \textbf{increasing the resolution, $J$, is good}: $\lambda$ tends to 1 on the right hand side, even though we take more time steps, $N$, to complete the integration. However, if we fix $J$ and decrease $\mu$, by decreasing the time step $\Delta t$, the damping increases and \textbf{the solution will be less accurate}. 
	
	%	\end{minipage}


\subsection{Recommendations on choosing $\Delta t$ and $\Delta x $}

%	\begin{minipage}{0.4\textwidth}
	\begin{itemize}
		\item perform numerical analysis, speficically for each scheme
		\item \textbf{for the upstream scheme}, the amplitude error can be minimised by using the largest stable value of $\mu$.
		\item that is, do not exaggerate in making $\mu$ much less than 1: this decision impacts both cost and quality of the solutions
		\item in other words, \underline{it pays off to live dangerously}!
	\end{itemize}
	%\end{minipage}
	\hfill
	%	\begin{minipage}{0.5\textwidth}	
		\begin{figure}
			\includegraphics[trim={0cm 0cm 0cm 0cm},clip,width=7.cm]{/Users/vidale/Projects/Teaching/MTMW14/Figures/Resolution_damping.png}
		\end{figure}
		%	\end{minipage}


\section{Appendix: Why advection by the grid point method is imperfect}

\subsection{The Advection Equation}

\begin{figure}[h]
	\begin{center}
		\includegraphics[scale=0.35]{/Users/vidale/Projects/Teaching/MTMW14/Figures/advection.pdf}
	\end{center}
	\caption{Temperature at a point increases when wind blows from the warm side.}
	\label{fig:ken}
\end{figure}


In Figure~\ref{fig:ken}, the wind is bringing air from the $-x$
direction. If the air `upstream' is warmer than the air `downstream',
the observer will see the temperature increasing. The rate of change of
temperature observed must depend on both the magnitude of this
gradient and the speed at which the air is moving, {\em i.e.,}
\begin{equation}
	\frac{\partial T}{\partial t} =- u \frac{\partial T}{\partial x}.
	\label{advectioncont}
\end{equation}
where the left hand side means the rate of change at a given position
$x$ and the right hand side is called the {\em advection term}. $u$ is
the wind component and $\partial T/\partial x$ means the gradient in
the $x$-direction at a given time. 

We can solve this analytically, but the question is what happens when we solve it numerically. As an example, let us look at CTCS.

\subsubsection{Computational modes: The CTCS advection scheme}

Let us try using the centred difference in time and the
centred difference in space (CTCS for short):
\BEQ \frac{T_{j}^{n+1}-T_{j}^{n-1}}{2\Delta t} + u \frac{T_{j+1}^{n} - T_{j-1}^{n}}{2\Delta x} = 0 \EEQ
We can rewrite this equation to get:
\BEQ T_{j}^{n+1} = T_{j}^{n-1} - \frac{u\Delta t}{\Delta x} \left[ T_{j+1}^{n} - T_{j-1}^{n} \right] \label{ctcsmodel} \EEQ
The quantity $u\Delta t / \Delta x$ is dimensionless.  It appears so often in
numerical schemes for the advection equation that it has its own name --- the
\emph{Courant number}:
\BEQ \alpha = \frac{u\Delta t}{\Delta x} .\EEQ

The von Neumann stability analysis for the CTCS scheme is most easily
done using complex notation for the trial solution, 
\BEQ T_j^n=e^{ikx_j}\;\; ; \;\;\;\; T_j^{n+1}=ST_j^n \EEQ
where $S=Ae^{i\phi}$ expresses 
the change in amplitude and phase over one time-step.



When we substitute the trial solution $T_j^n=e^{ikx_j}$ into a CTCS scheme, we shall end up with three terms that express what temperature is at points: \BEQ j-1, \;\;\;\ j \;\;\;\ \textrm{   and  } \;\; j+1 \EEQ
These three terms are: \BEQ e^{ik(x_j-\Delta x)},  \;\;\;\;\; e^{ikx_j} \;\;\;\;\; \textrm{   and   } \;\; e^{ik(x_j + \Delta x)}\EEQ

The common term $e^{ikx_j}$ cancels out, so that plugging the trial solution into the model (\ref{ctcsmodel}) yields:
\BEQ
S=S^{-1}-\alpha\left[ e^{ik\Delta x}-e^{-ik\Delta x} \right]
\EEQ

which can be re-arranged as a quadratic equation
for the complex factor, $S$, by remembering Euler's formula: $\frac {e^{i\theta} - e^{-i \theta}}{2i}= sin(\theta)$:
\BEQ
S^2+S\alpha 2i\sin k\Delta x -1=0
\EEQ

which has two roots:
\BEQ
S_{\pm}=-i \alpha \sin k\Delta x\pm \sqrt{1-\alpha^2 \sin^2 k\Delta x}
\label{asoln}
\EEQ

If the Courant number $|\alpha | \le 1$, the number under the root must
be positive since $\sin^2 k\Delta x \le 1$. 

{\em Exercise: Show that when $|\alpha | \le 1$ the magnitude of the
	amplification factor $|S|^2=S^* S=1$, meaning that the scheme is
	stable. However, when $|\alpha | > 1$ the $S_{-}$ solution is
	unstable.}

This is an improvement on the FTCS scheme, which was unconditionally
unstable. This is the single most important result in the numerical
modelling of fluid flows.  It is called the Courant-Friedrichs-Lewy
(CFL) condition.  What it boils down to is that, for a given
grid-spacing $\Delta x$, the time step must satisfy \BEQ \Delta t <
\frac{\Delta x}{u} . \EEQ The physical interpretation of this condition
is that \emph{the scheme is unstable if fluid parcels move more than
	one gridbox in one timestep}.

The solution with +ve real part in (\ref{asoln}) does not change sign
every timestep and is called the {\em physical mode}. However, the
solution with the -ve square root alternates sign every step which is
unphysical behaviour and is called the {\em computational mode}. The
relative amplitudes of the two modes is determined by projection from
the initial conditions. The implication is that although the CTCS
scheme is stable for $|\alpha | \le 1$ we can expect unphysical
oscillations that are worst at the grid-scale (where $\sin k\Delta
x=1$).

\begin{figure}
	%\begin{center}
	\mbox{\scalebox{0.05}{\includegraphics{Figures/adv-ctcs.png}}
		\scalebox{0.05}{\includegraphics{Figures/adv-upstream.png}}}
	%\end{center}
	\caption{\textsl{Comparing numerical solutions for the advection of a top-hat distribution by uniform flow. Left: CTCS solution exhibits unphysical undershoots and overshoots. Right: Upstream scheme remains monotonic.}}
	\label{fig:ctcs}
\end{figure}

%%%%%%%%%%%%%%%%%%%%%%%%%%%%%%%%%%%%%%%%%%%%%%%%%%

%%%%%%%%%%%%%%%%%%%%%%%%%%%%%%%%%%%%%%%%%%%%%%%%%%
% SWES
\newpage	
%%%%%%%%%%%%%%%%%%%%%%%%%%%%%%%%%%%%%%%%%%%%%%%%%%
\newpage
\chapterimage{2613-1477-max.jpg} % Chapter heading image
\chapter{The Shallow Water Equations}
%%%%%%%%%%%%%%%%%%%%%%%%%%%%%%%%%%%%%%%%%%%%%%%%%%%%

\vspace{1em}


\section{The governing equations}

In his lecture on Chaos, Tim Palmer explains how the Navier Stokes equations, describing the motion of a Newtonian fluid (loosely, a viscous fluid); the Euler equations describe the motion of a fluid with no viscosity. To this day have no analytical solutions (see the \href{https://en.wikipedia.org/wiki/Millennium_Prize_Problems}{Millenium Prize}). 

Predicting the behavior of geophysical fluids only requires a handful of equations, fundamentally Newton's second law, the ideal gas law, the first law of thermodynamics, and the continuity equation. Additionally, if we are imposing forcings, we will end up using Planck's law, as well as a number of empirical formulations describing phenomena such as turbulence, or the behavior of live organisms, such as bacteria, or cells in live tissues.

Leaving aside the empirical aspects of "Earth System" processes, that is, the parametrisations that we shall be discussing in later chapters, the main equations we are concerned with in prediction are:

\subsection{Equation of Motion}

Newton's second law for a fixed or inertial frame of reference:

\begin{equation}
	\frac{d_a \bm{V}_a}{dt}=\bm{M}
\end{equation}

where $\bm{M}$ represents the vector sum of all forces per unit mass. We are normally interested on fluid motion on our planet, which is rotating, thus not an inertial frame of reference. Assuming that the rotation is expressed by the angular velocity $\bm{\Omega}$, we can express the equation of motion in terms of a relative term and a rotation term:

\begin{equation}
	\bm{V}_a=\bm{V} + \bm{\Omega} \times \bm{r}
\end{equation}

where $\bm{r}$ is the position vector of a particle of fluid as measured from the origin of the Earth's centre.

After some substitutions, we can rewrite our equation of motion to relate the inertial system ($\frac{d_a}{dt}$) and the rotating system ($\frac{d}{dt}$):

\begin{equation}
	\frac{d_a \bm{V}_a}{dt}=   \frac{d \bm{V}_a}{dt}    + \bm{\Omega} \times \bm{V}_a
\end{equation}

In terms of $\bm{M}$. the main forces we are concerned with are the pressure force, gravitation  ($\bm{g}_a$) and friction ($\bm{F}$), so that, after further simplification (and incorporating the centrifugal force into $\bm{g}$)

\begin{equation}
	\frac{d \bm{V}}{dt} = \alpha \grad{p} -2 \bm{\Omega} \times \bm{V} + \bm{g} + \bm{F}
\end{equation}

and the second term on the rhs of the equation above is referred to as the \emph{Coriolis force}.

\subsubsection{The hydrostatic approximation}

For large-scale motion of air (see the scale analysis later in the chapter), the atmosphere may be assumed to be in hydrostatic equilibrium: vertical acceleration and the vertical component of the Coriolis force may be omitted.
\begin{eqnarray}
	\frac{d \bm{V}_H}{dt} = \alpha \grad{p} - \bm{f} \times \bm{V}_H + \bm{g} + \bm{F}_H\\
	0 = -\alpha \frac{\partial p}{\partial z} - g
\end{eqnarray}
where the subscript $H$ indicates horizontal quantities and $f=2\bm{\Omega}\sin{\phi}$ is the Coriolis parameter, that is the vertical component of the Earth's velocity at latitude $\phi$.

\subsection{Continuity}

Conservation of mass implies that the mass inflow (the difference in the flux across all faces of the volume of fluid) needs to be balanced by the change of mass inside the fluid:

\begin{equation}
	\frac{d \rho}{dt} =  -\bm{\nabla} \cdot (\rho \bm{V}) = - \rho \bm{\nabla} \cdot {\bm{V} }  -\bm{V} \cdot \grad{\rho}
\end{equation}

where $\rho$ is density. Another form of this is:

\begin{equation}
	-\frac{1}{\rho} \frac{d \rho}{dt} =   \frac{1}{\alpha} \frac{d \alpha}{dt}     = \bm{\nabla} \cdot  \bm{V} 
\end{equation}


where the specific volume $\alpha = 1/\rho$.

\subsection{Equation of state}

The most typical expression used in atmospheric physics is Boyle's and Charles' law:

\begin{equation}
	p\alpha=RT
	\label{ideal_gas}
\end{equation}


\subsection{Thermodynamics}

If we omit chemical, electrical and magnetic effects, the first law of thermodynamics for the atmosphere can be written as:

\begin{equation}
	Q=\frac{dl}{dt} + W
\end{equation}

where $Q$ is the rate at which heat is added, $W$ is the rate at which work accomplished by the gas on its environment by expansion and $dl/dt$ is the rate of change of internal energy of the gas. For a perfect gas, $dl/dt=c_vdT/dt$, where $c_v$ is the specific heat at constant volume and $W=p d\alpha/dt$. If we substitute:

 \begin{equation}
 	Q= c_v \frac{dT}{dt} + p\frac{d\alpha}{dt} 
 \end{equation}
If we now use the ideal gas law (\ref{ideal_gas}), we end up with a (dry) form that is quite familiar to atmospheric scientists:

 \begin{equation}
 	Q= c_p T \frac{d(\ln \theta)}{dt}
 \end{equation}

where $\theta = T{(p_0/p)}^k$ is the potential temperature, $k=R/c_p$ and $p_0=1000mb$ is the reference pressure.


\subsection{Primitive Equations}
The collection of momentum, continuity, state and thermodynamics come together, usually under the hydrostatic approximation, to form the \emph{Primitive Equations}, first written out in a form we use to this day by Bjerkness.

\begin{eqnarray}
		\frac{d \bm{V}_H}{dt} = \alpha \grad{p} - \bm{f} \times \bm{V}_H + \bm{g} + \bm{F}_H\\
	0 = -\alpha \frac{\partial p}{\partial z} - g \nonumber \\
	-\frac{1}{\rho} \frac{d \rho}{dt} =   \frac{1}{\alpha} \frac{d \alpha}{dt}     = \bm{\nabla} \cdot  \bm{V} \nonumber \\
	p\alpha=RT \nonumber \\
	c_p T \frac{d(\ln \theta)}{dt} = Q \nonumber
\end{eqnarray}

which is the minimal set we solve in a typical weather or climate model.


\section{Approximating the governing equations in the study of waves}

Geophysical fluids support a number of waves, with different amplitudes, phase speeds, directions of propagation. We shall initially look into some of the most common ones, and later on focus on just two of them: gravity inertia waves and Rossby waves. For now, let us look at how different approximations lead us to discovering the properties of a variety of waves supported by geophysical fluids.

We start with a simplified set in two dimensions, $x$ and $z$:

\begin{eqnarray}
	\frac{d u}{dt} + \alpha \frac{\partial p}{\partial x} = 0\\
		\frac{d w}{dt} + \alpha \frac{\partial p}{\partial z} + g =0 \nonumber \\
	\alpha \frac{d p}{d t} +p \gamma \frac{d \alpha}{dt}  = 0 \nonumber \\
  \alpha \bm{\nabla} \cdot  \bm{V}  -  \frac{d \alpha}{dt}     = 0 \nonumber
\end{eqnarray}

where $\gamma = c_p/c_v$ and $\alpha = 1/\rho$.

This system can be linearised using the perturbation method, that is, velocity comprises the sum of a constant current $U$ and of a perturbation $u'$; also the average thermodynamic variables are in hydrostatic balance.

\begin{eqnarray}
	\frac{d u'}{dt} + U \frac{\partial u'}{\partial x} + \bar{\alpha} \frac{\partial p'}{\partial x} = 0\\
	\delta_1 (\frac{\partial w'}{\partial t} + U \frac{\partial w'}{\partial x} ) + \bar{\alpha} \frac{\partial p'}{\partial z}  - \frac{g \alpha'}{\bar{\alpha}} =0 \nonumber\\
	\bar{\alpha}  (\frac{\partial p'}{\partial t} + U \frac{\partial p'}{\partial x} ) -g w'  +\bar{p}\gamma  ( \frac{\partial \alpha'}{\partial t} + U  \frac{\partial \alpha'}{\partial x} + w'  \frac{\partial \bar{\alpha}}{\partial z}) = 0 \nonumber \\
    (\frac{\partial u'}{\partial x} + \frac{\partial w'}{\partial z} ) \bar{\alpha} - \delta_2  ( \frac{\partial \alpha'}{\partial t} + U  \frac{\partial \alpha'}{\partial x}) - w'  \frac{\partial \bar{\alpha}}{\partial z}  = 0  \nonumber
\label{primitive_linear}
\end{eqnarray}

where $\delta_1$ and $\delta_2$ are handy coefficients that will enable us to suppress certain vertical acceleration and compressibility terms, by taking on the values of zero or one.

\subsection{Sound waves}

These are compression waves that can be isolated by setting $g=0$; ~~ $\delta_1=\delta_2=1$ and by letting $\bar{p}$ and $\bar{\alpha}$ be constants. 
We can then stipulate that all perturbation quantities are harmonic in $x,z,t$, with constant coefficients, in this form:

\begin{eqnarray}
u'=Se^{i (\mu x+kz-\nu t)}  ~~~~~~ w'=We^{i (\mu x+kz-\nu t ) }  \\
p'=Pe^{i (\mu x+kz-\nu t)}  ~~~~~~ \alpha'=Ae^{i (\mu x+kz-\nu t ) } \nonumber  
\end{eqnarray}

\begin{definition}
	The phase speed is the velocity of the phase lines $(\mu x+kz = \mathtt{constant})$ in the normal direction, and is related to frequency in this way:
	\begin{equation}
		c = \nu \mathcal{L} / 2 \pi
	\end{equation}
	where $\mathcal{L} $ is the wavelength normal to the phase lines. 
\end{definition}

Substitution into \ref{primitive_linear}, solving for the determinant of the resulting matrix leads eventually to an expression for the phase speed of sound waves:

\begin{equation}
	c = \pm \sqrt{\gamma R \bar{T}}
\end{equation}

\begin{exercise}
Try substituting typical values to see what the magnitude of this phase speed is.
\end{exercise}


\subsection{Sound waves and internal gravity waves}
If, instead of considering the average pressure and density to be constant, we impose the variation with height that is implied by the presence of gravity, the hydrostatic equation and state equations will have solutions of this type:

\begin{eqnarray}
	\bar{p}= \bar{p}(0) e^{-z/H}  ~~~~~~ \bar{\alpha}= \bar{\alpha}(0) e^{z/H}  
\end{eqnarray}

where $H = R\bar{T}/g$ is called the scale height. Substitution leads to a new set of harmonic functions:

\begin{eqnarray}
	u'=S\bar{\alpha}^{1/2} e^{i (\mu x+kz-\nu t)}  ~~~~~~ w'=W\bar{\alpha}^{1/2} e^{i (\mu x+kz-\nu t ) }  \\
~~	p'=P\bar{\alpha}^{1/2} e^{i (\mu x+kz-\nu t)}  ~~~~~~ \alpha'=A\bar{\alpha}^{3/2} e^{i (\mu x+kz-\nu t ) } \nonumber  
\end{eqnarray}

which, when substituted into \ref{primitive_linear}, leads in turn to a new matrix equation, which, after some algebra and dropping some terms of negligible magnitude, has this general solution:

 \begin{equation}
	\delta_1 \delta_2 \nu^4 -\gamma RT(k^2 + \mu^2\delta_1)\nu^2 + \mu^2 \frac{\gamma R \bar{T} g}{\bar{\theta}}\frac{\partial \bar{\theta}}{\partial z} = 0
	\label{GIS_sol.ution}
\end{equation}

This equation has four roots, of which two are sound waves and two are a pair of internal gravity waves. It is possible to isolate waves by specific settings of $g$, which eliminates the gravity waves, or by different settings of $\delta_1$ and $\delta_2$, to prevent sound waves, which will give pure gravity waves.

The phase speed will be:

\begin{equation}
	c = \pm \frac{\mu}{\mu^2+k^2} (\frac{g}{\bar{\theta}} \frac{\partial \bar{\theta}}{\partial z} )^{1/2}
\end{equation}

These oscillations are stable if the lapse rate is subadiabatic ($\frac{\partial \bar{\theta}}{\partial z}>0$), but unstable if the lapse rate is superadiabatic.

Depending on the relative size of the disturbance in the horizontal versus vertical direction, that is the relative size of the wave numbers, these waves will propagate preferentially in one direction. For instance, if the vertical scale is much smaller than the horizontal scale, we go back to the hydrostatic approximation, $|delta_1$ can be set to zero and the phase speed is simplified accordingly and we can impose solutions that propagate horizontally with a vertical variation.

The set \ref{primitive_linear} supports one more set of wave solutions for an isothermal fluid: these are horizontally propagating waves with the speed of sound ($c=\pm \sqrt{\gamma R \bar{T}}$) and they are called \emph{Lamb waves}, very similar to external gravity waves, which will be discussed later for the shallow water equations.

\section{Shallow atmosphere approximation} 

Also known as the {\em traditional approximation} since it also
applies to the oceans. The deepest motions in the atmosphere have
depth scales, $H\approx\,$20km. The oceans are even shallower. Therefore
$H/a< 1/300$. The shallow limit $H/a\ll 1$ gives:

\begin{itemize}
	\item
	radial coordinate $r\rightarrow a$ (i.e., constant)
	
	\item
	$\partd{}{r}\rightarrow \partd{}{z}$ where $z$ is height relative to Earth's surface
	
	\item
	In order to retain same form for kinetic energy and zonal angular
	momentum equations, must drop terms containing $w$ in horizontal
	momentum equations and the horizontal component of Coriolis
	acceleration.
\end{itemize}

The resulting {\em primitive equations} (PEs) are the basis of almost
all atmosphere and ocean {\em general circulation models} (GCMs)
except the Met Office Unified Model which does not make a shallow
atmosphere approximation.

\begin{itemize}
	\item
	Biggest errors are in Tropics associated with neglect of horizontal
	component of Coriolis acceleration.\footnote{\BTi White, Hoskins, Roulstone and Staniforth (2005) Consistent approximate models of the global atmosphere: shallow, deep, hydrostatic, quasi-hydrostatic and non-hydrostatic. \emph{Quart. J. Roy. Met. Soc.}, \textbf{131},
		2081-2107.
		\ETi }
	
	\item
	Effects of spherical geopotential approximation are unknown.\footnote{\BTi White, Staniforth and Wood (2008) Spheroidal coordinate systems for modelling global atmospheres. \emph{Quart. J. Roy. Met. Soc.}, \textbf{134},
		261-270.
		\ETi }
\end{itemize}


\subsection{Further approximations} 

Further approximations are usually based on {\em scaling} where
the typical spatial and length scales associated with motions of
interest are assumed. For example:

\begin{definition}[Planar approximation]

{\bf $L/a \ll 1$} Can use local Cartesian coordinates where $(dx,dy)=(a \cos\phi_0
\,d\lambda, a\,d\phi)$. 

Can drop spherical metric terms from PEs.
\end{definition}


\begin{definition}[Anelastic approximation]
{\bf $\Delta \rho /\rho_0 \ll 1$} Mass conservation equation becomes $\nabla.(\rho_r {\mathbf u})=0$
where $\rho_r(z)$ is a reference density.
Filters sound waves.
\end{definition}

\begin{definition}[Boussinesq approximation ]
{\bf $\Delta \rho /\rho_0 \ll 1$ and $H\ll H_{\rho}$} Huge density height scale, $H_{\rho}$, implies $w \partd{\rho_r}{z}\ll
\rho_r \partd{w}{z}$. In practice, the density variation is only important in the buoyancy term of the (vertical) momentum equation, $\rho g$. The \emph{continuity equation}:
\begin{equation}
	\frac{1}{\rho} \lagd{\rho}{t} + \nabla \cdot \mathbf{u} = 0
	\label{continuity-equation}
\end{equation}

ends up simplified to its incompressible ($\rho_r \approx$ constant) form: $\nabla \cdot \mathbf{u} = 0$. 

\end{definition}

\begin{definition}[Hydrostatic balance]
{\bf $H/L \ll 1$} Vertical momentum equation reduces to:
\begin{equation}
	\frac{1}{\rho}\partd{p}{z}=-g
	\label{hydrostatic}
\end{equation}
Filters vertically-propagating sound waves, but can support Lamb waves, and can distort gravity waves if $L_x \approx L_y$.

If we also want to filter out gravity waves, we must additionally set the local time rate of change of the divergence field to zero.
\end{definition}


\subsection{Practical example: filtering the equations to remove unwanted scales of motion} 

The problem with our Euler equations is that they support all types of motion, some of which may be unwanted, depending on what scientific questions we are trying to answer. 

\medskip

{\bf Example: sound waves.}\\
The governing equations of compressible fluid dynamics contain acoustic waves. In low Mach number flows, typical of atmospheric or oceanic conditions, acoustic waves play no essential role and yet they severely constrain the time step in numerical modeling. Thus, there has long been an interest in approximating the governing equations to eliminate or "filter out" acoustic waves. 

\medskip

\begin{tabular}{|c|c|c|c|c|}
	\hline
	Wave & horizontal & propagation & propagation  & time step \\
	type & dimension & speed & direction & required \\
	\hline
	Sound &  &  &  & \\
	\hline
	Short Gravity &  &  &  & \\
	\hline
	Long Gravity &  &  &  & \\
	\hline
	Rossby &  &  & & \\
	\hline
	Kelvin &  &  &  & \\
	\hline
\end{tabular}

\section{Simplifying the PEs}
\subsection{PEs on the sphere}
In an Eulerian framework, our momentum equations end up looking like this:
\begin{eqnarray}
	\frac {\partial u}{\partial t} + \left(\mathbf{v}\cdot\nabla\right)u=-\frac{1}{\rho} \frac{\partial p}{\partial x}+F_x \\
	\frac {\partial v}{\partial t} + \left(\mathbf{v}\cdot\nabla\right)v=-\frac{1}{\rho} \frac{\partial p}{\partial y}+F_y \\
	\frac {\partial w}{\partial t} + \left(\mathbf{v}\cdot\nabla\right)w=-\frac{1}{\rho} \frac{\partial p}{\partial z}-g+F_z
\end{eqnarray}

We must now consider the fact that we are studying fluids (atmosphere, ocean) on our planet, which is approximately a sphere of radius $a$, rotating with angular velocity $\Omega$,

\begin{figure}[h!]
	\begin{center}
		\includegraphics[trim={0cm 2cm 0cm 2cm},clip,width=7.cm]{/Users/vidale/Projects/Teaching/MTMW14/"2016 module"/RotatingEarthDiagram.pdf}
		\label{fig:Spherical}
	\end{center}
	\caption{Schematic of spherical coordinate system}
\end{figure}

	In the spherical coordinate system we use $\lambda$ (longitude), $\theta$ (latitude) and r, to replace $x,y,z$, where $x=r \cos{\theta} \lambda$; $y=r\theta$; $z=r-a$, so that:
	\begin{eqnarray}
		u=\frac {dx}{dt}=r \cos{\theta} \frac {d \lambda}{d t} ; ~~~~~~~~~~~ v=\frac {dy}{dt}= r\frac {d\theta}{dt}= ; ~~~~~~~~~~~ w=\frac {dz }{dt }
	\end{eqnarray}
	are how we write our spatial derivatives on the Sphere.	 The momentum equations in spherical coordinates are then given by:

\begin{eqnarray}
	\frac {\partial u}{\partial t} + \left(\mathbf{v}\cdot\nabla\right)u - \frac{u \tan \theta}{a} v  + \frac{w}{a}u =-\frac{1}{\rho} \frac{\partial p}{\partial x}+F_x \\
	\frac {\partial v}{\partial t} + \left(\mathbf{v}\cdot\nabla\right)v + \frac{u \tan \theta}{a} u  + \frac{w}{a}v =-\frac{1}{\rho} \frac{\partial p}{\partial y}+F_y \\
	\frac {\partial w}{\partial t} + \left(\mathbf{v}\cdot\nabla\right)w - \frac{u^2 + v^2}{a} =-\frac{1}{\rho} \frac{\partial p}{\partial z}-g+F_z
\end{eqnarray}


\subsection{PEs on the Sphere before scale analysis}

The full form of the momentum equations in a rotating spherical coordinate system is then:

\begin{eqnarray}
	\frac {\partial u}{\partial t} + \left(\mathbf{v}\cdot\nabla\right)u - \frac{2 \Omega \sin \theta + u \tan \theta}{a} v  + \frac{w}{a}u + w \cdot 2 \Omega \cos \theta =-\frac{1}{\rho} \frac{\partial p}{\partial x}+F_x \\
	\frac {\partial v}{\partial t} + \left(\mathbf{v}\cdot\nabla\right)v + \frac{2 \Omega \sin \theta +u \tan \theta}{a} u  + \frac{w}{a}v =-\frac{1}{\rho} \frac{\partial p}{\partial y}+F_y \\
	\frac {\partial w}{\partial t} + \left(\mathbf{v}\cdot\nabla\right)w - \frac{u^2 + v^2}{a} - u \cdot 2 \Omega \cos \theta =-\frac{1}{\rho} \frac{\partial p}{\partial z}-g+F_z
\end{eqnarray}

where the terms involving $\Omega$ are the Coriolis terms. The symbol $f$ is often used to represent the Coriolis term that appears in both horizontal momentum equations ($2 \Omega \sin \theta$ ). The equation of state and the continuity, thermodynamic and constituent equations are not modified by the effects of rotation.

\subsection{Small quantities, big quantities: Scale Analysis}

Some of the terms in the equation set are very small: do take a look at Table \ref{fig:Scale-Analysis-Tables} and you will see that there are differences amounting to several orders of magnitude. We are tempted to neglect the smaller terms, so that we may end up with a simpler equation set. This is in fact what  was commonly done  in the 1960s, when digital computers were far less powerful than today's computers
Cancelling out the smallest terms, we end up with simpler equations and some important approximations, for instance hydrostatic balance in the vertical momentum equation.

\begin{figure}[h!]
		\begin{center}
	\includegraphics[trim={4cm 1.5cm 0cm 1cm},clip,width=18.cm]{/Users/vidale/Projects/Teaching/MTMW14/"2016 module"/ScaleAnalysisTables.pdf}
		\end{center}
	\caption{Scale Analysis Tables for PE terms in the atmosphere (A) and ocean (O)}
		\label{fig:Scale-Analysis-Tables}
\end{figure}

\subsection{Simplified PEs after scale analysis}

If we make careful use of scale analysis, we can strongly simplify the PEs we want to integrate, by neglecting terms that are small, and do not significantly contribute to balance and/or the time evolution of our prognostic variables. Using the numbers in Table \ref{fig:Scale-Analysis-Tables}, we can, for instance removing terms smaller than $10^{-4}$, we arrive at:

\begin{eqnarray}
	\frac {\partial u}{\partial t} + \left(\mathbf{v}\cdot\nabla\right)u - \frac{2 \Omega \sin \theta + u \tan \theta}{a} v  = -\frac{1}{\rho} \frac{\partial p}{\partial x}+F_x \\
	\frac {\partial v}{\partial t} + \left(\mathbf{v}\cdot\nabla\right)v + \frac{2 \Omega \sin \theta +u \tan \theta}{a} u  = -\frac{1}{\rho} \frac{\partial p}{\partial y}+F_y \\
	0 =-\frac{1}{\rho} \frac{\partial p}{\partial z}-g+F_z
\end{eqnarray}

\subsection{Simplified PEs after scale analysis, now back on a PLANE}

$f$-plane: Coriolis term is a constant $f_0$ over the entire domain\\
$\beta$-plane: Coriolis term changes linearly in the meridional direction\\

\begin{eqnarray}
	\partd{u}{t} + \left( \mathbf{v} \cdot \nabla\right) u - f_0 v & = & -\frac{1}{\rho_o}\partd{p}{x} +F_x \\
	\partd{v}{t} + \left( \mathbf{v} \cdot \nabla\right) v + f_0 u & = & -\frac{1}{\rho_o}\partd{p}{y} +F_y \\
	0 & = & - \frac{1}{\rho} \partd{p}{z} - g + F_z
	%\partd{u}{x}+\partd{v}{y}+\partd{w}{z} & = & 0
\end{eqnarray}

and the continuity equation also becomes much simpler, as metric terms drop out. The simpler geometry of the plane may be retained while taking the latitudinal varations of $f$ into account, by defining a $\beta$-plane, for which $f$ is calculated by linearising the Coriolis terms around a constant $f_0=2\Omega \sin\phi_0$ -
the component of the Earth's rotation normal to the Earth's surface at
reference latitude $\phi_0$. 	


\section{Shallow water equations}

Making the planar, Boussinesq and hydrostatic approximations, and neglecting those unresolved forces, 
the PEs reduce to:
\begin{eqnarray}
	\lagd{u}{t}-f_0 v & = & -\frac{1}{\rho_o}\partd{p}{x} \\
	\lagd{v}{t}+f_0 u & = & -\frac{1}{\rho_o}\partd{p}{y} \\
	0 & = & - \frac{1}{\rho} \partd{p}{z} - g \\
	\partd{u}{x}+\partd{v}{y}+\partd{w}{z} & = & 0
\end{eqnarray}

plus thermodynamic eqn and eqn of state. Remember that $f_0=2\Omega \sin\phi_0$ is the Coriolis parameter normal to the Earth's surface at
reference latitude $\phi_0$.

Pressure and fluid depth are next divided into
a reference state and perturbation:
\begin{equation*}
	p=p_r(z)+p'(x,y,z,t)~~~~~;~~~~~h=H+\eta(x,y,z,t)
\end{equation*}

The shallow water equations are obtained by integrating the PEs
over a fluid layer of depth $h$. Integrating hydrostatic balance
downwards from the top to any level $z$:
\begin{eqnarray*}
	\frac{1}{\rho}\partd{p}{z}=-g \Rightarrow p_{top}-p(x,y,z,t) & = & -\rho_o g (h-z) \\
	\Rightarrow -\frac{1}{\rho_o} \partd{p}{x} & = & -g \partd{h}{x}
\end{eqnarray*}

assuming that there are no pressure perturbations on the top.
The pressure gradient terms become $-g\partd{\eta}{x}$ and
$-g\partd{\eta}{y}$, since the reference depth $H$ must be constant.

The mass conservation equation integrated vertically is:
\begin{eqnarray*}
	\partd{u}{x}+\partd{v}{y}+\partd{w}{z} = 0 \Rightarrow \left\{ \partd{u}{x}+\partd{v}{y} \right\} h+[w]^{top}_{bot} & = & 0 \\
	\left\{ \partd{u}{x}+\partd{v}{y} \right\} h+\lagd{h}{t} & = & 0 \\
	\partd{h}{t}+\partd{(hu)}{x}+\partd{(hv)}{y} & = & 0
\end{eqnarray*}
where $(u,v)$ is now the depth-average velocity and $w_{bot}=0$ was assumed. 


\subsection{SWEs linearised around a basic state} 

{\bf Shallow-water equations} on the
plane tangent to the Earth's surface, {\em linearised} about a resting basic state ($u=u_r(z)+u'(x,y,z,t); ~ v=v_r(z)+v'(x,y,z,t); ~ h=H+\eta(x,y,z,t)$):

\begin{eqnarray}
	\frac{\partial \eta}{\partial t} + H \left(
	\frac{\partial u}{\partial x}+\frac{\partial v}{\partial y}\right) & = & 0 \\
	\frac{\partial u}{\partial t} - f_0 v  + g\frac{\partial \eta}{\partial x} & = & 0 \\
	\frac{\partial v}{\partial t} + f_0 u  + g\frac{\partial \eta}{\partial y} & = & 0 
\end{eqnarray}
where $\eta$ is the surface elevation, $H$ is fluid depth, $(u,v)$ is
the depth-averaged horizontal velocity, $f$ is the Coriolis parameter
and $g$ is the gravitational acceleration. 

Now discretise the problem using regular grids to describe $u$, $v$ and $\eta$ 

e.g., $\eta_{ij}=$ free surface elevation at the $i$th longitude and $j$th
latitude point.

\section{Excursus: SWEs in 1D}

Before carrying out analysis of the different numerical approaches to integrating the 2D SWEs, can we predict what their behaviour will be, based on experience we have gained before? Let us start with advection in 1D, and build from here.

The simplest advection problem imposes a constant velocity to the transport of a species $q$:

\begin{equation}
	\frac{\partial q}{\partial t} = - c \frac{\partial q}{\partial x} 
\end{equation}

The most intuitive numerical approximation is:
\begin{equation}
	q_j^{n+1} = q_j^n - c \frac {\Delta t}{\Delta x} (q_{j+1}^n - q_j^n)
\end{equation}

But we can make this a bit more interesting, by relaxing the condition for constant velocity:
\begin{equation}
	\frac{\partial q}{\partial t} = - U \frac{\partial q}{\partial x}  \rightarrow  q_j^{n+1} = q_j^n - U_j^n \frac {\Delta t}{\Delta x} (q_{j+1}^n - q_j^n)
\end{equation}

which means that we require one more equation, to describe the time evolution of $U$, that is, $\frac{\partial U}{\partial t} = ?$

How can 1D SWEs help us understand grid stagger? If we want to move from the linear problem to the non-linear problem, we must add an equation for the time change of velocity. The 1D SWEs provide a good example:

\begin{eqnarray}
	\frac{\partial h}{\partial t} + H \frac{\partial u}{\partial x} = 0 \\
	\frac {\partial u}{\partial t} + g \frac{\partial h}{\partial x} = 0 \nonumber
\end{eqnarray}

Let $c^2 \equiv gH$ and substitute: we end up with two second order equations that have the very same form as the oscillation equation (see Chapter 2), that is: $\frac{\partial^2 u}{\partial t^2} = c^2 \frac{\partial^2 u}{\partial x^2}$ and $\frac{\partial^2 h}{\partial t^2} = c^2 \frac{\partial^2 h}{\partial x^2}$.

If we impose the usual wave solution (see the exercise) we end up with the dispersion relation: $\omega^2 = c^2k^2$, which has two solution, thus one wave moving eastward and one wave moving westward.

\begin{exercise}[Analytical versus numerical solutions]
	Start from the two shallow water equations on the previous page. Impose the wave solution proposed on the slide, for $u_j$ and $h_j$: 
	
	\begin{equation}
		(u_j,h_j) \sim e^{i(k x_j -\omega t)} = e^{i(k j \Delta x -\omega t)}
	\end{equation}
	
	\begin{enumerate}
		\item Substitute into the centred numerical expressions (in time and space) and show that you end up with an expression that contains a $\sin$ function.
		\item How does the numerical solution compare to the analytical solution in terms of a) amplitude and b) phase?
		\item Plot a wavenumber versus frequency plot for the analytical and numerical solution
		\item Expand your thinking to Numerical Weather Prediction models that might be using such schemes: discuss what operational problems we may encounter in the presence of a $\sin$ in such numerical predictions.
	\end{enumerate}
\end{exercise}

Let us now discretise the rhs, that is, the spatial gradient terms. Once again, we are tempted to go for the most intuitive way to do this, that is:

\begin{eqnarray}
	\frac{\partial h}{\partial t} + \frac {H}{2 \Delta x} (u_{j+1}^n - u_{j-1}^n)= 0 \\
	\frac {\partial u}{\partial t} + \frac {g}{2 \Delta x} (h_{j+1}^n - h_{j-1}^n) = 0 \nonumber
\end{eqnarray}

If we now discretise in time (using FT, just as an example, not as a recommended approach):

\begin{eqnarray}
	h_j^{n+1}  = h_j^n  -  \frac {H}{2 \Delta x} (u_{j+1}^n - u_{j-1}^n)  \\
	u_j^{n+1}  =  u_j^n -  \frac {g}{2 \Delta x} (h_{j+1}^n - h_{j-1}^n)  \nonumber
\end{eqnarray}

Analysis of the scheme above, and comparison with figure \ref{fig:1D-grid-stagger}, quickly reveals how we can anticipate a problematic computational mode, because two independent solutions can co-exist on the grid.

\begin{figure}[h!]
	\includegraphics[trim={0cm 0.cm 0cm 0.cm},clip,width=15.cm]{/Users/vidale/Projects/Teaching/MTMW14/Figures/1D-grid-stagger}
	\caption{Dave Randall's figure 10.1, a representation of centered variable stagger in solving the SWEs.}
	\label{fig:1D-grid-stagger}
\end{figure}

There is an additional shortcoming of the numerical approximation: if we impose a wave solution of the type $(u_j,h_j) \approx e^{i(kj\Delta x - \omega t)}$ (see a simplified example in the exercise in this section), we end up with a dispersion relation that is not identical to the analytical one. This is the numerical expression for the dispersion relation:

\begin{equation}
	\omega^2 = gH \left(  \frac{\sin k \Delta x}{\Delta x}^2   \right)
	\label{numerical-dispersion-relation}
	\end{equation}

\begin{exercise}[Shedding of waves, as dependent on wave number]
What does the numerical dispersion relation, eqn. \ref{numerical-dispersion-relation} mean for the propagation of a signal like a square wave (a delta function)? 
\end{exercise}

\subsection{Resolving waves on a grid: the problem of aliasing} 

We must always be aware of the dangers of under-representing our waves: we saw aliasing in the time domain, and the same principle applies here, for space. This is shown, using a figure borrowed from the time domain, with sampling points indicative of decreasing model resolution (decreasing sampling frequency), in figure \ref{fig:Waves-Aliasing}.

\begin{figure}[h!]
	\includegraphics[trim={0cm 0.cm 0cm 0.cm},clip,width=15.cm]{/Users/vidale/Projects/Teaching/MTMW14/"2019 module/Waves resolved on a Grid".png}
	\caption{Aliasing in 1D (example carried from the time domain, Dave Randall's Fig 1-17).}
	    \label{fig:Waves-Aliasing}
\end{figure}

\clearpage

\section{Arakawa's 2D grids}
\subsection { Horizontal gradients in SWEs: how to discretize in space?} 

SWEs seem nice and simple, but now we have spatial gradients: $\frac{\partial u}{\partial x}$, $\frac{\partial v}{\partial y}$, $\frac{\partial \eta}{\partial x}$, $\frac{\partial \eta}{\partial y}$.

How are we going to discretise the problem, using regular grids, to obtain a good simulation of $u$, $v$ and $\eta$? Something like this?

		\begin{eqnarray*}
			\frac{\partial u}{\partial x}\Big|_{ij} &\approx& \left(\frac{u_{i+1,j}+u_{i+1,j-1}}{2} - \frac{u_{i-1,j}+u_{i-1,j-1}}{2} \right)\frac{1}{2 \Delta x} \\
			\frac{\partial v}{\partial y}\Big|_{ij} &\approx& \left(\frac{v_{i,j+1}+v_{i-1,j+1}}{2} - \frac{v_{i,j-1}+v_{i-1,j-1}}{2} \right)\frac{1}{2 \Delta y} \\
			&& \\
			\frac{\partial \eta}{\partial x}\Big|_{ij} &\approx& \left(\frac{\eta_{i+1,j}+\eta_{i+1,j+1}}{2} - \frac{\eta_{i-1,j}+\eta_{i-1,j+1}}{2} \right)\frac{1}{2 \Delta x} \\
			f v\big|_{ij} &\approx& f_j \frac{v_{i,j}+v_{i,j+1}+v_{i-1,j}+v_{i-1,j+1}}{4} \\
			&& \\
			\frac{\partial \eta}{\partial y}\Big|_{ij} &\approx& \left(\frac{\eta_{i,j+1}+\eta_{i+1,j+1}}{2} - \frac{\eta_{i,j-1}+\eta_{i+1,j-1}}{2} \right)\frac{1}{2 \Delta y} \\
			f u\big|_{ij} &\approx& f_j \frac{u_{i,j}+u_{i+1,j}+u_{i,j-1}+u_{i+1,j-1}}{4}
		\end{eqnarray*}
~

{\bf But why? It all seems so arbitrary.} There are a few criteria: we want to avoid computational modes and we want waves to propagate/evolve in a realistic way in space and time.
\subsection{Horizontally staggered grids} 

Need to choose finite difference
methods to solve the {\bf shallow-water equations} on the
plane tangent to the Earth's surface, {\em linearised} about a resting basic state:
\begin{eqnarray}
	&&\frac{\partial \eta}{\partial t} + H \left(
	\frac{\partial u}{\partial x}+\frac{\partial v}{\partial y}\right) = 0,
	\\
	&&\frac{\partial u}{\partial t} - f_0 v = - g\frac{\partial \eta}{\partial x}, \\
	&&\frac{\partial v}{\partial t} + f_0 u = - g\frac{\partial \eta}{\partial y}, 
\end{eqnarray}
where $\eta$ is the surface elevation, $H$ is fluid depth, $(u,v)$ is
the depth-averaged horizontal velocity, $f$ is the Coriolis parameter
and $g$ is the gravitational acceleration. 

Now discretise the problem using regular grids to describe $u$, $v$ and $\eta$ 

e.g., $\eta_{ij}=$ free surface elevation at the $i$th longitude and $j$th
latitude point.


Arakawa\footnote{Arakawa
	A. (1966) Computational design for long-term numerical integration of
	the equations of fluid motion: Two-dimensional incompressible
	flow. Part I. \emph{Journal of Computational Physics}, \textbf{1},
	119-143.} proposed a number of staggered finite difference
grids that can be used to solve the shallow water equations.  


\subsection{Arakawa's A grid}
Simplest and intuitive, but there is the danger of computational modes.\\

\begin{tabular}{lc}
	\begin{minipage}[c]{0.6\textwidth}
		\begin{eqnarray*}
			\eta-grid \;\;\; \;\;\; \;\;\;
			\frac{\partial u}{\partial x}\Big|_{ij} &\approx& \frac{u_{i+1,j}-u_{i-1,j}}{2 \Delta x} \\
			\frac{\partial v}{\partial y}\Big|_{ij} &\approx& \frac{v_{i,j+1}-v_{i,j-1}}{2 \Delta y} \\
			&& \\
			u-grid \;\;\; \;\;\; \;\;\;
			\frac{\partial \eta}{\partial x}\Big|_{ij} &\approx& \frac{\eta_{i+1,j}-\eta_{i-1,j}}{2 \Delta x} \\
			f v\big|_{ij} &\approx& f v_{i,j} \\
			&& \\
			v-grid \;\;\; \;\;\; \;\;\;
			\frac{\partial \eta}{\partial y}\Big|_{ij} &\approx& \frac{\eta_{i,j+1}-\eta_{i,j-1}}{2 \Delta y} \\
			f u\big|_{ij} &\approx& f u_{i,j}
		\end{eqnarray*}
	\end{minipage}
	&
	\begin{minipage}[c]{0.4\textwidth}
		\setlength{\unitlength}{1 cm}
		\begin{picture}(4,4)
			\arakawa
			\put(0.93,0.93){$\bullet$} \put(1,1){\vector(1,0){0.5}} \put(1,1){\vector(0,1){0.5}}
			\put(2.93,0.93){$\bullet$} \put(3,1){\vector(1,0){0.5}} \put(3,1){\vector(0,1){0.5}}
			\put(0.93,2.93){$\bullet$} \put(1,3){\vector(1,0){0.5}} \put(1,3){\vector(0,1){0.5}}
			\put(2.93,2.93){$\bullet$} \put(3,3){\vector(1,0){0.5}} \put(3,3){\vector(0,1){0.5}}
			\put(1.1,0.7){$u_{ij}$,$v_{ij}$,}
			\put(1.1,0.4){$\eta_{ij}$}
			\put(3.1,3.6){$u_{i+1,j+1}$}
			\put(3.1,3.4){$v_{i+1,j+1}$}
			\put(3.1,3.2){$\eta_{i+1,j+1}$}
			\put(3.1,0.7){$u_{i+1,j}$}
			\put(3.1,0.5){$v_{i+1,j}$}
			\put(3.1,0.3){$\eta_{i+1,j}$}
		\end{picture}
	\end{minipage}
\end{tabular}

\subsection{Arakawa's B grid}
Stagger the variables, eliminate computational mode, but we pay the cost of lots of interpolation...\\

\begin{tabular}{lc}
	\begin{minipage}[c]{0.6\textwidth}
		\begin{eqnarray*}
			\frac{\partial u}{\partial x}\Big|_{ij} &\approx& \left(\frac{u_{i+1,j}+u_{i+1,j+1}}{2} - \frac{u_{i,j}+u_{i,j+1}}{2} \right)\frac{1}{\Delta x} \\
			\frac{\partial v}{\partial y}\Big|_{ij} &\approx& \left(\frac{v_{i,j+1}+v_{i+1,j+1}}{2} - \frac{v_{i,j}+v_{i+1,j}}{2} \right)\frac{1}{\Delta y} \\
			&& \\
			\frac{\partial \eta}{\partial x}\Big|_{ij} &\approx& \left(\frac{\eta_{i,j-1}+\eta_{i,j}}{2} - \frac{\eta_{i-1,j-1}+\eta_{i-1,j}}{2} \right)\frac{1}{\Delta x} \\
			f v\big|_{ij} &\approx& f v_{i,j} \\
			&& \\
			\frac{\partial \eta}{\partial y}\Big|_{ij} &\approx& \left(\frac{\eta_{i-1,j}+\eta_{i,j}}{2} - \frac{\eta_{i-1,j-1}+\eta_{i,j-1}}{2} \right)\frac{1}{\Delta y} \\
			f u\big|_{ij} &\approx& f u_{i,j}
		\end{eqnarray*}
	\end{minipage}
	&
	\begin{minipage}[c]{0.4\textwidth}
		\setlength{\unitlength}{1 cm}
		\begin{picture}(4,4)
			\arakawa
			\put(1.93,1.93){$\bullet$}
			\put(2.1,1.7){$\eta_{ij}$}
			\put(1,1){\vector(1,0){0.5}} \put(1,1){\vector(0,1){0.5}}
			\put(3,1){\vector(1,0){0.5}} \put(3,1){\vector(0,1){0.5}}
			\put(1,3){\vector(1,0){0.5}} \put(1,3){\vector(0,1){0.5}}
			\put(3,3){\vector(1,0){0.5}} \put(3,3){\vector(0,1){0.5}}
			\put(1.1,0.7){$u_{ij}$,$v_{ij}$}
			\put(3.1,3.4){$u_{i+1,j+1}$}
			\put(3.1,3.2){$v_{i+1,j+1}$}
			\put(3.1,0.7){$u_{i+1,j}$}
			\put(3.1,0.5){$v_{i+1,j}$}
		\end{picture}
	\end{minipage}
\end{tabular}

\clearpage

\subsection{Arakawa's C grid}
Stagger the variables, eliminate computational mode, interpolation only required for Coriolis terms!\\

\begin{tabular}{lc}
	\begin{minipage}[c]{0.6\textwidth}
		\begin{eqnarray*}
			\eta-grid \;\;\; \;\;\; 
			\frac{\partial u}{\partial x}\Big|_{ij} &\approx& \frac{u_{i+1,j}-u_{i,j}}{\Delta x} \\
			\frac{\partial v}{\partial y}\Big|_{ij} &\approx& \frac{v_{i,j+1}-v_{i,j}}{\Delta y} \\
			&& \\
			u-grid \;\;\; \;\;\; 
			\frac{\partial \eta}{\partial x}\Big|_{ij} &\approx& \frac{\eta_{i,j}-\eta_{i-1,j}}{\Delta x} \\
			f v\big|_{ij} &\approx& f_j \frac{v_{i,j}+v_{i,j+1}+v_{i-1,j}+v_{i-1,j+1}}{4} \\
			&& \\
			v-grid \;\;\; \;\;\; 
			\frac{\partial \eta}{\partial y}\Big|_{ij} &\approx& \frac{\eta_{i,j}-\eta_{i,j-1}}{\Delta y} \\
			f u\big|_{ij} &\approx& f_j \frac{u_{i,j}+u_{i+1,j}+u_{i,j-1}+u_{i+1,j-1}}{4}
		\end{eqnarray*}
	\end{minipage}
	&
	\begin{minipage}[c]{0.4\textwidth}
		\setlength{\unitlength}{1 cm}
		\begin{picture}(4,4)
			\arakawa
			\put(1.93,1.93){$\bullet$}
			\put(2.1,1.7){$\eta_{ij}$}
			\put(2,0.75){\vector(0,1){0.5}}
			\put(2,2.75){\vector(0,1){0.5}}
			\put(0.75,2){\vector(1,0){0.5}}
			\put(2.75,2){\vector(1,0){0.5}}
			\put(1.1,1.7){$u_{ij}$}
			\put(2.1,0.7){$v_{ij}$}
			\put(3.1,1.7){$u_{i+1,~j}$}
			\put(2.1,2.7){$v_{i,~j+1}$}
		\end{picture}
	\end{minipage}
\end{tabular}


\subsection{Arakawa's D grid}
Idea similar to B grid, but we push out $u,v$ instead of $\eta$: stagger the variables, eliminate computational mode, but lots of interpolation...\\

\begin{tabular}{lc}
	\begin{minipage}[c]{0.7\textwidth}
		\begin{eqnarray*}
			\frac{\partial u}{\partial x}\Big|_{ij} &\approx& \left(\frac{u_{i+1,j}+u_{i+1,j-1}}{2} - \frac{u_{i-1,j}+u_{i-1,j-1}}{2} \right)\frac{1}{2 \Delta x} \\
			\frac{\partial v}{\partial y}\Big|_{ij} &\approx& \left(\frac{v_{i,j+1}+v_{i-1,j+1}}{2} - \frac{v_{i,j-1}+v_{i-1,j-1}}{2} \right)\frac{1}{2 \Delta y} \\
			&& \\
			\frac{\partial \eta}{\partial x}\Big|_{ij} &\approx& \left(\frac{\eta_{i+1,j}+\eta_{i+1,j+1}}{2} - \frac{\eta_{i-1,j}+\eta_{i-1,j+1}}{2} \right)\frac{1}{2 \Delta x} \\
			f v\big|_{ij} &\approx& f_j \frac{v_{i,j}+v_{i,j+1}+v_{i-1,j}+v_{i-1,j+1}}{4} \\
			&& \\
			\frac{\partial \eta}{\partial y}\Big|_{ij} &\approx& \left(\frac{\eta_{i,j+1}+\eta_{i+1,j+1}}{2} - \frac{\eta_{i,j-1}+\eta_{i+1,j-1}}{2} \right)\frac{1}{2 \Delta y} \\
			f u\big|_{ij} &\approx& f_j \frac{u_{i,j}+u_{i+1,j}+u_{i,j-1}+u_{i+1,j-1}}{4}
		\end{eqnarray*}
	\end{minipage}
	&
	\begin{minipage}[c]{0.3\textwidth}
		\setlength{\unitlength}{1 cm}
		\begin{picture}(4,4)
			\arakawa
			\put(0.93,0.93){$\bullet$}
			\put(2.93,0.93){$\bullet$}
			\put(0.93,2.93){$\bullet$}
			\put(2.93,2.93){$\bullet$}
			\put(2,0.75){\vector(0,1){0.5}}
			\put(2,2.75){\vector(0,1){0.5}}
			\put(0.75,2){\vector(1,0){0.5}}
			\put(2.75,2){\vector(1,0){0.5}}
			\put(1.1,1.7){$u_{ij}$}
			\put(2.1,0.7){$v_{ij}$}
			\put(1.1,0.7){$\eta_{ij}$}
		\end{picture}
	\end{minipage}
\end{tabular}

\clearpage

\subsection{Arakawa's Sm\"org{\aa}sbord}

\begin{figure}[h]
	\centering
	\includegraphics[width=0.8\textwidth]{/Users/vidale/Documents/Presentations/"SWE 2D grid stagger types".pdf}
	\caption{Many of Arakawa's grid staggers}
	\label{fig:Arakawa_all}
\end{figure}

%%%%%%%%%%%%%%%%%%%%%%%%%%%%%%%%%%%%%%%%%%%%%%%%%%
\newpage

\section{Boundary Conditions (BCs)}

\subsection{Dirichlet, Neuman}
The \textbf{Dirichlet (or first-type) boundary condition }is a type of boundary condition, named after Peter Gustav Lejeune Dirichlet (1805–1859). When imposed on an ordinary or a partial differential equation, it specifies the values that a solution needs to take on along the boundary of the domain. For example, for an ODE:

\begin{equation}
	y''+y=0
\end{equation}
we could specify $y(a)=\alpha; y(b)=\beta$ on the interval $[a,b]$, where $\alpha$, $\beta$ are given numbers.

The \textbf{Neumann (or second-type) boundary condition} is a type of boundary condition, named after Carl Neumann. When imposed on an ordinary or a partial differential equation, the condition specifies the values in which the derivative of a solution is applied within the boundary of the domain. For example, for the same ODE above:

we could specify $y'(a)=\alpha; y'(b)=\beta$ on the interval $[a,b]$, where $\alpha$, $\beta$ are given numbers.

\subsection{Robin, mixed}
The \textbf{Robin boundary condition, or third type boundary condition}, is a type of boundary condition, named after Victor Gustave Robin (1855–1897). When imposed on an ordinary or a partial differential equation, it is a specification of a linear combination of the values of a function and the values of its derivative on the boundary of the domain. For instance:\\

\begin{equation}
	au+b\frac{\partial u}{\partial n}y=g \textrm{ on } \partial \Omega
\end{equation}


\begin{figure}[h!]
	\begin{center}
		\includegraphics[width=0.2\textwidth]{Figures/440px-Mixed_boundary_conditions.png}
	\end{center}
	\caption{Mixed boundary conditions}
\end{figure}


In 1-D, on the interval $\Omega=[0,1]$ we could for instance have:

\begin{eqnarray}
	au(0)-Bu'(0)=g(0) \\
	au(1)+Bu'(1)=g(1)
\end{eqnarray}

This contrasts with \textbf{mixed boundary conditions}, which are boundary conditions of different types specified on different subsets of the boundary.

\clearpage

\section{Time integration for 2D SWE}
\subsection{Staggered time integration schemes for the SWE}

The advection-diffusion equation in one dimension is a useful prototype with which to explore issues of stability, convergence and approximation error. However, the equations ocean modellers use are frequently multi-dimensional. This leads to special problems and results in some cases. With regards to temporal stability, however, schemes which are unstable in one dimension are also unstable in multi-dimensional problems. Schemes which are conditionally stable in one dimension are also conditionally stable in two dimensions or higher, but with more restrictive conditions on $\Delta t$.

\vspace{0.2cm}For equations that support more than one type of process, {\bf the stability criterion will depend on the fastest propagating processes}. However, the fastest propagating phenomena might be of little physical importance and therefore the stability condition might be too constraining. For instance, the linear shallow water equations allow the existence of inertia-gravity and Rossby waves. The propagation speed of the former is about $\sqrt{gH} \approx 100 $ ms$^{-1}$, which is quite large. For Rossby waves, the propagation is of the order of $\beta R_D^2$, which is much smaller. The time integration scheme will therefore greatly depend on the physical processes of interest. 

\subsection{A preview of Lecture 5: a Semi-Implicit (SI) scheme for SWEs}

The shallow water equations linearised about a state of rest are below
discretised using a leapfrog scheme for Coriolis terms but a
trapezoidal scheme (mixed implicit-explicit) for the gravity wave
terms:
\begin{eqnarray*}
	\frac{u^{n+1}-u^{n-1}}{2\Delta t}-fv^n+\frac{g}{2}
	\left( \partd{h}{x}^{n+1} +\partd{h}{x}^{n-1} \right) & = & 0 \\
	\frac{v^{n+1}-v^{n-1}}{2\Delta t}+fu^n+\frac{g}{2}
	\left( \partd{h}{y}^{n+1} +\partd{h}{y}^{n-1} \right) & = & 0 \\
	\frac{h^{n+1}-h^{n-1}}{2\Delta t}+\frac{H}{2}
	\left( \partd{u}{x}^{n+1} +\partd{u}{x}^{n-1}
	+\partd{v}{y}^{n+1} +\partd{v}{y}^{n-1} 
	\right) & = & 0. 
\end{eqnarray*}

Re-arranging with future values on the left:
\begin{eqnarray*}
	u^{n+1}+\Delta t g \partd{h}{x}^{n+1} & = & A \\
	v^{n+1}+\Delta t g \partd{h}{y}^{n+1} & = & B \\
	h^{n+1}+\Delta t H \left( \partd{u}{x}^{n+1}+\partd{v}{y}^{n+1} \right) & = & C
\end{eqnarray*}

Can we mix and match at will? Let us start again with the basic ideas of explicit and implicit.


\subsection{Implicit schemes}

In the shallow water equations, fast gravity waves are due to the divergence and gradient terms, and slow Rossby waves are due to the Coriolis term. By treating them differently, one could try to circumvent the stability condition due to gravity waves. The following time integration schemes allow one to change the degree of implicity of the divergence, Coriolis and gradient terms:;
\begin{eqnarray*}
	\eta^{n+1} + \alpha h \Delta t \bnabla \cdot \bu^{n+1} &=& \eta^n - (1-\alpha) h \Delta t \bnabla \cdot \bu^n,
	\\
	\bu^{n+1} + \beta f \Delta t \bk \times \bu^{n+1} + \gamma g \Delta t \bnabla \eta^{n+1} &=& \bu^n - (1-\beta) f \Delta t \bk \times \bu^n - (1-\gamma) g \Delta t \bnabla \eta^n.
\end{eqnarray*}
The implicity coefficients $\alpha$, $\beta$ and $\gamma \in [0,1]$. One way to have ``some information'' about the accuracy and stability of a time integration scheme is to compute the evolution of the energy:
\[
E^n = \int_{\Omega} \frac{1}{2} \rho \left( g (\eta^n)^2 + h \|\bu^n\|^2 \right) \ud \Omega.
\]
Note that this quantity only gives information about the changes in the amplitude of the solution. It does not indicate how the numerical method is affecting the phase of the solution.


\subsubsection{``Mostly implicit'' schemes are unconditionally stable}

\begin{center}
	\begin{tabular}{cc}
		$\alpha = \beta = \gamma = 1/2$  & $\alpha = \beta = \gamma = 1$ \\
		\includegraphics[width=0.4\textwidth]{Figures/energy_si.eps}
		&
		\includegraphics[width=0.4\textwidth]{Figures/energy_impl.eps}
	\end{tabular}
\end{center}
However, they can be quite dissipative if they are ``a bit too implicit''. The semi-implicit scheme ($\alpha = \beta = \gamma = 1/2$) is exactly conserving energy while a fully-implicit scheme ($\alpha = \beta = \gamma = 1$) is very dissipative and thus not always accurate.


\subsubsection{Both the gravity and Coriolis terms must be at least semi-implicit}

\begin{center}
	\begin{tabular}{cc}
		$\alpha = \gamma = 1/2$, $\beta=0$  & $\beta = 1/2$, $\alpha = \gamma = 0$ \\
		\includegraphics[width=0.4\textwidth]{Figures/energy_sifexpl.eps}
		&
		\includegraphics[width=0.4\textwidth]{Figures/energy_expl.eps}
	\end{tabular}
\end{center}


When there is no dissipation, the equations are purely hyperbolic ($\mathcal{R}e(\kappa) = 0$) and the time integration scheme can only be stable if the implicity coefficients are all larger or equal to 1/2. In that case, the solution of the system requires to invert a non-diagonal matrix, which can be computationally expensive. The effect of using a ``mostly implicit'' time integration scheme with a large time step is to slow down fast propagating gravity waves. These schemes are thus useful for long simulation for which gravity waves are physically insignificant.


\subsection{Explicit schemes}
\subsubsection{Example of a stable explicit scheme: Adams-Bashforth 3}

\begin{eqnarray*}
	&&\hspace{-0.6cm} \eta^{n+1} = \eta^n - h \Delta t \bnabla \cdot \left(\frac{23}{12} \bu^n - \frac{16}{12} \bu^{n-1} + \frac{5}{12} \bu^{n-2}\right),
	\\
	&&\hspace{-0.6cm} \bu^{n+1} = \bu^n - f \Delta t \bk \times \left(\frac{23}{12} \bu^n - \frac{16}{12} \bu^{n-1} + \frac{5}{12} \bu^{n-2}\right) - g \Delta t \bnabla \left( \frac{23}{12} \eta^n - \frac{16}{12} \eta^{n-1} + \frac{5}{12} \eta^{n-2} \right)
\end{eqnarray*}

%\begin{tabular}{lc}
%	\begin{minipage}{0.6\textwidth}
	
	\begin{figure}
		\includegraphics[trim={0.25cm 0.cm 0.25cm 0.25cm},clip,width=0.8\textwidth]{Figures/energy_AB3.eps}
	\end{figure}
	%	\end{minipage}
\hfill
%	\begin{minipage}{0.35\textwidth}
	
This scheme \footnote{More details: Durran D.R. (1991) ``The Third-order Adams-Bashforth method: An attractive alternative to Leap-frog time differencing''. \emph{Monthly Weather Review}, 119, 702-720.} is conditionally stable with a stability condition prescribed by gravity waves. A leap-frog scheme would have about the same properties but it would become unstable if some dissipation was added to the scheme. \\
	
	%	\end{minipage}
%\end{tabular}


\subsubsection{Another example: Forward-Backward in Time scheme}

The Forward-Backward in Time (FBT) scheme has been introduced by Sielecki (1968) and later analysed and improved by Beckers and Deleersnijder (1993)\footnote{Sielecki A. (1968) ``An energy conserving difference scheme for the storm surge equations'', \emph{Monthly Weather Review}, 96, 150-156. Beckers J. and Deleersnijder E. (1993) ``Stability of a FBTCS scheme applied to the propagation of shallow-water inertia-gravity waves on various grids'', \emph{Journal of Computational Physics}, 108, 95-104.}. It tries to mimic a semi-implicit scheme by alternatively changing the order in which the two momentum equations are solved for. The future height anomaly term is used in the two momentum equations as soon as it is available; the Coriolis term is discretized by using the most recently computed velocity component, which requires alternation:

\[
\left\{ \begin{array}{l}
	\eta^{n+1} = \eta^n - h \Delta t \bnabla \cdot \bu^n,
	\\
	u^{n+1} = u^n + f \Delta t v^n - g \Delta t \ds \frac{\partial \eta}{\partial x}^{n+1},
	\\
	v^{n+1} = v^n - f \Delta t u^{n+1} - g \Delta t \ds \frac{\partial \eta}{\partial y}^{n+1},
\end{array} \right.
\]
and
\[
\left\{ \begin{array}{l}
	\eta^{n+2} = \eta^{n+1} - h \Delta t \bnabla \cdot \bu^{n+1},
	\\
	v^{n+2} = v^{n+1} - f \Delta t u^{n+1} - g \Delta t \ds \frac{\partial \eta}{\partial y}^{n+2},
	\\
	u^{n+2} = u^{n+1} + f \Delta t v^{n+2} - g \Delta t \ds \frac{\partial \eta}{\partial x}^{n+2}.
\end{array} \right.
\]

It can be seen that for each set of equations, the Coriolis term is first discretized explicitly and then implicitly. The order of the two momentum equations is switched in the second set of equations. As a result, the time discretization of the Coriolis term appears to be semi-implicit ``on average''. The divergence is always explicit and the gradient is always implicit. It can be shown that the scheme is conditionally stable with about the same stability condition as AB3.

\begin{center}
	\includegraphics[width=0.8\textwidth]{Figures/energy_FBT.eps}
\end{center}

\section{Real world applications: variable staggering in W \& C models}

\begin{figure}[h!]	
	\includegraphics[width=0.8\textwidth]{Figures/TC-precipitation}
	\caption{Hurricane and Typhoon simulation in climate GCMs: precipitation composite}
\end{figure}

\subsection{Unified Model: New Dynamics vs EndGame grids} 

The choice of grid stagger makes a difference even in very complex GCMs. Here is an example of simply switching the variables defined at the poles.

\begin{tabular}{cc}
	\centering
	\includegraphics[width=0.75\textwidth]{Figures/New_Dynamics_vs_EndGame_grids.pdf}
\end{tabular}

This is, in fact, not at all a simple matter: implementation took years, but the consequences were profound in terms of scalability and numerical stability.

\subsection{Unified Model DynCore evolution: stability and scalability} 

The evolution from New Dynamics (2002) to EndGame (2014) meant greater scalability, so that we can use up to 12'000 cores efficiently, as well as numerical stability, which makes long climate simulations at high-resolution (up to N2560, 5km) feasible.

\begin{figure}[h!]
	\centering
	\begin{subfigure}[b]{0.9\textwidth}
		\includegraphics[width=1.\textwidth]{Figures/EndGame_stability.pdf}
		\caption{}
		\label{fig:UM-Instabilities} 
	\end{subfigure}
	\begin{subfigure}[b]{0.9\textwidth}
		\includegraphics[width=1.\textwidth]{Figures/EndGame_scalability_2.png}
		\caption{}
		\label{fig:UM-scalablity}
	\end{subfigure}
	\caption{(a) Numerical solutions for the Unified model around the South Pole, using the New Dynamics and EndGame dynamical cores; (b) the parallel scalability of the two DynCores.}
\end{figure}

%%%%%%%%%%%%%%%%%%%%%%%%%%%%%%%%%%%%%%%%%%%%%%%%%%

%%%%%%%%%%%%%%%%%%%%%%%%%%%%%%%%%%%%%%%%%%%%%%%%%%
% WAVES IN 2D WAVES
\newpage	
\input{Waves-Chapter}
%%%%%%%%%%%%%%%%%%%%%%%%%%%%%%%%%%%%%%%%%%%%%%%%%%%%	

%%%%%%%%%%%%%%%%%%%%%%%%%%%%%%%%%%%%%%%%%%%%%%%%%%
% Semi-Implicit Semi-Lagrangian 
\newpage	
\chapterimage{2613-1477-max.jpg} % Chapter heading image	
\chapter{Semi-Implicit Semi-Lagrangian}
%%%%%%%%%%%%%%%%%%%%%%%%%%%%%%%%%%%%%%%%%%%%%%%%%%%%

\section{Experimental design: choosing the type of model}
\subsection{Grid point methods}
	
	Let us say that we are working with a periodic function $f$ in the interval $0..2\pi$, as shown in the figure below.
	
	\begin{center}
		\begin{tabular}{c}
			\includegraphics[width=0.85\textwidth]{Figures/Durran-figOne1.png}
		\end{tabular}
	\end{center}
	
	In grid-point methods, each function is approximated by a value at a set of discrete grid points. Using a \underline{finite difference method}, we can approximate the derivative this way:
	
	\begin{equation}
		\frac{df}{dx}(x_0)\approx \frac{f(x_0+\Delta x)-f(x_0- \Delta x)}{2\Delta x}
	\end{equation}

\subsection{Finite volumes}

	Using a \underline{finite volume method}, we do not really need to compute derivatives, rather the fluxes between cells, but we will need to approximate the structure of the solution inside the grid cell, for instance with a piecewise constant function, or a piecewise linear function, like this:
	
	\begin{equation}
		f(x)\approx f_j +\sigma_j(x-j \Delta x) \text{ for all } x \in (j-\frac{1}{2}\Delta x,j+\frac{1}{2}\Delta x)
	\end{equation}
	
	where $f_j$ is the average of the approximate solution over the grid cell centered at $j \Delta x$ and $\sigma_j =\frac{(f_{J+1}-f_j)}{\Delta x}$ 
	
	\begin{center}
		\begin{tabular}{c}
			\includegraphics[width=0.85\textwidth]{Figures/Durran-figOne2.png}
		\end{tabular}
	\end{center}
	

\subsection{Series expansion methods: spectral methods}
	
	In series expansion methods, the unknown function is approximated by a linear combination of a finite set of continuous \underline{expansion functions}, and the data set describing the approximated function is the finite set of \underline{expansion coefficients}.\\
	
	\medskip
	When the expansion functions form an \textbf{orthogonal set}, the series-expansion approach is a \underline{spectral method}. For instance, we have seen already the use of Fourier series:
	
	\begin{equation}
		a_1+a_2\cos{x}+a_3\sin{x}+a_4\cos{2x}+a_5\sin{2x}
	\end{equation}
	
	\medskip
	The goal is always to find coefficients that minimise the error. The five coefficients above need not be chosen such that the value of the Fourier series exactly matches the value of $f(x)$ at any specific point in the interval $0 \leq x \leq \pi$. However, we could do it at specific grid points by using the grid method we discussed in the spectral methods lecture.\\
	
	An alternative method is to try to minimise the integral of the square of the error (residual) in the $x$ domain.


\subsection{Series expansion methods: Finite Elements}
	
	
	If the expansion functions are nonzero in only a small part of the domain, the series expansion technique is a \underline{finite-element method}.
	
	\begin{equation}
		f(x)\approx b_0s_0(x)+b_1s_1(x)+...+b_5s_5(x)
	\end{equation}	

	similar to the spectral method, but now the functions $s_n$ differ from trigonometric functions, because each function $s$ is zero over most of the domain. The simplest FE expansion functions $s$ are piecewise-linear functions defined over a grid: each function is equal to 1 at one grid point (or node) and zero everywhere else. The values of the expansion function between nodes are determined by linear interpolation.
	
	\begin{center}
		\begin{tabular}{c}
			\includegraphics[width=0.85\textwidth]{Figures/Durran-figOne3.png}
		\end{tabular}
	\end{center}
	

\subsection{Numerical analysis tied to phenomena: examples from Project 2}
	
%\begin{minipage}[c]{0.7\textwidth}	
\begin{center}
	\begin{tabular}{cc}
			\includegraphics[width=0.9\textwidth]{Figures/Logical decisions for Project 2.pdf}
	\end{tabular}
\end{center}
%\end{minipage}

\begin{tabular}{l}
%		\begin{minipage}[r]{0.3\textwidth}	
			This is all to say that there is a logical sequence to be followed when making key decisions on how to set up our experiment. We start with our scientific objectives, but we must consider several worst case scenarios, in order to make sure that our model will not become unstable, else return poor quality solutions.
%		\end{minipage}
\end{tabular}		



\section{Advanced topic: semi-implicit schemes}
\subsection{Semi-implicit schemes}

So far we have focussed on the spatial discretisation. Now let us
consider the discretisation of time. The two are linked, as you shall learn in Project 2.

The simplest choice for time derivatives is a simple forward or
centred (leapfrog) scheme. We can also choose the time-level of the RHS terms. For example,
\BEQ
\frac{u^{n+1}-u^n}{\Delta t}=fv^n ~~~~;~~~~~\frac{u^{n+1}-u^n}{\Delta t}=fv^{n+1}
\EEQ

are forward explicit and implicit schemes respectively.


Rule of thumb: implicit schemes are more stable than explicit
(e.g., implicit schemes for exponential decay are stable for any
time-step). However, fully implicit schemes for coupled equations are
difficult to solve and involve expensive iterations. 


Numerical weather prediction models use semi-implicit methods where
only the ``gravity wave terms'' are treated implicity and the rest are
explicit. The mass conservation equation can be {\em inverted} to
solve for future $\eta^{n+1}$. No iteration is required and schemes
are devised so that the matrix inversion is only done once. 


{\bf Benefit:} Lifts CFL restriction on time-step associated with
the fast GWs.

{\bf Cost:} Distorts and slows gravity waves.


\subsection{The simplest possible approach to SI}

In atmospheric models, the fastest gravity waves, i.e., the external-gravity or “Lamb” waves, have speeds on the order of 300 $m s^{-1}$ , which is also the speed of sound. The typical time step for a model with a 10km mesh will thus have to be \underline{\hspace{1cm}}? This is unfortunate, because the external gravity modes are believed to play only a minor role in weather and climate dynamics. 

\begin{eqnarray}
	 \frac{h_{j}^{n+1} - h_{j}^{n}}{\Delta t}  + H \left(  \frac{u_{j+1}^{n+1} - u_{j}^{n+1}}{\Delta x}  \right)  & = & 0 \\
	 	\frac{u_{j+1}^{n+1} - u_{j+1}^{n}}{\Delta t}  + g \left( \frac{h_{j+1}^{n+1} - h_{j}^{n+1}}{\Delta x}   \right)  & = & 0 
\end{eqnarray}

%\begin{minipage}{0.4\textwidth}
%\begin{figure}
%	\includegraphics[trim={2cm 4.5cm 0cm 3cm},clip,width=12.cm]{/Users/vidale/Projects/Teaching/MTMW14/"2016 module"/Figures/SWE_GWs_Implicit}
%	\label{fig:Spherical}
	%\setbeamerfont{caption name}{size=\tiny}
%	\caption{\label{fig:blue_rectangle} 1D SWEs in implicit form}
%\end{figure}
%\end{minipage} 

Gravity waves are, therefore, commonly treated with implicit schemes, in order to mitigate this problem. However, this means solving a matrix problem (see following slides).	


%\begin{minipage}{0.4\textwidth}
%\begin{figure}
%	\includegraphics[trim={1cm 2cm 0cm 2cm},clip,width=12cm]{/Users/vidale/Projects/Teaching/MTMW14/"2016 module"/Figures/SWE_GWs_FWBW_h_equation}\\
%	\includegraphics[trim={0cm 2cm 0cm 2cm},clip,width=12cm]{/Users/vidale/Projects/Teaching/MTMW14/"2016 module"/Figures/SWE_GWs_FWBW_u_equation}
%	\label{fig:Spherical}
	%\setbeamerfont{caption name}{size=\tiny}
%	\caption{\label{fig:blue_rectangle} 1D SWEs in FW-BW form}
%\end{figure}
%\end{minipage} \hfill

%\begin{minipage}{0.55\textwidth}
Another approach is to go for the so called \emph{forward-backward} scheme, which eliminated the need for solving a matrix problem.
%\end{minipage}

\begin{eqnarray}
	\frac{h_{j}^{n+1} - h_{j}^{n}}{\Delta t}  + H \left(  \frac{u_{j+1}^{n} - u_{j}^{n}}{\Delta x}  \right)  & = & 0 \\
	\frac{u_{j+1}^{n+1} - u_{j+1}^{n}}{\Delta t}  + g \left( \frac{h_{j+1}^{n+1} - h_{j}^{n+1}}{\Delta x}   \right)  & = & 0 
\end{eqnarray}

\subsection{A semi-implicit scheme for SWEs}

The shallow water equations linearised about a state of rest are below
discretised using a leapfrog scheme for Coriolis terms but a
trapezoidal scheme (mixed implicit-explicit) for the gravity wave
terms:
\begin{eqnarray*}
	\frac{u^{n+1}-u^{n-1}}{2\Delta t}-fv^n+\frac{g}{2}
	\left( \partd{h}{x}^{n+1} +\partd{h}{x}^{n-1} \right) & = & 0 \\
	\frac{v^{n+1}-v^{n-1}}{2\Delta t}+fu^n+\frac{g}{2}
	\left( \partd{h}{y}^{n+1} +\partd{h}{y}^{n-1} \right) & = & 0 \\
	\frac{h^{n+1}-h^{n-1}}{2\Delta t}+\frac{H}{2}
	\left( \partd{u}{x}^{n+1} +\partd{u}{x}^{n-1}
	+\partd{v}{y}^{n+1} +\partd{v}{y}^{n-1} 
	\right) & = & 0. 
\end{eqnarray*}

Re-arranging with future values on the left:
\begin{eqnarray*}
	u^{n+1}+\Delta t g \partd{h}{x}^{n+1} & = & A \\
	v^{n+1}+\Delta t g \partd{h}{y}^{n+1} & = & B \\
	h^{n+1}+\Delta t H \left( \partd{u}{x}^{n+1}+\partd{v}{y}^{n+1} \right) & = & C
\end{eqnarray*}

\subsection{A semi-implicit scheme for SWEs}

Substituting $u^{n+1}$ and $v^{n+1}$ into the mass conservation
equation gives:
\begin{eqnarray*}
	\left\{ 1-\Delta t^2 gH \left( \partd{^2}{x^2}+\partd{^2}{y^2} \right) 
	\right\} h^{n+1}
	& = & C-\Delta t H \left\{ \partd{A}{x}+\partd{B}{y} \right\} \\
	{\cal L}\,h^{n+1} & = & F(u^{n-1},u^n,v^{n-1},v^n,h^{n-1},h^n) \\
	h^{n+1} & = & {\cal L}^{-1} F
\end{eqnarray*}

In words, the future depth can be found if the operator $\cal{L}$ can be
{\em inverted}. Once $h^{n+1}$ has been found, we can easily solve for
$u^{n+1}$ and $v^{n+1}$. 


\subsection{A semi-implicit scheme for SWEs}
The form of the operator $\cal{L}$ depends on the
representation of spatial derivatives by the numerical model. For
example, if a second order finite difference scheme is used:
\begin{equation}
	\partd{^2 h}{x^2}\approx \frac{h_{i+1}-2h_i+h_{i-1}}{\Delta x^2}
\end{equation}
then the $\cal{L}$ operator (in 1-D) becomes a tri-diagonal matrix:
\begin{equation}
	\left(
	\begin{array}{ccccccc}
		1-d & d & 0 & 0 & 0 & 0 & ... \\
		d & 1-2d & d & 0 & 0 & 0 & ... \\
		0 & d & 1-2d & d & 0 & 0 & ... \\
		& & \vdots & & & & ... 
	\end{array}
	\right) \left(
	\begin{array}{c}
		h^{n+1}_1 \\
		h^{n+1}_2 \\
		h^{n+1}_3 \\
		\vdots
	\end{array}
	\right) = \left(
	\begin{array}{c}
		F_1 \\
		F_2 \\
		F_3 \\
		\vdots
	\end{array}
	\right)
\end{equation}

where $d=gH\Delta t^2/\Delta x^2$. In this case the matrix is
time-invariant and only needs to be inverted once. The semi-implicit
scheme barely costs more that an explicit scheme per time-step but
enables a much longer time-step because it lifts the CFL restriction
associated with gravity wave speed $\sqrt{gH}$.

\clearpage

\section{Frames of reference}
\subsection{A reminder: Eulerian and Lagrangian frames of reference}
	
	We want to describe  the evolution of a chemical tracer $\Psi(x,t)$ in a one dimensional flow field, with sources/sinks $S(x,t)$.	We can do so within two frameworks:	

%\begin{tabular}{lc}
%\begin{minipage}[l]{0.5\textwidth}	

\begin{definition}[Eulerian]
\begin{equation}
\frac {\partial \Psi}{\partial t} + u \frac {\partial \Psi}{\partial x}= S
\label{eqn:eulerian}
\end{equation}
\end{definition}

\begin{definition}[Lagrangian]
\begin{equation}
\frac {d \Psi}{d t} = S
\label{eqn:lagrangian}
\end{equation}
\end{definition}

	\begin{center}
		\begin{tabular}{cc}
			\rotatebox{0}{\includegraphics[width=0.3\textwidth]{Figures/UP-Eulerian.png}} \\
			\rotatebox{0}{\includegraphics[width=0.15\textwidth]{Figures/UP-Lagrangian.png}}
		\end{tabular}
	\end{center}


The two are tied by:

\begin{definition}[The total derivative]
	\begin{equation}
\frac {d}{d t} = \frac {\partial }{\partial t} + \frac {d x}{d t} \frac {\partial }{\partial x}
\label{eqn:totalderivative}
\end{equation}
\end{definition}

and

\begin{definition}[velocity]
		\begin{equation}
\frac {d x}{d t} = u
\label{eqn:velocity}
\end{equation}
\end{definition}

We could solve \ref{eqn:lagrangian} as an initial value problem, by choosing a number of regularly spaced fluid particles at $t=0$, assigning a $\Psi$ value to each and then following them around the flow by integrating the two simple ODEs (\ref{eqn:lagrangian} and \ref{eqn:velocity}) in time.\\

\begin{exercise}[Inhomogeneous distribution and sampling]
The parcels would very likely spread out in a non-homogeneous fashion, so that any numerical approximation of $\Psi(x,t)$ will become highly inaccurate wherever they are sparse (see the extra slides at the end for more). What can we do?
\end{exercise}

\section{The Lagrangian method}
\subsection{Trajectory calculations}
\subsection{Lagrangian models in practice}

\setcounter{equation}{2}	
	First calculate fluid parcel trajectories by
	integrating:
	\begin{equation}
	\frac{{d x}}{d t}={ u}({ x},t)
	\label{velocity}	
	\end{equation}
	
	and then calculate the evolution of air parcel properties, $\Psi$,
	following fluid parcels by modelling $S$ and integrating
	(\ref{velocity}).
	

	{\bf Advantages:} No underlying grid, so can represent air masses
	accurately as they thin to arbitrarily fine-scales. Also, no CFL
	criterion to limit time-step (i.e., parcels can travel far in one
	time-step).
	

	{\bf Disadvantages:} Velocity stirs air parcels so they are
	irregularly spaced and often far apart, making estimation of gradients
	difficult. In {\em kinematic models}, velocity is given but mixing
	between air-masses typically depends on concentration gradients. 
	

\section{The Semi-Lagrangian method}

\subsection{A compromise}

We learned that:
\begin{itemize}
	\item \underline{Eulerian frameworks} are limited by CFL criteria, thus requiring short time steps (e.g. I would need to use $\Delta t = 0.3s$ in my 10km GCM, instead of 4 minutes).
	\item \underline{Lagrangian frameworks} are most often impractical \footnote{\tiny we could remove parcels from regions where they are too abundant, add them where they are sparse} and/or locally inaccurate.
\end{itemize}

A better scheme could be chosen, in which we re-define the number and distribution of the fluid parcels at every time step. We choose parcels in this set as being those arriving at each node on a regularly spaced grid at the end of each time step. This will automatically regulate the number and distribution of the fluid parcels. \textbf{This is known as the \emph{semi-Lagrangian method} (Wiin-Nielsen, 1959).}\\

~

In practice, choose $t^n=n\Delta t$ and $x_j=n\Delta x$, then (1) can be approximated as:
 
 \begin{equation}
 \frac {\Phi(x_j,t^{n+1}) - \Phi (\tilde{x_j^n},t^{n})}{\Delta t}=\frac{S(x_j,t^{n+1}) + S (\tilde{x_j^n},t^{n})}{2}
 \label{SL}	
 \end{equation}

where $\Phi$ is the numerical approximation of $\Psi$ and $\tilde{x_j^n}$ is the estimated x coordinate of the \underline{departure point} of the trajectory originating at time $t$ and arriving at grid point and time: $(x_j,t^{n+1})$. The value of $\tilde{x_j^n}$ can be found by integrating (3) backwards over a time interval $\Delta t$, with initial condition: $x(t^{n+1})= x_j$. We shall need interpolation...


\subsection{A simple 1D example: temperature evolution.}

We want to predict the evolution of temperature for a fluid. We postulate that there are no sources/sinks of energy, so that temperature is conserved and we can say that, from the \emph{Lagrangian} perspective:
\begin{equation}
\frac{d T } {d t} = 0
\label{eq_Lagr}
\end{equation}

If we now take the \emph{Eulearian} view and we decide to predict the evolution of temperature at a particular location, the local change of temperature will be governed by temperature advection alone: 
\begin{equation}
\frac{\partial T } {\partial t} = -U \frac{\partial T } {\partial x}
\label{eq_Eul}
\end{equation}

where $T$ is temperature $[K]$, the variable we are predicting; $t$ is time $[s]$, $x$ is the zonal distance $[m]$ and $U$ is the zonal velocity $[ms^{-1}]$.\\

We can also solve \ref{eq_Lagr} by using the semi-Lagrangian (SL) method, which also enables a longer time step: 
\begin{equation}
\Delta t_{s}=n\Delta t_{e}
\end{equation}

where $\Delta t_{s}$ is the SL time step and $\Delta t_{e}$ is the Eulerian time step.
We will start from our original grid at time $t^0=0s$. \textbf{Remember that, with SL, you will be using the total derivative}, because the problem, seen from a Lagrangian perspective, is simply:
\begin{equation}
\frac{d T } {d t} = 0
\label{eqn_Lagrange}
\end{equation}

A semi-Lagrangian approximation to \ref{eqn_Lagrange} can be written in this form:

\begin{equation}
\frac {T(x_j,t^{n+1})-T(\tilde{x}^n_j,t^n)}{\Delta t} = 0
\end{equation}

where $\tilde{x}^n_j$ denotes the point of origin of a trajectory originating at time $t^n$ and arriving at point $(x_j,t^{n+1})$.

Since velocity $U$ is constant, it is quite easy to show that:
\begin{equation}
\tilde{x}^n_j=x_j-U \Delta t
\label{eqn_departure_point}
\end{equation}


\subsection{Semi-Lagrangian Methods in 2D: how to find $\tilde{x}^n_{i,j}$}

	Calculate short trajectories backward-in-time from a fixed grid and
	use them to evaluate the Lagrangian rates of change at every
	grid-point. Trajectory calculation can be cheap, because quite short. 
	For example, the two-stage mid-point method:
	\begin{eqnarray*}
		x_* & = & x^{n+1}_{i,j}-u(x^{n+1}_{i,j},t^n) \Delta t/2 \\
		\tilde{x}^n_{i,j} & = & x^{n+1}_{i,j}-u(x_*,t^{n+\frac{1}{2}}) \Delta t
	\end{eqnarray*}
	
	gives the {\em departure points}, $\tilde{x}^n_{i,j}$. The advection equation
	is then solved using simple methods such a trapezoidal scheme (see
	Durran's book, Chap.~6):
	\begin{equation}
	\frac{\Phi^{n+1}_{i,j}-\Phi(\tilde{x}^n_{i,j}, t^n)}{\Delta t}\approx \frac{1}{2}\left\{ 
	S^{n+1}_{i,j}+\tilde{S}^n_{i,j} \right\} 
	\end{equation}
	
	{\bf Advantages:} avoids CFL criterion for numerical stability
	(especially nonlinear ${\bf u}.\nabla {\bf u}$ term), allowing longer
	time-step. For same accuracy Ritchie {\em et al}\footnote{\BTi Ritchie {\em et al} (1995)
		\emph{Monthly Weather Rev.}, {\bf 114}, 135-146.\ETi},  found that the
	time-step of the ECMWF forecast model could be increased from 3 to 15
	minutes.
	
	{\bf Disadvantages:} the schemes are not positive
	definite. Interpolation is necessary from the grid to the departure
	points, which is equivalent to strong {\em numerical dissipation}. 
	

\subsection{Real world applications of the Semi-Lagrangian method}

\begin{center}	
			\setlength{\unitlength}{1 cm}
			\begin{picture}(1,7.5)
			\arakawa
			\put(1.93,1.93){$\bullet$}
			\put(2.1,1.7){$\Phi_{ij}$}
			\put(2,0.75){\vector(0,1){0.5}}
			\put(2,2.75){\vector(0,1){0.5}}
			\put(0.75,2){\vector(1,0){0.5}}
			\put(2.75,2){\vector(1,0){0.5}}
			\put(1.1,1.7){$u_{ij}$}
			\put(2.1,0.7){$v_{ij}$}
			\end{picture}
\end{center}
			
We aim to predict the value of $\Phi$ at time $t= n+1$, so $\Phi(x^{n+1})$, or $\Phi^{n+1}_{i,j}$. \\
			
Because we are projecting a Lagrangian calculation onto an Eulearian grid, we always know where we are going to be at time t=n+1: the \textbf{arrival point} $x^{n+1}=x_{i,j}$. \\
			
Unlike in purely Eulearian frameworks, we do not know a-priori where information is coming from; we must compute the \textbf{departure point} $\tilde{x}^n_{i,j}$\\
						
{\bf Target is the future (time = n+1): ~ ~ }  $\Phi^{n+1}_{i,j}$ \\
			
\medskip
			
{\bf Available to us now (time = n):  ~ ~ }  $\Phi^n_{i,j}, ~ u^n_{i,j}, ~ S^n_{i,j}$ \\
			
We integrate ${u}^n_{i,j}$ to find $\tilde{x}^n_{i,j}$, thus $\Phi (\tilde{x_{i,j}^n},t^{n})$ and $S(\tilde{x_{i,j}^n},t^{n})$.
			
Finally, we compute the future state like this: 

\begin{equation}
			\Phi^{n+1}_{i,j}  \approx  \Phi(\tilde{x}^n_{i,j}, t^n) 
			+ \Delta t \left\{ \frac{S(x_{i,j},t^{n+1}) + S (\tilde{x_{i,j}^n},t^{n})}{2}\right\} 
\end{equation}


\subsection{From first order to second order}

	
	It is perfectly possible to only use a single time level to compute departure points. However, if we were to simply make use of $u^n_{i,j}$ to find $\tilde{x}^n_{i,j}$, we would inevitably end up with a first order scheme:
	
	\begin{eqnarray*}
		\tilde{x}^n_{i,j} & = & x^{n+1}_{i,j}-u(x^{n+1}_{i,j},t^n) \Delta t \\
	\end{eqnarray*}
		
	Instead, we saw previously that we can use a two-stage mid-point method:
	\begin{eqnarray*}
		x_* & = & x^{n+1}_{i,j}-u(x^{n+1}_{i,j},t^n) \Delta t/2 \\
		\tilde{x}^n_{i,j} & = & x^{n+1}_{i,j}-u(x_*,t^{n+\frac{1}{2}}) \Delta t
	\end{eqnarray*}
	
	Ideally we would want to compute the mid-point, $x_*$, based on current and future velocities, but this would result in an implicit scheme (and expensive iteration).\\
	
	It is possible, instead, to use interpolation and \emph{forward extrapolation} in time to keep the scheme explicit and yet second-order. When computing this term: $u(x_*,t^{n+\frac{1}{2}})$, we can write:
	\begin{equation}
	u(t^{n+\frac{1}{2}})=\frac{3}{2}u(t^n)-\frac{1}{2}u(t^{n-1})
	\end{equation}
	
	\underline{Note that we now need to carry two time levels.}
	

\section{Semi-Implicit method combined with the Semi-Lagrangian method}

Explicit methods tend to have a time-step restriction for numerical
stability at a given spatial resolution, summarised by the CFL criterion:
\[
\alpha=\frac{c \Delta t}{\Delta x}< 1
\]

where $c$ is the fastest speed of information propagation. Generally,
flows are {\em unbalanced} meaning that there are fast waves involving
fluid movement that is not related to PV. Such fast waves can limit
the time-step. Then, \emph{the semi-Lagrangian method} does not
help, because it only ensures stability  \underline{with respect to advection}.

\begin{itemize}
\item
Sound waves are fastest but filtered out by the anelastic
approximation.

\item
Gravity waves are not filtered out unless more severe balance
approximations are made - therefore $c_{GW} (> U)$ limits $\Delta t$.

\item
Semi-implicit methods lift this restriction by treating the {\bf
gravity wave terms} implicitly while the rest of the terms in the equations are
explicit.
\end{itemize}

All global NWP models currently use semi-implicit, semi-Lagrangian
methods attempting to achieve stability for long time-steps ($1 < \alpha <
10$).

\section{Extras}

\subsection{Eulerian and Lagrangian rates of change}
\subsection{Monotonicity and Positive Definiteness}

	\begin{itemize}
		\item
		Many properties are carried with the flow . A statement of {\bf
			material conservation} is:
		\begin{equation}
		\lagd{q}{t}=S
		\label{lagdef}
		\end{equation}
		
		where the {\em Lagrangian derivative} is the rate of change following a fluid parcel:
		\begin{equation}
		\lagd{q}{t}=\partd{q}{t}+{\bf u}.\nabla q
		\end{equation}
		
		Some implications for solutions (for conserved properties where $S=0$) are:
		
%		\begin{tabular}{lc}
%			\begin{minipage}[l]{0.6\textwidth}
				
				\begin{itemize}
					\item
					{\bf monotonicity preservation:} no new maxima or minima in $q$ can appear
					
					\item
					{\bf positive definiteness:} if {\em tracer} has $q\ge 0$ everywhere at
					initial time, no negative values can appear (e.g., humidity mixing
					ratio).
				\end{itemize}
				
%			\end{minipage}
%			\begin{minipage}[c]{0.4\textwidth}
				\rotatebox{0}{\includegraphics[width=0.6\textwidth]{Figures/q_contour.eps}}
%			\end{minipage}
%		\end{tabular}
		
	\end{itemize}

\subsection{Kinematic Lagrangian model example} 

	
	This example shows trajectories calculated from analysed winds
	(gridded in space and time). Initiated from the coordinates of the
	flight of a research aircraft\footnote{Methven,
		J. {\em et al.}  (2006) \emph{J. Geophys. Res.}, {\bf 111}, D23S62,
		doi:10.1029/2006JD007540. }, July 2004. Ozone concentrations
	have been integrated forward in time along the trajectories using a
	Lagrangian model for photochemistry and mixing, initialised with the
	aircraft measurements of constituent concentrations.
	\begin{center}
		\begin{tabular}{cc}
			\rotatebox{90}{\includegraphics[width=0.32\textwidth]{Figures/Ftj0_x4_2004715.eps}} &
			\rotatebox{90}{\includegraphics[width=0.32\textwidth]{Figures/Fxvy0_x8_y10_2004715.eps}}
		\end{tabular}
	\end{center}
	
	The crosses show the coordinates and ozone measurements made during
	three further flights intercepting the air-mass downstream.


\subsection{Contour dynamics}
\subsection{Balanced dynamics} 

	
	{\em Dynamics} involves solution for velocity (and pressure) as well as
	tracer equations. Neither of these are conserved variables.
	
	\vspace{0.5cm}
	However, in {\em balanced dynamics}, potential vorticity (PV) is carried as a
	{\em tracer} and all other variables are obtained by {\em inverting PV}.
	
	\vspace{0.5cm}
	Simplest example is the barotropic vorticity equation where
	
	\[
	q=f+\nabla^2 \psi
	\]
	
	This is inverted to obtain the streamfunction and geostrophic velocity:
	\[
	\psi=\nabla^{-2} (q-f)\;\;\;\;\;
	u_g=-\partd{\psi}{y}  \;\;\;\;\;
	v_g=\partd{\psi}{x} \]
	
	The inversion operation typically does not depend on time, but is a
	difficult {\em boundary value problem} akin to integration.
	


\subsection{Models for balanced dynamics} 
	
	A family of models is obtained by treating the {\em material
		conservation} and {\em PV inversion} operations in an Eulerian or
	Lagrangian framework.
	
	\vspace{0.5cm}
	{\bf Contour dynamics}\footnote{\BTi
		Zabusky {\em et al} (1979) \emph{J. Comput. Phys.}, {\bf 30},
		96-106.\ETi}
	- a fully Lagrangian method
	
	\begin{itemize}
		\item
		Nodes are placed around PV contours (more nodes where curvature is higher). 
		\item
		The inversion step to obtain velocity at the nodes is done by
		integration around PV contours (without the use of a grid).
		\item
		Nodes are stepped forward by calculating trajectories (4th order
		Runge-Kutta method), giving updated PV contours.
		\item
		Repeat the inversion and trajectory steps.
	\end{itemize}
	
	{\bf Advantages:} No underlying grid, so accurate in principal and no CFL $\Delta t$  restriction.
	
	{\bf Disadvantages:} As the contours stir into convoluted shapes, the
	number of nodes increases exponentially to retain accuracy. Inversion
	by contour integration cost scales with {\em nodes} $\times$ {\em
		contours} and becomes too expensive.
	
	Also implicit assumption is that PV values are discretised changes
	wave dynamics.
	

\section{Hybrid models}
\subsection{Contour-advective semi-Lagrangian (CASL)}
	
			Potential vorticity is treated in a Lagrangian way (by calculating trajectories of nodes on tracer contours).		The example in Fig. \ref{fig:CASL} shows a single contour advected by an unsteady polar vortex. Note the finescale filamentary structure.

\begin{center}
\begin{figure}[h!]
\includegraphics[angle=90,width=0.35\textwidth]{Figures/T170F6_t170.eps}
\label{fig:CASL}
\caption{CASL as illustrated in Dritschel et al. (1997)}
\end{figure}
\end{center}

	But velocity and other {\em wave-like} variables are stored on a fixed
	grid\footnote{\BTi Dritschel, D.G. and Ambaum, M.H.P. (1997) \emph{ Quart. J. Roy. Meteor. Soc.}, {\bf 123},	1097-1130.\ETi}.
	
	{\bf Advantages:} Avoids numerical dissipation of conserved variables
	and the CFL criterion. Potential vorticity inversion to obtain {\em
		balanced velocity} is cheap because contour-to-grid conversion
	estimates {\em coarse-grained PV} on the velocity grid and then inversion
	can be done by fast Fourier transform methods.
	
	{\bf Disadvantages:} Allowing for non-conservation (e.g.,
	sources/sinks) is difficult because such processes violate
	monotonicity preservation and must introduce new $q$-contours.
	
	
%%%%%%%%%%%%%%%%%%%%%%%%%%%%%%%%%%%%%%%%%%%%%%%%%%

%%%%%%%%%%%%%%%%%%%%%%%%%%%%%%%%%%%%%%%%%%%%%%%%%%
\newpage	
\chapterimage{2613-1477-max.jpg} % Chapter heading image
\chapter{The Spectral Method}	

%%%%%%%%%%%%%%%%%%%%%%%%%%%%%%%%%%%%%%%%%%%%%%%%%%%%	
\chapterimage{2613-1477-max.jpg} % Chapter heading image
\chapter{Numerical Modelling and HPC}	
	%%%%%%%%%%%%%%%%%%%%%%%%%%%%%%%%%%%%%%%%%%%%%%%%%%%%
	
	\vspace{1em} 

%%%%%%%%%%%%%%%%%%%%%%%%%%%%%%%%%%%%%%%%%%%%%%%%%%
% APPENDICES
\newpage	
% ----------------------------------------------------------------------------------------
% 	APPENDICES
% ----------------------------------------------------------------------------------------

\appendix
\chapter{Stability analysis in 1D}

\section{Simple advection}


\section{Finite Difference Methods: 1D Examples}
\subsection{The Advection Equation}
\textsl{Partial differential equations, order of approximation, von
	Neumann stability analysis, computational modes, numerical
	dispersion, numerical dissipation.}

In many cases we need to solve equations containing
derivatives with respect to \emph{more than one} variable (e.g.~time,
$t$, and space, $x$).  These are called \emph{partial} differential
equations (PDEs). The solution of partial differential equations using
numerical methods is probably the most important and also the most
complex part of numerical analysis.

A quick point about notation before we begin.  Derivatives in ODEs are
written with the `straight~d',
\BEQ \frac{\mathrm{dT}}{\mathrm{d}t}  \;\;\; , \EEQ
but {\em partial derivatives} in PDEs are written with the `curly~d',
\BEQ \frac{\partial T}{\partial t} \;\;\; , \;\;\; \frac{\partial T}{\partial x} . \EEQ
The first means the derivative of $T$ with respect to time at a fixed
location ($x$) and the second means the derivative of $T$ with respect
to $x$ at a fixed time.

In atmosphere or ocean science we often consider quantities that are
carried with the flow without changing. Examples include the mixing
ratio of an unreactive trace gas or potential temperature. Such
quantities can be used to mark air parcels and follow them and
therefore they are called {\em tracers}. If the wind field ${\bf
	u}({\bf x}, t)$ is known the {\em trajectories} of air parcels can be
found by numerically integrating the ODE:
\begin{equation}
	\frac{D{\bf x}}{Dt}={\bf u}({\bf x}, t)
\end{equation}
where ${\bf x}$ is the position of an air parcel and the capital
$D/Dt$ is used to mean rate of change following an air parcel. 
The tracer value, $\theta_j$, does not change following parcel $j$ so
if we know the values $\theta_j$ for all parcels at time $t_i$ and
their initial positions, we can use their trajectories
to infer the new distribution of tracer at a later time $t_f$. A model
based on following fluid parcels is called a {\em Lagrangian model}.

However, many atmospheric processes occur relative to a fixed
position. Examples include a pollution source from a city and
temperature fluxes from the land or ocean. Sometimes it is better
devise a model which considers atmospheric changes relative to fixed
points. These are called {\em Eulerian
	models}. What happens when air masses carrying tracers pass over a
point?

\begin{figure}
	\begin{center}
		\scalebox{0.5}{\includegraphics{Figures/advection.pdf}}
	\end{center}
	\caption{\textsl{Temperature at a point increases when wind blows from warm side.}}
	\label{fig:ken}
\end{figure}

In Figure~\ref{fig:ken}, the wind is bringing air from the $-x$
direction. If the air `upstream' is warmer than the air `downstream',
the observer will see the temperature increasing. The rate of change of
temperature observed must depend on both the magnitude of this
gradient and the speed at which the air is moving, {\em i.e.,}
\begin{equation}
	\frac{\partial T}{\partial t} =- u \frac{\partial T}{\partial x}.
	\label{advectioncont}
\end{equation}
where the left hand side means the rate of change at a given position
$x$ and the right hand side is called the {\em advection term}. $u$ is
the wind component and $\partial T/\partial x$ means the gradient in
the $x$-direction at a given time. 

\subsubsection{Order of approximation in finite differences}

In the special case of uniform velocity we can find a solution of the
form $T(x,t)=F(x-ut)$, but for general flows we cannot find an analytic
solution. Perhaps the simplest numerical solution is obtained using a
finite difference method where derivatives are replaced by finite
differences. Time and space are divided into finite segments, the time
step $\Delta t$ and a space step $\Delta x$, and the points between
steps are denoted $t_n=n\Delta t$ and $x_i=i\Delta x$.  We will use
the shorthand: 
\BEQ T(x_i,t_n) = T_i^n . \EEQ 
where superscript $n$ in
$T_i^n$ refers to the $n$th timestep.  It does \emph{not} mean $T_i$
to the power of $n$.

The advection equation contains $\partial T / \partial t$ and
$\partial T / \partial x$. There are three obvious ways of
discretizing $\partial T / \partial t$ at the $n$th timestep and $i$th
spatial point:
\BEQ \frac{T_i^{n+1} - T_i^n}{\Delta t} \;\;\; , \;\;\; 
\frac{T_i^n - T_i^{n-1}}{\Delta t} \;\;\; , \;\;\;
\frac{T_i^{n+1} - T_i^{n-1}}{2\Delta t} . \EEQ

These are called the forward difference, backward difference and
centred difference in time, respectively.  Similarly, there are three
obvious ways of discretizing $\partial T / \partial x$ at the $n$th
timestep and $i$th spatial point:
\BEQ \frac{T_{i+1}^n - T_i^n}{\Delta x} \;\;\; , \;\;\; 
\frac{T_i^n - T_{i-1}^n}{\Delta x} \;\;\; , \;\;\; 
\frac{T_{i+1}^n - T_{i-1}^n}{2\Delta x} . \EEQ

By combining these possible discretizations --- three for $\partial T
/ \partial t$ and three for $\partial T / \partial x$ --- we could
write down \emph{nine} possible numerical schemes for the advection
equation.  At first sight the nine schemes all seem equally
reasonable, but they are not. 

Firstly, the centred scheme has a higher {\em order of
	approximation}. This can be seen by expressing temperature at position
$x+\Delta x$ in terms of a Taylor expansion about position $x$:
\BEQ T(x+\Delta x) = T(x) + T'(x)\Delta x + \frac{1}{2} T''(x) (\Delta x)^2 + 
\frac{1}{6} T'''(x) (\Delta x)^3 +O(\Delta x^4) \EEQ

where $T'=\partial T/\partial x$, $T''=\partial^2 T/\partial x^2$ and so on.
Re-arrangement shows that:
\BEQ
T'(x)=\frac{ T(x+\Delta x)-T(x)}{\Delta x}-\frac{1}{2} T''(x) \Delta x+...
\EEQ

so the error in approximating the first derivative by a forward
difference is proportional to $\Delta x$ or {\em first order} assuming
that temperature variations have a characteristic lengthscale much
longer than the grid-scale ($\Delta x/L\ll 1$)\footnote{Scaling
	arguments assume wavelike structure and $\partd{}{x}\sim
	\frac{1}{L}=k$.}.

The order of approximation for the centred difference is obtained by subtracting the expansion for $T(x-\Delta x)$:
\BEQ T(x-\Delta x) = T(x) - T'(x)\Delta x + \frac{1}{2} T''(x) (\Delta x)^2 - 
\frac{1}{6} T'''(x) (\Delta x)^3 +O(\Delta x^4) \EEQ

from $T(x+\Delta x)$ giving a second order leading error $\sim
\frac{1}{3}T'''(x)\Delta x^2$. If the temperature variations are
smooth, the higher order scheme will be more accurate.

\subsubsection{Numerical stability: The FTCS advection scheme}

Let us try using the forward difference in time and the
centred difference in space (FTCS for short).
The scheme is:
\BEQ \frac{T_{i}^{n+1}-T_{i}^{n}}{\Delta t} + u \frac{T_{i+1}^{n} - T_{i-1}^{n}}{2\Delta x} = 0 \EEQ
We can rewrite this equation to get:
\BEQ T_{i}^{n+1} = T_{i}^{n} - \frac{u\Delta t}{2\Delta x} \left[ T_{i+1}^{n} - T_{i-1}^{n} \right]
\label{ftcsadvection}\EEQ

At each time step, $n$, we apply this equation at each spatial point,
$i$.  A computer program to do this would therefore have a loop over
$i$ \emph{within} a loop over $n$.

A possible way of examining the stability of this scheme is the
\emph{von Neumann stability analysis}.
To understand how this works, first note that a particular solution
of the continuous advection equation (\ref{advectioncont}) is
\BEQ T(x,t)=\cos k(x-ut) .\EEQ
{\em Exercise: Show that this is indeed a solution.}

The von Neumann method starts with a {\em trial} solution:
\BEQ T_i^n = \cos kx_i , \EEQ
and considers what the numerical scheme does over one time step.
We know that the numerical scheme is just an approximation:
it might change the amplitude of the cosine
wave and the phase shift might be wrong.  Let us write 
the outcome from the numerical scheme after one time step as:
\BEQ T_i^{n+1} = A\cos (kx_i-\phi),\EEQ
where $A$ and $\phi$ are unknowns that we want to work out.  If our
numerical scheme were exact, we would find $A=1$ and $\phi=k u \Delta
t$.  When we substitute the above equations for $T_i^n$ and
$T_i^{n+1}$ in the numerical scheme (\ref{ftcsadvection}) we find:
\BEQ A\cos (kx_i-\phi) = \cos kx_i - \frac{u \Delta t}{2\Delta x}\left( \cos kx_{i+1} - \cos kx_{i-1} \right). \EEQ
Remembering that $x_{i+1}=x_i+\Delta x$ and $x_{i-1}=x_i-\Delta x$,
and using the fact that $\cos (a+b) = \cos a\cos b -\sin a\sin b$,
we can rearrange this equation to find:
\BEQ A (\cos kx_i\cos\phi +\sin kx_i\sin\phi) = \cos kx_i + \frac{u\Delta t}{\Delta x}\sin kx_i\sin k\Delta x. \EEQ
This equation should be valid for all $x_i$.
Equating the terms in $\cos kx_i$ and $\sin kx_i$ we find, respectively:
\begin{eqnarray}
	A \cos \phi & = & 1 \\
	A \sin \phi & = & \frac{u\Delta t}{\Delta x}\sin k\Delta x .
\end{eqnarray}
By adding the squares of these equations we find that
\BEQ A^2 = 1+\left(\frac{u\Delta t}{\Delta x}\sin k\Delta x\right)^2. \EEQ 
We see that the amplification factor $A$ is always bigger than one: in this
sense the scheme is \emph{unconditionally unstable} and the solutions
will amplify every timestep.

\subsubsection{Computational modes: The CTCS advection scheme}

So it's back to the drawing board. Let us try using the centred difference in time and the
centred difference in space (CTCS for short):
\BEQ \frac{T_{i}^{n+1}-T_{i}^{n-1}}{2\Delta t} + u \frac{T_{i+1}^{n} - T_{i-1}^{n}}{2\Delta x} = 0 \EEQ
We can rewrite this equation to get:
\BEQ T_{i}^{n+1} = T_{i}^{n-1} - \frac{u\Delta t}{\Delta x} \left[ T_{i+1}^{n} - T_{i-1}^{n} \right] \label{ctcsmodel} \EEQ
The quantity $u\Delta t / \Delta x$ is dimensionless.  It appears so often in
numerical schemes for the advection equation that it has its own name --- the
\emph{Courant number}:
\BEQ \alpha = \frac{u\Delta t}{\Delta x} .\EEQ

The von Neumann stability analysis for the CTCS scheme is most easily
done using complex notation for the trial solution, 
\BEQ T_i^n=e^{ikx_i}\;\; ; \;\;\;\; T_i^{n+1}=ST_i^n \EEQ
where $S=Ae^{i\phi}$ expresses 
the change in amplitude and phase over one time-step.
Plugging the trial solution into the model (\ref{ctcsmodel}) yields:
\BEQ
S=S^{-1}-\alpha\left[ e^{ik\Delta x}-e^{-ik\Delta x} \right]
\EEQ 

which can be re-arranged as a quadratic equation
for the complex factor, $S$: 
\BEQ
S^2+S\alpha 2i\sin k\Delta x -1=0
\EEQ

which has two roots:
\BEQ
S_{\pm}=-i \alpha \sin k\Delta x\pm \sqrt{1-\alpha^2 \sin^2 k\Delta x}
\label{asoln}
\EEQ

If the Courant number $|\alpha | \le 1$, the number under the root must
be positive since $\sin^2 k\Delta x \le 1$. 

{\em Exercise: Show that when $|\alpha | \le 1$ the magnitude of the
	amplification factor $|S|^2=S^* S=1$, meaning that the scheme is
	stable. However, when $|\alpha | > 1$ the $S_{-}$ solution is
	unstable.}

This is an improvement on the FTCS scheme, which was unconditionally
unstable. This is the single most important result in the numerical
modelling of fluid flows.  It is called the Courant-Friedrichs-Lewy
(CFL) condition.  What it boils down to is that, for a given
grid-spacing $\Delta x$, the time step must satisfy \BEQ \Delta t <
\frac{\Delta x}{u} . \EEQ The physical interpretation of this condition
is that \emph{the scheme is unstable if fluid parcels move more than
	one gridbox in one timestep}.

The solution with +ve real part in (\ref{asoln}) does not change sign
every timestep and is called the {\em physical mode}. However, the
solution with the -ve square root alternates sign every step which is
unphysical behaviour and is called the {\em computational mode}. The
relative amplitudes of the two modes is determined by projection from
the initial conditions. The implication is that although the CTCS
scheme is stable for $|\alpha | \le 1$ we can expect unphysical
oscillations that are worst at the grid-scale (where $\sin k\Delta
x=1$).

\begin{figure}
	%\begin{center}
	\mbox{\scalebox{0.1}{\includegraphics{Figures/adv-ctcs.png}}
		\scalebox{0.1}{\includegraphics{Figures/adv-upstream.png}}}
	%\end{center}
	\caption{\textsl{Comparing numerical solutions for the advection of a top-hat distribution by uniform flow. Left: CTCS solution exhibits unphysical undershoots and overshoots. Right: Upstream scheme remains monotonic.}}
	\label{fig:ctcs}
\end{figure}


\subsubsection{Numerical dispersion}

The phase shift in one time-step for a wave moving with speed $c$
would be $\phi=-kc\Delta t$. By comparing the imaginary part of
$e^{i\phi}$ with (\ref{asoln}) we obtain the phase speed as
represented by the numerical scheme: 
\BEQ kc\Delta t=\sin^{-1}\left\{
\alpha \sin k\Delta x \right\} \EEQ

While for the exact solution the tracer pattern must move with the
uniform velocity $u$, irrespective of its shape or lengthscale, the
numerical solution moves with a phase speed dependent on the
wavenumber of the tracer pattern, $k$. The ratio of numerical to exact speed, $\alpha^*/\alpha$ is plotted in Figure \ref{fig:disperse}, which shows that long waves ($k\Delta x \ll 1$) move almost at the flow speed, but short waves move much more slowly.\\
Waves in the CTCS numerical model are {\em
	dispersive}, even though they should not be. Note also that the
computational mode moves with the same phase speed as the physical
mode but in the opposite direction to the flow. The net effect is
severe for sharp features where small scale waves propagate out behind
the leading edge of tracer.

%\begin{figure}
\begin{figure}
	%\centering
	\scalebox{0.19} {\includegraphics{Figures/alphastar-vs-kdx.png}}
	\caption{\textsl{Ratio of numerical speed to flow speed as a function of wavenumber, $k\Delta x$, for the CTCS numerical scheme.}}
	\label{fig:disperse}
\end{figure}
%\end{figure}


\subsubsection{Numerical dissipation}

The CTCS scheme maintained wave amplitude for all
wavenumbers. However, many schemes result in a decay of amplitude with
time which is faster at short wavelengths. One example is the {\em
	first order upstream scheme}:
\BEQ
\frac{q_i^{n+1}-q_i^n}{\Delta t}+u\frac{q_{i}^n-q_{i-1}^n}{\Delta x}=0
\EEQ

which is ``upstream'' for $u>0$ since the interval $x_{i-1}\rightarrow
x_i$ is upstream of point $x_i$. 

{\em Exercise: Show using von Neumann stability analysis that the
	upstream scheme results in a wave amplitude $|S|$ that decays with
	time.}

In the model, short waves decay more rapidly than long waves, with the
net result that the tracer distribution becomes smoother with
time. This effect is called {\em numerical dissipation} or {\em
	numerical diffusion}. The advantage is that unphysical overshoots are
eliminated (Fig.~\ref{fig:ctcs}).

\subsection{The Diffusion Equation}

\textsl{Diffusion, partial differential equations, 
	von Neumann stability analysis, boundary conditions.}

A second major transport process operating in gases, liquids and
solids is diffusion. As an example, think of the layers of soil near
the surface of the Earth. In the top layers soil temperature varies
with air temperature, but deep down below the surface, the temperature
is more-or-less constant. Over time, heat is transferred between the
surface and lower levels.

There is a {\em flux} of heat (denoted by $F$) through the surface
of the Earth defined so that $F>0$ implies a flow of heat in the
direction of increasing depth ($z$). Each layer of soil will have a
flux of heat in from above and a flux of heat out from below. If
the flux in and the flux out are equal, the temperature of the layer
will remain constant. The temperature will only change if there is a
{\em flux gradient}. We can define $F$ so that the rate of change of
temperature $T$ is equal to minus the gradient of $F$,
\begin{equation}
	\partd{T}{t} = -\partd{F}{z}.
	\label{eq:flux-tend}
\end{equation}
{\em i.e.}, if more heat flux enters than leaves a slab of soil,
the temperature of the slab must increase.

When an object is placed into contact with an environment with lower
temperature, heat can flow from hot to cold, as with the cup of tea
example. In a solid, molecules in hot regions jiggle more than in cold
regions and the heat is transferred as the jiggling molecules push and
pull their neighbours through intermolecular forces. In a fluid,
molecules can move and collide with each other resulting in a {\em
	random walk}. If a molecule travels from a hot region and collides
with a molecule in a neighbouring colder region it will transfer some
energy from hot to cold. Here we
assume that all these physical processes result in a flux that is {\em
	proportional} to the temperature gradient locally, {\em i.e.,} giving
the {\em flux-gradient relation}:
\begin{equation}
	F = -D(z) \partd{T}{z}.
	\label{eq:flux-val}
\end{equation}
The {\em diffusion coefficient} $D(z)$ is positive and the `--' sign
shows us that the flux is {\em down-gradient} from hot to cold.

Plugging eqn~(\ref{eq:flux-val}) into eqn~(\ref{eq:flux-tend})
gives:
\begin{eqnarray}
	\partd{T}{t} & = & \partd{}{z} \left(
	D(z) \partd{T}{z}
	\right)		\\
	& = & D \partd{^{2}T}{z^2}
	~~\mbox{if $D$ is constant}
	\label{diffeqn}
\end{eqnarray}
Physically, the diffusion equation can be interpreted as describing
the way heat travels from warm areas to cold areas. The diffusion
equation will take an initial temperature distribution $T(z)$ and
`smear it out' over time, eventually leaving a constant temperature
gradient $\partial T/\partial z$.

This is a \emph{second-order} partial differential
equation, because $T$ is differentiated twice with respect to $z$ on the 
right-hand side.  Compare this with the advection equation from the previous 
chapter, which was a \emph{first-order} partial differential equation.
The diffusion equation is therefore slightly more complicated.

{\em Exercise: Show that $T(z,t) = \mathrm{e}^{-D k^2 t} \cos kz$ is a
	solution of the diffusion eqn~(\ref{diffeqn}).}

\subsubsection{Numerical schemes for the diffusion equation}

Let us look at how we might solve the diffusion equation on a
computer. One way forward is again to replace the derivatives with
finite differences. As before, we can consider a series of {\em
	snapshots} of the temperature $T$ each separated by the time-step
$\Delta t$. The depth $z$ can be similarly {\em discretised}: we can divide
the distance between the surface and bottom of the soil at $z=H$ into
$J$ slabs of depth $\Delta z=H/J$ and consider the characteristic
temperature of each slab (Fig.~\ref{soil}).

\begin{figure}
	\begin{center}
		\scalebox{0.15}{\includegraphics{Figures/temp-diffuse.png}}
	\end{center}
	\caption{\textsl{Diffusion of heat through soil near the Earth's surface.}}
	\label{soil}
\end{figure}

As in the previous chapter, to shorten the equations we define
\BEQ T(z_i,t_n) = T_i^n . \EEQ
To discretize the right-hand side of the diffusion equation, 
consider the following two Taylor expansions:
\BEQ T(z+\Delta z) = T(z) + T'(z)\Delta z + \frac{1}{2} T''(z) (\Delta z)^2 + 
\frac{1}{6} T'''(z) (\Delta z)^3 +O(\Delta z^4) \EEQ
\BEQ T(z-\Delta z) = T(z) - T'(z)\Delta z + \frac{1}{2} T''(z) (\Delta z)^2 - 
\frac{1}{6} T'''(z) (\Delta z)^3 +O(\Delta z^4) \EEQ
Adding these two expansions together we find that
\BEQ T(z+\Delta z) + T(z-\Delta z) = 2T(z) + T''(z) (\Delta z)^2 +O(\Delta z^4), \EEQ
which we can re-arrange to give
\BEQ T''(z) = \frac{T(z+\Delta z) - 2T(z) + T(z-\Delta z)}{(\Delta z)^2}+O(\Delta z^2) .\EEQ
Our discretization of the second derivative $\partial^2 T / \partial z^2$ is therefore a {\em second order approximation} and 
at the $i$th grid-point and the $n$th time-point can be written:
\BEQ \frac{T_{i+1}^n-2T_i^n+T_{i-1}^n}{(\Delta z)^2} . 
\label{d2t}
\EEQ
To discretize the left-hand side of the diffusion equation, we could choose
a forward in time, backward in time, or centred in time scheme.  Here, we will
only study the first of these three possible choices, the FTCS scheme:
\BEQ \frac{T_i^{n+1}-T_i^n}{\Delta t} = D \; \frac{T_{i+1}^n-2T_i^n+T_{i-1}^n}{(\Delta z)^2} 
\label{ftcs:diff}
\EEQ
which can be re-arranged to give
\BEQ T_i^{n+1} = T_i^n + \gamma ( T_{i+1}^n -2T_i^n + T_{i-1}^n ) \label{diffusionscheme}\EEQ
where we have defined a dimensionless model parameter
\BEQ \gamma = \frac{D\Delta t}{(\Delta z)^2} . \EEQ

\subsubsection{Stability analysis}

As for advection, the stability of our numerical scheme can
be assessed using von Neumann stability analysis.  Remember that an
exact solution of the continuous diffusion equation (\ref{diffeqn}) is
$T(z,t) = \mathrm{e}^{-D k^2 t} \cos kz$.  As $t$ increases, the phase
of the cosine wave does not change but its amplitude is reduced.  We
must determine if our numerical scheme also has these properties.
Once again we begin with the {\em trial solution}:
\BEQ T_i^n = \cos kz_i , \EEQ
and examine the outcome from the numerical scheme after one time step:
\BEQ T_i^{n+1} = A\cos (kz_i-\phi),\EEQ
where $A$ and $\phi$ are the unknown amplitude and phase that we want
to work out.  When we substitute these expressions for $T_i^n$ and
$T_i^{n+1}$ into the numerical scheme (\ref{diffusionscheme}) we find:
\BEQ A\cos (kz_i-\phi) = \cos kz_i + \gamma ( \cos kz_{i+1} - 2\cos kz_i + \cos kz_{i-1} ). \EEQ
Remembering that $z_{i+1}=z_i+\Delta z$ and $z_{i-1}=z_i-\Delta z$,
we find:
\BEQ A (\cos kz_i\cos\phi +\sin kz_i\sin\phi) = \cos kz_i + 2\gamma (\cos k\Delta z -1) \cos kz_i . \EEQ
Since this equation must be valid for all $z_i$, we can equate
the terms in $\cos kz_i$ and $\sin kz_i$ to find, respectively:

\begin{eqnarray}
	A \cos \phi & = & 1 + 2\gamma (\cos k\Delta z -1) \\
	A \sin \phi & = & 0 .
\end{eqnarray}

Since we are not interested in the case when $A=0$, the second of
these equations tells us that $\phi=0$.  In other words, the phase of
the cosine wave is not changed by our numerical scheme.  This is good
news because it agrees with the true solution.  Substituting $\phi=0$
into the first equation, and remembering that $\cos 0=1$,
we obtain a formula for the amplification factor:
\BEQ A = 1 + 2\gamma (\cos k\Delta z -1) \EEQ
If our numerical scheme is to be stable for a general diffusion
problem, then it must be stable for all possible values of $k$.  There
will always be a value of $k$ for which $\cos k\Delta z = -1$, and it
is this extreme value of $k$ that will determine the stability.  For
this value of $k$ the amplification factor becomes
\BEQ A = 1 -4 \gamma .\EEQ
The stability of the numerical solution depends solely on the value of
$\gamma$.  There are three cases we need to consider:

\begin{tabbing}
	$0<\gamma < \frac{1}{4}$~~~~ \= Implies $0< A<1$: the solution decays monotonically like the exact solution \\
	$\frac{1}{4}<\gamma < \frac{1}{2}$ \> Implies $-1< A<0$: the solution decays but oscillates unphysically \\
	$\gamma > \frac{1}{2}$ \> Implies $A<-1$: the solution is unstable and oscillates with growing amplitude
\end{tabbing}

We conclude that the FTCS numerical scheme for the diffusion equation is
\emph{conditionally stable}.  The condition for realistic behaviour is
$\gamma<1/4$ or equivalently:  
\BEQ \Delta t < \frac{(\Delta z)^2}{4D} . \EEQ

\subsection{Boundary conditions}

We need to consider \emph{boundary conditions} when solving the
diffusion equation or the advection equation.  For example, to apply
the spatial discretization (\ref{d2t}) to a soil layer we require
model layers above and below it (see Figure~\ref{soil}). We run into
problems with the top model layer, because there is no layer above and
we must employ a {\em boundary condition}.

These are commonly-used boundary conditions:

\paragraph{No flux boundary conditions} 

assume that the fluxes across the top and bottom of the domain are
zero. We can discretise eqn~(\ref{eq:flux-tend}) as follows:

\begin{equation}
	\frac{T^{n+1}_i-T^n_i}{\Delta t} \approx 
	-\frac{F_{i+1/2}-F_{i-1/2}}{\Delta z} 
	\label{fluxdiff}
\end{equation}

where the flux gradient has been estimated by a finite difference
between the top and bottom interface of each layer. The flux at the
interface $z_{i+1/2}$ midway between levels $z_i$ and $z_{i+1}$ is
also defined by centred difference of eqn~(\ref{eq:flux-val}):
\BEQ F_{i+1/2}\approx -D
\frac{(T^n_{i+1}-T^n_i)}{\Delta z}.
\label{flux-cen}  
\EEQ 
The no flux boundary condition is implemented in the top layer ($i=1$) using
$F_{1/2}=F(0)=0$ in eqn.(\ref{fluxdiff}) giving:
\begin{eqnarray}
	\frac{T^{n+1}_1-T^n_1}{\Delta t} & = & 
	-\frac{F_{3/2}}{\Delta z} \\\nonumber
	& = & D\frac{T^n_2-T^n_1}{(\Delta z)^2}
\end{eqnarray}
and similarly $F_{J+1/2}=F(H)=0$ is used in the equation for the
bottom layer. 

{\em Exercise: Show that the numerical scheme for the diffusion
	equation at any level-$i$ (away from the boundaries) obtained by
	substituting eqn~(\ref{flux-cen}) into eqn~(\ref{fluxdiff}) is
	identical to eqn~(\ref{ftcs:diff}).}

\paragraph{Fixed value boundary conditions} 

assume that the temperature at the boundaries is specified. In the
soil example, the top boundary is at air temperature and the lower
boundary is at the underlying rock temperature. In our numerical
scheme the boundary conditions are best included when the fluxes at
the boundaries are estimated using eqn.(\ref{flux-cen}). For example,
at the top boundary:
\BEQ
F_{1/2}=F(0)\approx -D \frac{(T^n_1-T_{AIR})}{\frac{1}{2}\Delta z}
\EEQ
and then this is plugged into eqn.(\ref{fluxdiff}).

The boundary conditions have a profound effect on the physics of the
solution and the equilibrium temperature profile that the soil tends
towards. With no flux boundary conditions, the vertical average of
temperature must stay constant with time because no heat enters or
leaves the soil. However, the temperature at the top and bottom of the
soil layer can change. With fixed value boundary conditions, the
temperature at the boundaries cannot change but the vertical average
can, implying a net warming or cooling of the soil.

\paragraph{Periodic boundary conditions}

\begin{figure}
	\begin{center}
		\scalebox{0.8}{\includegraphics{Figures/adv-grid.png}}
	\end{center}
	\caption{\textsl{Model grid-points around latitude circle.}}
	\label{adv-grid}
\end{figure}

In many advection problems we assume that the domain is {\em
	periodic}. For example, we might consider advection by a jet blowing
around a latitude circle. If the jet does not deviate to the north or
south, then a tracer leaving Reading will return to Reading after
circulating the globe. In Figure~\ref{adv-grid}, a tracer value crossing $x=L$
will re-appear at $x=0$. The discretised {\em periodic boundary
	conditions} can be written:

\begin{equation}
	T(x_1-\Delta x)=T(x_J)~~~ ; ~~~T(x_J+\Delta x)=T(x_1)
\end{equation}

where there are $J$ grid-points across the domain with regular spacing
$\Delta x=L/J$. The two temperature values above, at points outside the
domain $0<x<L$, can be used to find the temperature gradient at the
boundaries.

\paragraph{Kinematic BCs}

The fluid moves with the boundary in the direction normal to the boundary.
\BEQ
{\bf u.n}={\bf u}_b.{\bf n}
\EEQ

\paragraph{Dynamic BCs}

There is a balance of internal fluid forces normal to the boundary. If
the boundary is not rigid and there is fluid on both sides this
amounts to pressures being equal on both sides of the boundary: 
\BEQ
p_1=p_2 
\EEQ

\paragraph{Flow parallel to boundaries}

The {\bf no slip} condition is that the fluid moves with the boundary
\BEQ
{\bf u}={\bf u}_b
\EEQ
in contrast to {\bf free slip} where the fluid is free to move along
the boundary as if no friction were acting. This is unrealistic at
finescales but is used when the fluid away from boundaries
is being modelled without dealing with the {\em boundary layer}. 

\vspace{1em} 

%%%%%%%%%%%%%%%%%%%%%%%%%%%%%%%%%%%%%%%%%%%%%%%%%%
\newpage

%%%%%%%%%%%%%%%%%%%%%%%%%%%%%%%%%%%%%%%%%%%%%%%%%%
\newpage
\chapter{Complex numbers}

\section{Euler relations}
\href{https://github.com/pierluigividale/MTMW14/blob/main/Notebooks/Euler%20equations.ipynb}{Link to Euler formulae Notebook}

\section{Representation of wave functions in the complex plane}

\section{What are the benefits of using such wave functions}
Link to iPython Notebook

\vspace{1em} 


\chapter{Vorticity equation derivation}

\section{What is PV?}

\subsection{Derivation of an important conservative quantity: PV}

We saw in Lecture 4 how we can derive the Shallow Water Equations (SWEs). There are three independent variables: $u,v,p$ in that system:
\begin{eqnarray}
	\partd{u}{x}+\partd{v}{y}+\partd{w}{z} & = & 0 \\
	\lagd{u}{t}-fv & = & -\frac{1}{\rho_o}\partd{p}{x} \\
	\lagd{v}{t}+fu & = & -\frac{1}{\rho_o}\partd{p}{y}
\end{eqnarray}

Using the vertically integrated hydrostatic balance: $-\frac{1}{\rho_o} \partd{p}{x} = -g \partd{h}{x}$, we can write: 
\begin{eqnarray}
	\lagd{u}{t} & = & -g\partd{h}{x} +fv\\
	\lagd{v}{t} & = & -g\partd{h}{y} -fu
\end{eqnarray}

It is possible to combine the three governing equations into a single one by cross-differentiation of the two momentum equations and substitution of mass continuity:
\begin{eqnarray}
	&&\frac{\partial}{\partial y} \left( \lagd{u}{t} = -g\partd{h}{x} +fv \right) \\
	&&\frac{\partial}{\partial x} \left( \lagd{v}{t} = -g\partd{h}{y} -fu \right) 
\end{eqnarray}

Subtract the second from the first equation; use the definition of relative vorticity: $\zeta = \frac{\partial v}{\partial x} - \frac{\partial u}{\partial y}$:

\begin{eqnarray}
	&&\frac{D_h}{Dt}\left({\zeta+f}\right) = -(\zeta+f) \left( \frac{\partial u}{\partial x} + \frac{\partial v}{\partial y} \right);\\
	&&\lagdh{h}{t} = -h \left( \frac{\partial u}{\partial x} + \frac{\partial v}{\partial y} \right)
\end{eqnarray}
to yield:
\begin{eqnarray}
	\frac{D_h}{Dt}\left[\frac {\zeta+f}{h} \right]=0
	\label{PVcons}
\end{eqnarray} 

\subsection{Dispersion relation for Rossby waves on $\beta$-plane}

To derive the dispersion relation for Rossby waves we revert to the use of basic state and perturbations: $u=\bar{u}+u'; v=v'; \zeta = \bar{\zeta}+\zeta'$ on a $\beta$-plane: $f=f_0+\beta y$ in the barotropic vorticity equation: $\frac{D_h}{Dt}\left({\zeta+f} \right)=0$
and making use of a streamfunction $\psi$ to define: $u'=-\partd{\psi'}{y}; v'=-\partd{\psi'}{x}$ to yield:
\begin{eqnarray}
	\zeta'&=&\nabla^2\psi'\\
	\left(\partd{}{t}+\bar{u}\partd{}{x}\right)\nabla^2\psi'+\beta \partd{\psi'}{x}&=&0
\end{eqnarray}

and inserting a solution of this type: $\psi'=Re[\Psi exp(i\phi)]$, where $\phi=kx+ly-\omega t$ results in:

\begin{eqnarray}
	(-\omega + k\bar{u})(-k^2-l^2)+k \beta & = & 0\\
	c_x-\bar{u} & = & -\frac{\beta}{K^2}
\end{eqnarray}

where: $\omega=\bar{u}k-\frac{\beta k}{K^2}$ and $K^2=k^2+l^2$.\\


\vspace{1em} 

%%%%%%%%%%%%%%%%%%%%%%%%%%%%%%%%%%%%%%%%%%%%%%%%%%
\newpage
\chapter{Eigenvalues and eigenvectors}

\section{How can we perform stability analysis in 2D?}

\section{What are eigenvalues and eigenvectors, and how can they help?}
Link to iPython Notebook

\vspace{1em} 

%%%%%%%%%%%%%%%%%%%%%%%%%%%%%%%%%%%%%%%%%%%%%%%%%%
\newpage
\chapter{The Lorenz Model}
%%%%%%%%%%%%%%%%%%%%%%%%%%%%%%%%%%%%%%%%%%%%%%%%%%%%

\section{What is chaos?}

\section{Resource: the Lorenz model}
\href{https://www.dropbox.com/scl/fi/rw3p2aig729zogynv6r0l/run_Lorenz_example.ipynb?rlkey=dw8byerasoreiumat80r0frea&dl=0}{Link to iPython Notebook on Lorenz's 1963 model}

\vspace{1em} 

% ----------------------------------------------------------------------------------------
% 	BIBLIOGRAPHY
% ----------------------------------------------------------------------------------------

\begin{thebibliography}{9}
	
	\printbibliography[%
	heading=bibempty
	]
	
\end{thebibliography}



\end{document}