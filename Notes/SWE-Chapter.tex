%%%%%%%%%%%%%%%%%%%%%%%%%%%%%%%%%%%%%%%%%%%%%%%%%%
\newpage
\chapterimage{2613-1477-max.jpg} % Chapter heading image
\chapter{The Shallow Water Equations}
%%%%%%%%%%%%%%%%%%%%%%%%%%%%%%%%%%%%%%%%%%%%%%%%%%%%

\vspace{1em}

\section{Shallow atmosphere approximation} 

Also known as the {\em traditional approximation} since it also
applies to the oceans. The deepest motions in the atmosphere have
depth scales, $H\approx\,$20km. The oceans are even shallower. Therefore
$H/a< 1/300$. The shallow limit $H/a\ll 1$ gives:

\begin{itemize}
	\item
	radial coordinate $r\rightarrow a$ (i.e., constant)
	
	\item
	$\partd{}{r}\rightarrow \partd{}{z}$ where $z$ is height relative to Earth's surface
	
	\item
	In order to retain same form for kinetic energy and zonal angular
	momentum equations, must drop terms containing $w$ in horizontal
	momentum equations and the horizontal component of Coriolis
	acceleration.
\end{itemize}

The resulting {\em primitive equations} (PEs) are the basis of almost
all atmosphere and ocean {\em general circulation models} (GCMs)
except the Met Office Unified Model which does not make a shallow
atmosphere approximation.

\begin{itemize}
	\item
	Biggest errors are in Tropics associated with neglect of horizontal
	component of Coriolis acceleration.\footnote{\BTi White, Hoskins, Roulstone and Staniforth (2005) Consistent approximate models of the global atmosphere: shallow, deep, hydrostatic, quasi-hydrostatic and non-hydrostatic. \emph{Quart. J. Roy. Met. Soc.}, \textbf{131},
		2081-2107.
		\ETi }
	
	\item
	Effects of spherical geopotential approximation are unknown.\footnote{\BTi White, Staniforth and Wood (2008) Spheroidal coordinate systems for modelling global atmospheres. \emph{Quart. J. Roy. Met. Soc.}, \textbf{134},
		261-270.
		\ETi }
\end{itemize}


\subsection{Filtering the equations to remove unwanted scales of motion} 

The problem with our Euler equations is that they support all types of motion, some of which may be unwanted, depending on what scientific questions we are trying to answer. 

\medskip

{\bf Example: sound waves.}\\
The governing equations of compressible fluid dynamics contain acoustic waves. In low Mach number flows, typical of atmospheric or oceanic conditions, acoustic waves play no essential role and yet they severely constrain the time step in numerical modeling. Thus, there has long been an interest in approximating the governing equations to eliminate or "filter out" acoustic waves. 

\medskip

\begin{tabular}{|c|c|c|c|c|}
	\hline
	Wave & horizontal & propagation & propagation  & time step \\
	type & dimension & speed & direction & required \\
	\hline
	Sound &  &  &  & \\
	\hline
	Short Gravity &  &  &  & \\
	\hline
	Long Gravity &  &  &  & \\
	\hline
	Rossby &  &  & & \\
	\hline
	Kelvin &  &  &  & \\
	\hline
\end{tabular}


\subsection{Further approximations} 

Further approximations are usually based on {\em scaling} where
the typical spatial and length scales associated with motions of
interest are assumed. For example:

~

{\bf Planar approximation ($L/a \ll 1$)} Can use local Cartesian coordinates where $(dx,dy)=(a \cos\phi_0
\,d\lambda, a\,d\phi)$. 

Can drop spherical metric terms from PEs.

~

{\bf Anelastic approximation ($\Delta \rho /\rho_0 \ll 1$)} Mass conservation equation becomes $\nabla.(\rho_r {\mathbf u})=0$
where $\rho_r(z)$ is a reference density.
Filters sound waves.

~

{\bf Boussinesq approximation ($\Delta \rho /\rho_0 \ll 1$ and $H\ll H_{\rho}$)} Huge density height scale, $H_{\rho}$, implies $w \partd{\rho_r}{z}\ll
\rho_r \partd{w}{z}$. In practice, the density variation is only important in the buoyancy term of the (vertical) momentum equation, $\rho g$. The continuity equation:
\begin{equation}
	\frac{1}{\rho} \lagd{\rho}{t} + \nabla \cdot \mathbf{u} = 0
\end{equation}

ends up simplified to its incompressible ($\rho_r \approx$ constant) form: $\nabla \cdot \mathbf{u} = 0$. 

~

{\bf Hydrostatic balance ($H/L \ll 1$)} Vertical momentum equation reduces to:
\begin{equation}
	\frac{1}{\rho}\partd{p}{z}=-g
	\label{hydrostatic}
\end{equation}
Filters vertically-propagating sound waves, but can support Lamb waves, and can distort gravity waves if $L_x \approx L_y$.

If we also want to filter out gravity waves, we must additionally set the local time rate of change of the divergence field to zero.


\section{Simplifying the PEs}
\subsection{PEs on the sphere}
In an Eulerian framework, our equation (1) ends up looking like this:
\begin{eqnarray}
	\frac {\partial u}{\partial t} + \left(\mathbf{v}\cdot\nabla\right)u=-\frac{1}{\rho} \frac{\partial p}{\partial x}+F_x \\
	\frac {\partial v}{\partial t} + \left(\mathbf{v}\cdot\nabla\right)v=-\frac{1}{\rho} \frac{\partial p}{\partial y}+F_y \\
	\frac {\partial w}{\partial t} + \left(\mathbf{v}\cdot\nabla\right)w=-\frac{1}{\rho} \frac{\partial p}{\partial z}-g+F_z
\end{eqnarray}

We must now consider the fact that we are studying fluids (atmosphere, ocean) on our planet, which is approximately a sphere of radius $a$, rotating with angular velocity $\Omega$,

\begin{minipage}{0.4\textwidth}
	%	\begin{figure}
		%\includegraphics[trim={0cm 2cm 0cm 2cm},clip,width=3.cm]{/Users/vidale/Projects/Teaching/MTMW14/"2016 module"/RotatingEarthDiagram.pdf}
		%\label{fig:Spherical}
		%%\setbeamerfont{caption name}{size=\tiny}
		%\caption{\label{fig:blue_rectangle} \tiny Schematic of spherical coordinate system}
		%	\end{figure}
\end{minipage} \hfill

\begin{minipage}{0.55\textwidth}
	In the spherical coordinate system we use $\lambda$ (longitude), $\theta$ (latitude) and r, to replace $x,y,z$, where $x=r \cos{\theta} \lambda$; $y=r\theta$; $z=r-a$, so that:
	\begin{eqnarray}
		u=\frac {dx}{dt}=r \cos{\theta} \frac {d \lambda}{d t} ; v=\frac {dy}{dt}= r\frac {d\theta}{dt}= ; w=\frac {dz }{dt }
	\end{eqnarray}
	are how we write our spatial derivatives on the Sphere.	
\end{minipage}

\subsection{PEs on the Sphere before scale analysis}
\includegraphics[trim={2cm 10.5cm 0cm 2.cm},clip,width=10.cm]{/Users/vidale/Projects/Teaching/MTMW14/"2016 module"/PEs_Sphere_1.pdf}
\label{fig:Spherical-PEs1}
%	%\setbeamerfont{caption name}{size=\tiny}
%\caption{\label{fig:blue_rectangle} \tiny PEs on the Sphere}

\includegraphics[trim={2cm 5cm 0cm 4cm},clip,width=10.cm]{/Users/vidale/Projects/Teaching/MTMW14/"2016 module"/PEs_Sphere_2.pdf}
\label{fig:Spherical-PEs2}
%	%\setbeamerfont{caption name}{size=\tiny}
%\caption{\label{fig:blue_rectangle} \tiny PEs on the Sphere}


\subsection{Small quantities, big quantities: Scale Analysis}

Cancelling out the smallest terms, we end up with simpler equations and some important approximations, for instance hydrostatic in the vertical momentum equation.

\begin{figure}
	\includegraphics[trim={0cm 0cm 0cm 1cm},clip,width=11.cm]{/Users/vidale/Projects/Teaching/MTMW14/"2016 module"/ScaleAnalysisTables.pdf}
	\label{fig:Spherical}
	%	%\setbeamerfont{caption name}{size=\tiny}
	\caption{\label{fig:blue_rectangle} \tiny Scale Analysis Tables}
\end{figure}

\subsection{Simplified PEs after scale analysis}

\includegraphics[trim={2cm 2.5cm 0cm 2.cm},clip,width=12.cm]{/Users/vidale/Projects/Teaching/MTMW14/"2016 module"/PEs_Sphere_3.pdf}
\label{fig:Spherical-PEs1}
%	%\setbeamerfont{caption name}{size=\tiny}
%\caption{\label{fig:blue_rectangle} \tiny PEs on the Sphere}


\subsection{Simplified PEs after scale analysis, now back on a PLANE}

$f$-plane: Coriolis term is a constant $f_0$ over the entire domain\\
$\beta$-plane: Coriolis term changes linearly in the meridional direction\\

\begin{eqnarray}
	\partd{u}{t} + \left( \mathbf{v} \cdot \nabla\right) u - f_0 v & = & -\frac{1}{\rho_o}\partd{p}{x} +F_x \\
	\partd{v}{t} + \left( \mathbf{v} \cdot \nabla\right) v + f_0 u & = & -\frac{1}{\rho_o}\partd{p}{y} +F_y \\
	0 & = & - \frac{1}{\rho} \partd{p}{z} - g + F_z
	%\partd{u}{x}+\partd{v}{y}+\partd{w}{z} & = & 0
\end{eqnarray}

and the continuity equation also becomes much simpler, as metric terms drop out. The simpler geometry of the plane may be retained while taking the latitudinal varations of $f$ into account, by defining a $\beta$-plane, for which $f$ is calculated by linearising the Coriolis terms around a constant $f_0=2\Omega \sin\phi_0$ -
the component of the Earth's rotation normal to the Earth's surface at
reference latitude $\phi_0$. 	

%\includegraphics[trim={2cm 0.5cm 0.cm 10.cm},clip,width=12.cm]{/Users/vidale/Projects/Teaching/MTMW14/"2016 module"/PEs_Sphere_4.pdf}
%\label{fig:Spherical-PEs1}
%%\setbeamerfont{caption name}{size=\tiny}
%\caption{\label{fig:blue_rectangle} \tiny PEs on the Sphere}


\section{Shallow water equations}

Making the planar, Boussinesq and hydrostatic approximations, and neglecting those unresolved forces, 
the PEs reduce to:
\begin{eqnarray}
	\lagd{u}{t}-f_0 v & = & -\frac{1}{\rho_o}\partd{p}{x} \\
	\lagd{v}{t}+f_0 u & = & -\frac{1}{\rho_o}\partd{p}{y} \\
	0 & = & - \frac{1}{\rho} \partd{p}{z} - g \\
	\partd{u}{x}+\partd{v}{y}+\partd{w}{z} & = & 0
\end{eqnarray}

plus thermodynamic eqn and eqn of state. Remember that $f_0=2\Omega \sin\phi_0$ is the Coriolis parameter normal to the Earth's surface at
reference latitude $\phi_0$.

~

Pressure and fluid depth are next divided into
a reference state and perturbation:
\begin{equation*}
	p=p_r(z)+p'(x,y,z,t)~~~~~;~~~~~h=H+\eta(x,y,z,t)
\end{equation*}

The shallow water equations are obtained by integrating the PEs
over a fluid layer of depth $h$. Integrating hydrostatic balance
downwards from the top to any level $z$:
\begin{eqnarray*}
	\frac{1}{\rho}\partd{p}{z}=-g \Rightarrow p_{top}-p(x,y,z,t) & = & -\rho_o g (h-z) \\
	\Rightarrow -\frac{1}{\rho_o} \partd{p}{x} & = & -g \partd{h}{x}
\end{eqnarray*}

assuming that there are no pressure perturbations on the top.
The pressure gradient terms become $-g\partd{\eta}{x}$ and
$-g\partd{\eta}{y}$, since the reference depth $H$ must be constant.

The mass conservation equation integrated vertically is:
\begin{eqnarray*}
	\partd{u}{x}+\partd{v}{y}+\partd{w}{z} = 0 \Rightarrow \left\{ \partd{u}{x}+\partd{v}{y} \right\} h+[w]^{top}_{bot} & = & 0 \\
	\left\{ \partd{u}{x}+\partd{v}{y} \right\} h+\lagd{h}{t} & = & 0 \\
	\partd{h}{t}+\partd{(hu)}{x}+\partd{(hv)}{y} & = & 0
\end{eqnarray*}
where $(u,v)$ is now the depth-average velocity and $w_{bot}=0$ was assumed. 


\subsection{SWEs linearised around a basic state} 

{\bf Shallow-water equations} on the
plane tangent to the Earth's surface, {\em linearised} about a resting basic state ($u=u_r(z)+u'(x,y,z,t); v=v_r(z)+v'(x,y,z,t);h=H+\eta(x,y,z,t)$):
\begin{eqnarray}
	&&\frac{\partial \eta}{\partial t} + H \left(
	\frac{\partial u}{\partial x}+\frac{\partial v}{\partial y}\right) = 0,
	\\
	&&\frac{\partial u}{\partial t} - f_0 v = - g\frac{\partial \eta}{\partial x}, \\
	&&\frac{\partial v}{\partial t} + f_0 u = - g\frac{\partial \eta}{\partial y}, 
\end{eqnarray}
where $\eta$ is the surface elevation, $H$ is fluid depth, $(u,v)$ is
the depth-averaged horizontal velocity, $f$ is the Coriolis parameter
and $g$ is the gravitational acceleration. 

~~

Now discretise the problem using regular grids to describe $u$, $v$ and $\eta$ 

e.g., $\eta_{ij}=$ free surface elevation at the $i$th longitude and $j$th
latitude point.

\subsection { Horizontal gradients in SWEs: how to discretize in space?} 

SWEs seem nice and simple, but now we have spatial gradients: $\frac{\partial u}{\partial x}$, $\frac{\partial v}{\partial y}$, $\frac{\partial \eta}{\partial x}$, $\frac{\partial \eta}{\partial y}$.\\

~

How are we going to discretise the problem, using regular grids, to obtain a good simulation of $u$, $v$ and $\eta$? Something like this?

\begin{tabular}{lc}
	\begin{minipage}[c]{0.7\textwidth}
		\begin{eqnarray*}
			\frac{\partial u}{\partial x}\Big|_{ij} &\approx& \left(\frac{u_{i+1,j}+u_{i+1,j-1}}{2} - \frac{u_{i-1,j}+u_{i-1,j-1}}{2} \right)\frac{1}{2 \Delta x} \\
			\frac{\partial v}{\partial y}\Big|_{ij} &\approx& \left(\frac{v_{i,j+1}+v_{i-1,j+1}}{2} - \frac{v_{i,j-1}+v_{i-1,j-1}}{2} \right)\frac{1}{2 \Delta y} \\
			&& \\
			\frac{\partial \eta}{\partial x}\Big|_{ij} &\approx& \left(\frac{\eta_{i+1,j}+\eta_{i+1,j+1}}{2} - \frac{\eta_{i-1,j}+\eta_{i-1,j+1}}{2} \right)\frac{1}{2 \Delta x} \\
			f v\big|_{ij} &\approx& f_j \frac{v_{i,j}+v_{i,j+1}+v_{i-1,j}+v_{i-1,j+1}}{4} \\
			&& \\
			\frac{\partial \eta}{\partial y}\Big|_{ij} &\approx& \left(\frac{\eta_{i,j+1}+\eta_{i+1,j+1}}{2} - \frac{\eta_{i,j-1}+\eta_{i+1,j-1}}{2} \right)\frac{1}{2 \Delta y} \\
			f u\big|_{ij} &\approx& f_j \frac{u_{i,j}+u_{i+1,j}+u_{i,j-1}+u_{i+1,j-1}}{4}
		\end{eqnarray*}
	\end{minipage}
\end{tabular}

~

{\bf But why? It all seems so arbitrary.} There are a few criteria: we want to avoid computational modes and we want waves to propagate/evolve in a realistic way in space and time.

\section{Arakawa's 2D grids}
\subsection{Horizontally staggered grids} 

Need to choose finite difference
methods to solve the {\bf shallow-water equations} on the
plane tangent to the Earth's surface, {\em linearised} about a resting basic state:
\begin{eqnarray}
	&&\frac{\partial \eta}{\partial t} + H \left(
	\frac{\partial u}{\partial x}+\frac{\partial v}{\partial y}\right) = 0,
	\\
	&&\frac{\partial u}{\partial t} - f_0 v = - g\frac{\partial \eta}{\partial x}, \\
	&&\frac{\partial v}{\partial t} + f_0 u = - g\frac{\partial \eta}{\partial y}, 
\end{eqnarray}
where $\eta$ is the surface elevation, $H$ is fluid depth, $(u,v)$ is
the depth-averaged horizontal velocity, $f$ is the Coriolis parameter
and $g$ is the gravitational acceleration. 

~~

Now discretise the problem using regular grids to describe $u$, $v$ and $\eta$ 

e.g., $\eta_{ij}=$ free surface elevation at the $i$th longitude and $j$th
latitude point.

~

Arakawa\footnote{\BTi Arakawa
	A. (1966) Computational design for long-term numerical integration of
	the equations of fluid motion: Two-dimensional incompressible
	flow. Part I. \emph{Journal of Computational Physics}, \textbf{1},
	119-143.\ETi} proposed a number of staggered finite difference
grids that can be used to solve the shallow water equations.  


\subsection{Arakawa's A grid}
Simplest and intuitive, but there is the danger of computational modes.
\begin{tabular}{lc}
	\begin{minipage}[c]{0.6\textwidth}
		\begin{eqnarray*}
			\eta-grid \;\;\; \;\;\; \;\;\;
			\frac{\partial u}{\partial x}\Big|_{ij} &\approx& \frac{u_{i+1,j}-u_{i-1,j}}{2 \Delta x} \\
			\frac{\partial v}{\partial y}\Big|_{ij} &\approx& \frac{v_{i,j+1}-v_{i,j-1}}{2 \Delta y} \\
			&& \\
			u-grid \;\;\; \;\;\; \;\;\;
			\frac{\partial \eta}{\partial x}\Big|_{ij} &\approx& \frac{\eta_{i+1,j}-\eta_{i-1,j}}{2 \Delta x} \\
			f v\big|_{ij} &\approx& f v_{i,j} \\
			&& \\
			v-grid \;\;\; \;\;\; \;\;\;
			\frac{\partial \eta}{\partial y}\Big|_{ij} &\approx& \frac{\eta_{i,j+1}-\eta_{i,j-1}}{2 \Delta y} \\
			f u\big|_{ij} &\approx& f u_{i,j}
		\end{eqnarray*}
	\end{minipage}
	&
	\begin{minipage}[c]{0.4\textwidth}
		\setlength{\unitlength}{1 cm}
		\begin{picture}(4,4)
			\arakawa
			\put(0.93,0.93){$\bullet$} \put(1,1){\vector(1,0){0.5}} \put(1,1){\vector(0,1){0.5}}
			\put(2.93,0.93){$\bullet$} \put(3,1){\vector(1,0){0.5}} \put(3,1){\vector(0,1){0.5}}
			\put(0.93,2.93){$\bullet$} \put(1,3){\vector(1,0){0.5}} \put(1,3){\vector(0,1){0.5}}
			\put(2.93,2.93){$\bullet$} \put(3,3){\vector(1,0){0.5}} \put(3,3){\vector(0,1){0.5}}
			\put(1.1,0.7){$u_{ij}$,$v_{ij}$,}
			\put(1.1,0.4){$\eta_{ij}$}
			\put(3.1,3.6){$u_{i+1,j+1}$}
			\put(3.1,3.4){$v_{i+1,j+1}$}
			\put(3.1,3.2){$\eta_{i+1,j+1}$}
			\put(3.1,0.7){$u_{i+1,j}$}
			\put(3.1,0.5){$v_{i+1,j}$}
			\put(3.1,0.3){$\eta_{i+1,j}$}
		\end{picture}
	\end{minipage}
\end{tabular}

\subsection{Arakawa's B grid}
Stagger the variables, eliminate computational mode, but lots of interpolation...\\

\begin{tabular}{lc}
	\begin{minipage}[c]{0.6\textwidth}
		\begin{eqnarray*}
			\frac{\partial u}{\partial x}\Big|_{ij} &\approx& \left(\frac{u_{i+1,j}+u_{i+1,j+1}}{2} - \frac{u_{i,j}+u_{i,j+1}}{2} \right)\frac{1}{\Delta x} \\
			\frac{\partial v}{\partial y}\Big|_{ij} &\approx& \left(\frac{v_{i,j+1}+v_{i+1,j+1}}{2} - \frac{v_{i,j}+v_{i+1,j}}{2} \right)\frac{1}{\Delta y} \\
			&& \\
			\frac{\partial \eta}{\partial x}\Big|_{ij} &\approx& \left(\frac{\eta_{i,j-1}+\eta_{i,j}}{2} - \frac{\eta_{i-1,j-1}+\eta_{i-1,j}}{2} \right)\frac{1}{\Delta x} \\
			f v\big|_{ij} &\approx& f v_{i,j} \\
			&& \\
			\frac{\partial \eta}{\partial y}\Big|_{ij} &\approx& \left(\frac{\eta_{i-1,j}+\eta_{i,j}}{2} - \frac{\eta_{i-1,j-1}+\eta_{i,j-1}}{2} \right)\frac{1}{\Delta y} \\
			f u\big|_{ij} &\approx& f u_{i,j}
		\end{eqnarray*}
	\end{minipage}
	&
	\begin{minipage}[c]{0.4\textwidth}
		\setlength{\unitlength}{1 cm}
		\begin{picture}(4,4)
			\arakawa
			\put(1.93,1.93){$\bullet$}
			\put(2.1,1.7){$\eta_{ij}$}
			\put(1,1){\vector(1,0){0.5}} \put(1,1){\vector(0,1){0.5}}
			\put(3,1){\vector(1,0){0.5}} \put(3,1){\vector(0,1){0.5}}
			\put(1,3){\vector(1,0){0.5}} \put(1,3){\vector(0,1){0.5}}
			\put(3,3){\vector(1,0){0.5}} \put(3,3){\vector(0,1){0.5}}
			\put(1.1,0.7){$u_{ij}$,$v_{ij}$}
			\put(3.1,3.4){$u_{i+1,j+1}$}
			\put(3.1,3.2){$v_{i+1,j+1}$}
			\put(3.1,0.7){$u_{i+1,j}$}
			\put(3.1,0.5){$v_{i+1,j}$}
		\end{picture}
	\end{minipage}
\end{tabular}

\subsection{Arakawa's C grid}
Stagger the variables, eliminate computational mode, interpolation only required for Coriolis terms!\\

\begin{tabular}{lc}
	\begin{minipage}[c]{0.6\textwidth}
		\begin{eqnarray*}
			\eta-grid \;\;\; \;\;\; 
			\frac{\partial u}{\partial x}\Big|_{ij} &\approx& \frac{u_{i+1,j}-u_{i,j}}{\Delta x} \\
			\frac{\partial v}{\partial y}\Big|_{ij} &\approx& \frac{v_{i,j+1}-v_{i,j}}{\Delta y} \\
			&& \\
			u-grid \;\;\; \;\;\; 
			\frac{\partial \eta}{\partial x}\Big|_{ij} &\approx& \frac{\eta_{i,j}-\eta_{i-1,j}}{\Delta x} \\
			f v\big|_{ij} &\approx& f_j \frac{v_{i,j}+v_{i,j+1}+v_{i-1,j}+v_{i-1,j+1}}{4} \\
			&& \\
			v-grid \;\;\; \;\;\; 
			\frac{\partial \eta}{\partial y}\Big|_{ij} &\approx& \frac{\eta_{i,j}-\eta_{i,j-1}}{\Delta y} \\
			f u\big|_{ij} &\approx& f_j \frac{u_{i,j}+u_{i+1,j}+u_{i,j-1}+u_{i+1,j-1}}{4}
		\end{eqnarray*}
	\end{minipage}
	&
	\begin{minipage}[c]{0.4\textwidth}
		\setlength{\unitlength}{1 cm}
		\begin{picture}(4,4)
			\arakawa
			\put(1.93,1.93){$\bullet$}
			\put(2.1,1.7){$\eta_{ij}$}
			\put(2,0.75){\vector(0,1){0.5}}
			\put(2,2.75){\vector(0,1){0.5}}
			\put(0.75,2){\vector(1,0){0.5}}
			\put(2.75,2){\vector(1,0){0.5}}
			\put(1.1,1.7){$u_{ij}$}
			\put(2.1,0.7){$v_{ij}$}
			\put(3.1,1.7){$u_{i+1,~j}$}
			\put(2.1,2.7){$v_{i,~j+1}$}
		\end{picture}
	\end{minipage}
\end{tabular}


\subsection{Arakawa's D grid}
Idea similar to B grid, but we push out $u,v$ instead of $\eta$: stagger the variables, eliminate computational mode, but lots of interpolation...\\

\begin{tabular}{lc}
	\begin{minipage}[c]{0.7\textwidth}
		\begin{eqnarray*}
			\frac{\partial u}{\partial x}\Big|_{ij} &\approx& \left(\frac{u_{i+1,j}+u_{i+1,j-1}}{2} - \frac{u_{i-1,j}+u_{i-1,j-1}}{2} \right)\frac{1}{2 \Delta x} \\
			\frac{\partial v}{\partial y}\Big|_{ij} &\approx& \left(\frac{v_{i,j+1}+v_{i-1,j+1}}{2} - \frac{v_{i,j-1}+v_{i-1,j-1}}{2} \right)\frac{1}{2 \Delta y} \\
			&& \\
			\frac{\partial \eta}{\partial x}\Big|_{ij} &\approx& \left(\frac{\eta_{i+1,j}+\eta_{i+1,j+1}}{2} - \frac{\eta_{i-1,j}+\eta_{i-1,j+1}}{2} \right)\frac{1}{2 \Delta x} \\
			f v\big|_{ij} &\approx& f_j \frac{v_{i,j}+v_{i,j+1}+v_{i-1,j}+v_{i-1,j+1}}{4} \\
			&& \\
			\frac{\partial \eta}{\partial y}\Big|_{ij} &\approx& \left(\frac{\eta_{i,j+1}+\eta_{i+1,j+1}}{2} - \frac{\eta_{i,j-1}+\eta_{i+1,j-1}}{2} \right)\frac{1}{2 \Delta y} \\
			f u\big|_{ij} &\approx& f_j \frac{u_{i,j}+u_{i+1,j}+u_{i,j-1}+u_{i+1,j-1}}{4}
		\end{eqnarray*}
	\end{minipage}
	&
	\begin{minipage}[c]{0.3\textwidth}
		\setlength{\unitlength}{1 cm}
		\begin{picture}(4,4)
			\arakawa
			\put(0.93,0.93){$\bullet$}
			\put(2.93,0.93){$\bullet$}
			\put(0.93,2.93){$\bullet$}
			\put(2.93,2.93){$\bullet$}
			\put(2,0.75){\vector(0,1){0.5}}
			\put(2,2.75){\vector(0,1){0.5}}
			\put(0.75,2){\vector(1,0){0.5}}
			\put(2.75,2){\vector(1,0){0.5}}
			\put(1.1,1.7){$u_{ij}$}
			\put(2.1,0.7){$v_{ij}$}
			\put(1.1,0.7){$\eta_{ij}$}
		\end{picture}
	\end{minipage}
\end{tabular}


\subsection{Arakawa's Sm\"org{\aa}sbord}

\begin{figure}[h]
	\centering
	\includegraphics[width=0.8\textwidth]{/Users/vidale/Documents/Presentations/"SWE 2D grid stagger types".pdf}
	\caption{Many of Arakawa's grid staggers}
	\label{fig:Arakawa_all}
\end{figure}


\section{Excursus: SWEs in 1D}

\begin{figure}
\includegraphics[trim={0cm 0.5cm 0cm 2.cm},clip,width=11.cm]{/Users/vidale/Projects/Teaching/MTMW14/"2016 module"/Adv_2_SWEs_1.pdf}
\label{fig:Spherical-PEs1}
%	%\setbeamerfont{caption name}{size=\tiny}
\caption{\label{fig:blue_rectangle} \tiny PEs on the Sphere}
\end{figure}

\subsection{Resolving waves on a grid} 
\begin{figure}
\includegraphics[trim={0cm 0.cm 0cm 0.cm},clip,width=9.cm]{/Users/vidale/Projects/Teaching/MTMW14/"2019 module/Waves resolved on a Grid".png}
\label{fig:Spherical-PEs2}
%	%\setbeamerfont{caption name}{size=\tiny}
\caption{\label{fig:blue_rectangle} \tiny PEs on the Sphere}
\end{figure}

\subsection{How 1D SWEs help to understand grid stagger} 
\begin{figure}
\includegraphics[trim={0cm 0.5cm 0cm 2.cm},clip,width=11.cm]{/Users/vidale/Projects/Teaching/MTMW14/"2016 module"/Adv_2_SWEs_2.pdf}
\label{fig:Spherical-PEs2}
%	%\setbeamerfont{caption name}{size=\tiny}
\caption{\label{fig:blue_rectangle} \tiny PEs on the Sphere}
\end{figure}

\begin{figure}
\includegraphics[trim={0cm 0.5cm 0cm 2.cm},clip,width=11.cm]{/Users/vidale/Projects/Teaching/MTMW14/"2016 module"/Adv_2_SWEs_3.pdf}
\label{fig:Spherical-PEs2}
%	%\setbeamerfont{caption name}{size=\tiny}
\caption{\label{fig:blue_rectangle} \tiny PEs on the Sphere}
\end{figure}

\begin{figure}[h]
\includegraphics[trim={0cm 0.5cm 0cm 1.cm},clip,width=11.cm]{/Users/vidale/Projects/Teaching/MTMW14/"2016 module"/Adv_2_SWEs_4.pdf}
\label{fig:Spherical-PEs2}
%	%\setbeamerfont{caption name}{size=\tiny}
\caption{\label{fig:blue_rectangle} \tiny PEs on the Sphere}
\end{figure}

\subsection{Exercise: analytical versus numerical solutions} 

Start from the two shallow water equations on the previous page. Impose the wave solution proposed on the slide, for $u_j$ and $h_j$: 

\begin{equation}
	(u_j,h_j) \sim e^{i(k x_j -\omega t)} = e^{i(k j \Delta x -\omega t)}
\end{equation}

\begin{enumerate}
	\item Substitute into the centred numerical expressions (in time and space) and show that you end up with an expression that contains a $\sin$ function.
	\item How does the numerical solution compare to the analytical solution in terms of a) amplitude and b) phase?
	\item Plot a wavenumber versus frequency plot for the analytical and numerical solution
	\item Expand your thinking to Numerical Weather Prediction models that might be using such schemes: discuss what operational problems we may encounter in the presence of a $\sin$ in such numerical predictions.
\end{enumerate}


%%%%%%%%%%%%%%%%%%%%%%%%%%%%%%%%%%%%%%%%%%%%%%%%%%
\newpage

\section{Boundary Conditions (BCs)}

\subsection{Dirchlet, Neuman}
\subsection{Different types of BC: what they are and what they do}

The \textbf{Dirichlet (or first-type) boundary condition }is a type of boundary condition, named after Peter Gustav Lejeune Dirichlet (1805–1859). When imposed on an ordinary or a partial differential equation, it specifies the values that a solution needs to take on along the boundary of the domain. For example, for an ODE:

\begin{equation}
	y''+y=0
\end{equation}
we could specify $y(a)=\alpha; y(b)=\beta$ on the interval $[a,b]$, where $\alpha$, $\beta$ are given numbers.

\medskip

The \textbf{Neumann (or second-type) boundary condition} is a type of boundary condition, named after Carl Neumann. When imposed on an ordinary or a partial differential equation, the condition specifies the values in which the derivative of a solution is applied within the boundary of the domain. For example, for the same ODE above:

we could specify $y'(a)=\alpha; y'(b)=\beta$ on the interval $[a,b]$, where $\alpha$, $\beta$ are given numbers.


\subsection{Robin, mixed}


The \textbf{Robin boundary condition, or third type boundary condition}, is a type of boundary condition, named after Victor Gustave Robin (1855–1897). When imposed on an ordinary or a partial differential equation, it is a specification of a linear combination of the values of a function and the values of its derivative on the boundary of the domain. For instance:\\

\begin{equation}
	au+b\frac{\partial u}{\partial n}y=g \textrm{ on } \partial \Omega
\end{equation}


\begin{figure}
	\begin{center}
		\includegraphics[width=0.2\textwidth]{Figures/440px-Mixed_boundary_conditions.png}
	\end{center}
	\caption{Mixed boundary conditions}
\end{figure}


In 1-D, on the interval $\Omega=[0,1]$ we could for instance have:

\begin{eqnarray}
	au(0)-Bu'(0)=g(0) \\
	au(1)+Bu'(1)=g(1)
\end{eqnarray}

This contrasts with \textbf{mixed boundary conditions}, which are boundary conditions of different types specified on different subsets of the boundary.

\section{Real world applications: variable staggering in W \& C models}
\subsection{Unified Model: New Dynamics vs EndGame grids} 

The choice of grid stagger makes a difference even in very complex GCMs. Here is an example of simply switching the variables defined at the poles.

	\begin{tabular}{cc}
			\centering
			\includegraphics[width=0.75\textwidth]{Figures/New_Dynamics_vs_EndGame_grids.pdf}
		\end{tabular}

This is, in fact, not at all a simple matter: implementation took years, but the consequences were profound in terms of scalability and numerical stability.

\subsection{Unified Model DynCore evolution: stability and scalability} 
	
	The evolution from New Dynamics (2002) to EndGame (2014) meant greater scalability, so that we can use up to 12'000 cores efficiently, as well as numerical stability, which makes long climate simulations at high-resolution (up to N2560, 5km) feasible.
	
			\begin{tabular}{cc}
					\centering
					\includegraphics[width=0.5\textwidth]{Figures/EndGame_stability.pdf}
					\includegraphics[width=0.5\textwidth]{Figures/EndGame_scalability_2.png}
				\end{tabular}
				
				

\section{Time integration for 2D SWE}
\subsection{Staggered time integration schemes for the SWE}

The advection-diffusion equation in one dimension is a useful prototype with which to explore issues of stability, convergence and approximation error. However, the equations ocean modellers use are frequently multi-dimensional. This leads to special problems and results in some cases. With regards to temporal stability, however, schemes which are unstable in one dimension are also unstable in multi-dimensional problems. Schemes which are conditionally stable in one dimension are also conditionally stable in two dimensions or higher, but with more restrictive conditions on $\Delta t$.

\vspace{0.2cm}For equations that support more than one type of process, {\bf the stability criterion will depend on the fastest propagating processes}. However, the fastest propagating phenomena might be of little physical importance and therefore the stability condition might be too constraining. For instance, the linear shallow water equations allow the existence of inertia-gravity and Rossby waves. The propagation speed of the former is about $\sqrt{gH} \approx 100 $ ms$^{-1}$, which is quite large. For Rossby waves, the propagation is of the order of $\beta R_D^2$, which is much smaller. The time integration scheme will therefore greatly depend on the physical processes of interest. 

\hspace{2cm}


\subsection{Reminder from Lecture 5: a semi-implicit scheme for SWEs}

The shallow water equations linearised about a state of rest are below
discretised using a leapfrog scheme for Coriolis terms but a
trapezoidal scheme (mixed implicit-explicit) for the gravity wave
terms:
\begin{eqnarray*}
	\frac{u^{n+1}-u^{n-1}}{2\Delta t}-fv^n+\frac{g}{2}
	\left( \partd{h}{x}^{n+1} +\partd{h}{x}^{n-1} \right) & = & 0 \\
	\frac{v^{n+1}-v^{n-1}}{2\Delta t}+fu^n+\frac{g}{2}
	\left( \partd{h}{y}^{n+1} +\partd{h}{y}^{n-1} \right) & = & 0 \\
	\frac{h^{n+1}-h^{n-1}}{2\Delta t}+\frac{H}{2}
	\left( \partd{u}{x}^{n+1} +\partd{u}{x}^{n-1}
	+\partd{v}{y}^{n+1} +\partd{v}{y}^{n-1} 
	\right) & = & 0. 
\end{eqnarray*}

Re-arranging with future values on the left:
\begin{eqnarray*}
	u^{n+1}+\Delta t g \partd{h}{x}^{n+1} & = & A \\
	v^{n+1}+\Delta t g \partd{h}{y}^{n+1} & = & B \\
	h^{n+1}+\Delta t H \left( \partd{u}{x}^{n+1}+\partd{v}{y}^{n+1} \right) & = & C
\end{eqnarray*}

Can we mix and match at will? Let us start again with the basic ideas of explicit and implicit.


\subsection{Implicit schemes}

In the shallow water equations, fast gravity waves are due to the divergence and gradient terms, and slow Rossby waves are due to the Coriolis term. By treating them differently, one could try to circumvent the stability condition due to gravity waves. The following time integration schemes allow one to change the degree of implicity of the divergence, Coriolis and gradient terms:;
\begin{eqnarray*}
	\eta^{n+1} + \alpha h \Delta t \bnabla \cdot \bu^{n+1} &=& \eta^n - (1-\alpha) h \Delta t \bnabla \cdot \bu^n,
	\\
	\bu^{n+1} + \beta f \Delta t \bk \times \bu^{n+1} + \gamma g \Delta t \bnabla \eta^{n+1} &=& \bu^n - (1-\beta) f \Delta t \bk \times \bu^n - (1-\gamma) g \Delta t \bnabla \eta^n.
\end{eqnarray*}
The implicity coefficients $\alpha$, $\beta$ and $\gamma \in [0,1]$. One way to have ``some information'' about the accuracy and stability of a time integration scheme is to compute the evolution of the energy:
\[
E^n = \int_{\Omega} \frac{1}{2} \rho \left( g (\eta^n)^2 + h \|\bu^n\|^2 \right) \ud \Omega.
\]
Note that this quantity only gives information about the changes in the amplitude of the solution. It does not indicate how the numerical method is affecting the phase of the solution.


\subsection{``Mostly implicit'' schemes are unconditionally stable}

\begin{center}
	\begin{tabular}{cc}
		$\alpha = \beta = \gamma = 1/2$  & $\alpha = \beta = \gamma = 1$ \\
		\includegraphics[width=0.4\textwidth]{Figures/energy_si.eps}
		&
		\includegraphics[width=0.4\textwidth]{Figures/energy_impl.eps}
	\end{tabular}
\end{center}
However, they can be quite dissipative if they are ``a bit too implicit''. The semi-implicit scheme ($\alpha = \beta = \gamma = 1/2$) is exactly conserving energy while a fully-implicit scheme ($\alpha = \beta = \gamma = 1$) is very dissipative and thus not always accurate.


\subsection{Both the gravity and Coriolis terms must be at least semi-implicit}

\begin{center}
	\begin{tabular}{cc}
		$\alpha = \gamma = 1/2$, $\beta=0$  & $\beta = 1/2$, $\alpha = \gamma = 0$ \\
		\includegraphics[width=0.4\textwidth]{Figures/energy_sifexpl.eps}
		&
		\includegraphics[width=0.4\textwidth]{Figures/energy_expl.eps}
	\end{tabular}
\end{center}

\BTi
When there is no dissipation, the equations are purely hyperbolic ($\mathcal{R}e(\kappa) = 0$) and the time integration scheme can only be stable if the implicity coefficients are all larger or equal to 1/2. In that case, the solution of the system requires to invert a non-diagonal matrix, which can be computationally expensive. The effect of using a ``mostly implicit'' time integration scheme with a large time step is to slow down fast propagating gravity waves. These schemes are thus useful for long simulation for which gravity waves are physically insignificant.
\ETi



\subsection{Explicit schemes}
\subsection{Example of a stable explicit scheme: Adams-Bashforth 3}

\begin{eqnarray*}
	&&\hspace{-0.6cm} \eta^{n+1} = \eta^n - h \Delta t \bnabla \cdot \left(\frac{23}{12} \bu^n - \frac{16}{12} \bu^{n-1} + \frac{5}{12} \bu^{n-2}\right),
	\\
	&&\hspace{-0.6cm} \bu^{n+1} = \bu^n - f \Delta t \bk \times \left(\frac{23}{12} \bu^n - \frac{16}{12} \bu^{n-1} + \frac{5}{12} \bu^{n-2}\right) - g \Delta t \bnabla \left( \frac{23}{12} \eta^n - \frac{16}{12} \eta^{n-1} + \frac{5}{12} \eta^{n-2} \right)
\end{eqnarray*}

%\begin{tabular}{lc}
%	\begin{minipage}{0.6\textwidth}
	
	\begin{figure}
		\includegraphics[trim={0.25cm 0.cm 0.25cm 0.25cm},clip,width=0.6\textwidth]{Figures/energy_AB3.eps}
	\end{figure}
	%	\end{minipage}
\hfill
%	\begin{minipage}{0.35\textwidth}
	
	This scheme is conditionally stable with a stability condition prescribed by gravity waves. A leap-frog scheme would have about the same properties but it would become unstable if some dissipation was added to the scheme. \\
	
	%	\end{minipage}
%\end{tabular}

\medskip

\footnote{More details: Durran D.R. (1991) ``The Third-order Adams-Bashforth method: An attractive alternative to Leap-frog time differencing''. \emph{Monthly Weather Review}, 119, 702-720.}


\subsection{Another example: Forward-Backward in Time scheme}

The Forward-Backward in Time (FBT) scheme has been introduced by Sielecki (1968) and later analysed and improved by Beckers and Deleersnijder (1993)\footnote{Sielecki A. (1968) ``An energy conserving difference scheme for the storm surge equations'', \emph{Monthly Weather Review}, 96, 150-156. Beckers J. and Deleersnijder E. (1993) ``Stability of a FBTCS scheme applied to the propagation of shallow-water inertia-gravity waves on various grids'', \emph{Journal of Computational Physics}, 108, 95-104.}. It tries to mimic a semi-implicit scheme by alternatively changing the order in which the two momentum equations are solved for. The future height anomaly term is used in the two momentum equations as soon as it is available; the Coriolis term is discretized by using the most recently computed velocity component, which requires alternation:

\[
\left\{ \begin{array}{l}
	\eta^{n+1} = \eta^n - h \Delta t \bnabla \cdot \bu^n,
	\\
	u^{n+1} = u^n + f \Delta t v^n - g \Delta t \ds \frac{\partial \eta}{\partial x}^{n+1},
	\\
	v^{n+1} = v^n - f \Delta t u^{n+1} - g \Delta t \ds \frac{\partial \eta}{\partial y}^{n+1},
\end{array} \right.
\]
and
\[
\left\{ \begin{array}{l}
	\eta^{n+2} = \eta^{n+1} - h \Delta t \bnabla \cdot \bu^{n+1},
	\\
	v^{n+2} = v^{n+1} - f \Delta t u^{n+1} - g \Delta t \ds \frac{\partial \eta}{\partial y}^{n+2},
	\\
	u^{n+2} = u^{n+1} + f \Delta t v^{n+2} - g \Delta t \ds \frac{\partial \eta}{\partial x}^{n+2}.
\end{array} \right.
\]

It can be seen that for each set of equations, the Coriolis term is first discretized explicitly and then implicitly. The order of the two momentum equations is switched in the second set of equations. As a result, the time discretization of the Coriolis term appears to be semi-implicit ``on average''. The divergence is always explicit and the gradient is always implicit. It can be shown that the scheme is conditionally stable with about the same stability condition as AB3.

\begin{center}
	\includegraphics[width=0.4\textwidth]{Figures/energy_FBT.eps}
\end{center}

\section{Some real world applications}
{Hurricane and Typhoon simulation in climate GCMs: precipitation composite}

\begin{center}	
	\includegraphics[width=0.8\textwidth]{Figures/TC-precipitation}
\end{center}
