% ----------------------------------------------------------------------------------------
% 	APPENDICES
% ----------------------------------------------------------------------------------------

\appendix
\chapter{Stability analysis in 1D}

\section{Simple advection}


\section{Finite Difference Methods: 1D Examples}
\subsection{The Advection Equation}
\textsl{Partial differential equations, order of approximation, von
	Neumann stability analysis, computational modes, numerical
	dispersion, numerical dissipation.}

In many cases we need to solve equations containing
derivatives with respect to \emph{more than one} variable (e.g.~time,
$t$, and space, $x$).  These are called \emph{partial} differential
equations (PDEs). The solution of partial differential equations using
numerical methods is probably the most important and also the most
complex part of numerical analysis.

A quick point about notation before we begin.  Derivatives in ODEs are
written with the `straight~d',
\BEQ \frac{\mathrm{dT}}{\mathrm{d}t}  \;\;\; , \EEQ
but {\em partial derivatives} in PDEs are written with the `curly~d',
\BEQ \frac{\partial T}{\partial t} \;\;\; , \;\;\; \frac{\partial T}{\partial x} . \EEQ
The first means the derivative of $T$ with respect to time at a fixed
location ($x$) and the second means the derivative of $T$ with respect
to $x$ at a fixed time.

In atmosphere or ocean science we often consider quantities that are
carried with the flow without changing. Examples include the mixing
ratio of an unreactive trace gas or potential temperature. Such
quantities can be used to mark air parcels and follow them and
therefore they are called {\em tracers}. If the wind field ${\bf
	u}({\bf x}, t)$ is known the {\em trajectories} of air parcels can be
found by numerically integrating the ODE:
\begin{equation}
	\frac{D{\bf x}}{Dt}={\bf u}({\bf x}, t)
\end{equation}
where ${\bf x}$ is the position of an air parcel and the capital
$D/Dt$ is used to mean rate of change following an air parcel. 
The tracer value, $\theta_j$, does not change following parcel $j$ so
if we know the values $\theta_j$ for all parcels at time $t_i$ and
their initial positions, we can use their trajectories
to infer the new distribution of tracer at a later time $t_f$. A model
based on following fluid parcels is called a {\em Lagrangian model}.

However, many atmospheric processes occur relative to a fixed
position. Examples include a pollution source from a city and
temperature fluxes from the land or ocean. Sometimes it is better
devise a model which considers atmospheric changes relative to fixed
points. These are called {\em Eulerian
	models}. What happens when air masses carrying tracers pass over a
point?

\begin{figure}
	\begin{center}
		\scalebox{0.5}{\includegraphics{Figures/advection.pdf}}
	\end{center}
	\caption{\textsl{Temperature at a point increases when wind blows from warm side.}}
	\label{fig:ken}
\end{figure}

In Figure~\ref{fig:ken}, the wind is bringing air from the $-x$
direction. If the air `upstream' is warmer than the air `downstream',
the observer will see the temperature increasing. The rate of change of
temperature observed must depend on both the magnitude of this
gradient and the speed at which the air is moving, {\em i.e.,}
\begin{equation}
	\frac{\partial T}{\partial t} =- u \frac{\partial T}{\partial x}.
	\label{advectioncont}
\end{equation}
where the left hand side means the rate of change at a given position
$x$ and the right hand side is called the {\em advection term}. $u$ is
the wind component and $\partial T/\partial x$ means the gradient in
the $x$-direction at a given time. 

\subsubsection{Order of approximation in finite differences}

In the special case of uniform velocity we can find a solution of the
form $T(x,t)=F(x-ut)$, but for general flows we cannot find an analytic
solution. Perhaps the simplest numerical solution is obtained using a
finite difference method where derivatives are replaced by finite
differences. Time and space are divided into finite segments, the time
step $\Delta t$ and a space step $\Delta x$, and the points between
steps are denoted $t_n=n\Delta t$ and $x_i=i\Delta x$.  We will use
the shorthand: 
\BEQ T(x_i,t_n) = T_i^n . \EEQ 
where superscript $n$ in
$T_i^n$ refers to the $n$th timestep.  It does \emph{not} mean $T_i$
to the power of $n$.

The advection equation contains $\partial T / \partial t$ and
$\partial T / \partial x$. There are three obvious ways of
discretizing $\partial T / \partial t$ at the $n$th timestep and $i$th
spatial point:
\BEQ \frac{T_i^{n+1} - T_i^n}{\Delta t} \;\;\; , \;\;\; 
\frac{T_i^n - T_i^{n-1}}{\Delta t} \;\;\; , \;\;\;
\frac{T_i^{n+1} - T_i^{n-1}}{2\Delta t} . \EEQ

These are called the forward difference, backward difference and
centred difference in time, respectively.  Similarly, there are three
obvious ways of discretizing $\partial T / \partial x$ at the $n$th
timestep and $i$th spatial point:
\BEQ \frac{T_{i+1}^n - T_i^n}{\Delta x} \;\;\; , \;\;\; 
\frac{T_i^n - T_{i-1}^n}{\Delta x} \;\;\; , \;\;\; 
\frac{T_{i+1}^n - T_{i-1}^n}{2\Delta x} . \EEQ

By combining these possible discretizations --- three for $\partial T
/ \partial t$ and three for $\partial T / \partial x$ --- we could
write down \emph{nine} possible numerical schemes for the advection
equation.  At first sight the nine schemes all seem equally
reasonable, but they are not. 

Firstly, the centred scheme has a higher {\em order of
	approximation}. This can be seen by expressing temperature at position
$x+\Delta x$ in terms of a Taylor expansion about position $x$:
\BEQ T(x+\Delta x) = T(x) + T'(x)\Delta x + \frac{1}{2} T''(x) (\Delta x)^2 + 
\frac{1}{6} T'''(x) (\Delta x)^3 +O(\Delta x^4) \EEQ

where $T'=\partial T/\partial x$, $T''=\partial^2 T/\partial x^2$ and so on.
Re-arrangement shows that:
\BEQ
T'(x)=\frac{ T(x+\Delta x)-T(x)}{\Delta x}-\frac{1}{2} T''(x) \Delta x+...
\EEQ

so the error in approximating the first derivative by a forward
difference is proportional to $\Delta x$ or {\em first order} assuming
that temperature variations have a characteristic lengthscale much
longer than the grid-scale ($\Delta x/L\ll 1$)\footnote{Scaling
	arguments assume wavelike structure and $\partd{}{x}\sim
	\frac{1}{L}=k$.}.

The order of approximation for the centred difference is obtained by subtracting the expansion for $T(x-\Delta x)$:
\BEQ T(x-\Delta x) = T(x) - T'(x)\Delta x + \frac{1}{2} T''(x) (\Delta x)^2 - 
\frac{1}{6} T'''(x) (\Delta x)^3 +O(\Delta x^4) \EEQ

from $T(x+\Delta x)$ giving a second order leading error $\sim
\frac{1}{3}T'''(x)\Delta x^2$. If the temperature variations are
smooth, the higher order scheme will be more accurate.

\subsubsection{Numerical stability: The FTCS advection scheme}

Let us try using the forward difference in time and the
centred difference in space (FTCS for short).
The scheme is:
\BEQ \frac{T_{i}^{n+1}-T_{i}^{n}}{\Delta t} + u \frac{T_{i+1}^{n} - T_{i-1}^{n}}{2\Delta x} = 0 \EEQ
We can rewrite this equation to get:
\BEQ T_{i}^{n+1} = T_{i}^{n} - \frac{u\Delta t}{2\Delta x} \left[ T_{i+1}^{n} - T_{i-1}^{n} \right]
\label{ftcsadvection}\EEQ

At each time step, $n$, we apply this equation at each spatial point,
$i$.  A computer program to do this would therefore have a loop over
$i$ \emph{within} a loop over $n$.

A possible way of examining the stability of this scheme is the
\emph{von Neumann stability analysis}.
To understand how this works, first note that a particular solution
of the continuous advection equation (\ref{advectioncont}) is
\BEQ T(x,t)=\cos k(x-ut) .\EEQ
{\em Exercise: Show that this is indeed a solution.}

The von Neumann method starts with a {\em trial} solution:
\BEQ T_i^n = \cos kx_i , \EEQ
and considers what the numerical scheme does over one time step.
We know that the numerical scheme is just an approximation:
it might change the amplitude of the cosine
wave and the phase shift might be wrong.  Let us write 
the outcome from the numerical scheme after one time step as:
\BEQ T_i^{n+1} = A\cos (kx_i-\phi),\EEQ
where $A$ and $\phi$ are unknowns that we want to work out.  If our
numerical scheme were exact, we would find $A=1$ and $\phi=k u \Delta
t$.  When we substitute the above equations for $T_i^n$ and
$T_i^{n+1}$ in the numerical scheme (\ref{ftcsadvection}) we find:
\BEQ A\cos (kx_i-\phi) = \cos kx_i - \frac{u \Delta t}{2\Delta x}\left( \cos kx_{i+1} - \cos kx_{i-1} \right). \EEQ
Remembering that $x_{i+1}=x_i+\Delta x$ and $x_{i-1}=x_i-\Delta x$,
and using the fact that $\cos (a+b) = \cos a\cos b -\sin a\sin b$,
we can rearrange this equation to find:
\BEQ A (\cos kx_i\cos\phi +\sin kx_i\sin\phi) = \cos kx_i + \frac{u\Delta t}{\Delta x}\sin kx_i\sin k\Delta x. \EEQ
This equation should be valid for all $x_i$.
Equating the terms in $\cos kx_i$ and $\sin kx_i$ we find, respectively:
\begin{eqnarray}
	A \cos \phi & = & 1 \\
	A \sin \phi & = & \frac{u\Delta t}{\Delta x}\sin k\Delta x .
\end{eqnarray}
By adding the squares of these equations we find that
\BEQ A^2 = 1+\left(\frac{u\Delta t}{\Delta x}\sin k\Delta x\right)^2. \EEQ 
We see that the amplification factor $A$ is always bigger than one: in this
sense the scheme is \emph{unconditionally unstable} and the solutions
will amplify every timestep.

\subsubsection{Computational modes: The CTCS advection scheme}

So it's back to the drawing board. Let us try using the centred difference in time and the
centred difference in space (CTCS for short):
\BEQ \frac{T_{i}^{n+1}-T_{i}^{n-1}}{2\Delta t} + u \frac{T_{i+1}^{n} - T_{i-1}^{n}}{2\Delta x} = 0 \EEQ
We can rewrite this equation to get:
\BEQ T_{i}^{n+1} = T_{i}^{n-1} - \frac{u\Delta t}{\Delta x} \left[ T_{i+1}^{n} - T_{i-1}^{n} \right] \label{ctcsmodel} \EEQ
The quantity $u\Delta t / \Delta x$ is dimensionless.  It appears so often in
numerical schemes for the advection equation that it has its own name --- the
\emph{Courant number}:
\BEQ \alpha = \frac{u\Delta t}{\Delta x} .\EEQ

The von Neumann stability analysis for the CTCS scheme is most easily
done using complex notation for the trial solution, 
\BEQ T_i^n=e^{ikx_i}\;\; ; \;\;\;\; T_i^{n+1}=ST_i^n \EEQ
where $S=Ae^{i\phi}$ expresses 
the change in amplitude and phase over one time-step.
Plugging the trial solution into the model (\ref{ctcsmodel}) yields:
\BEQ
S=S^{-1}-\alpha\left[ e^{ik\Delta x}-e^{-ik\Delta x} \right]
\EEQ 

which can be re-arranged as a quadratic equation
for the complex factor, $S$: 
\BEQ
S^2+S\alpha 2i\sin k\Delta x -1=0
\EEQ

which has two roots:
\BEQ
S_{\pm}=-i \alpha \sin k\Delta x\pm \sqrt{1-\alpha^2 \sin^2 k\Delta x}
\label{asoln}
\EEQ

If the Courant number $|\alpha | \le 1$, the number under the root must
be positive since $\sin^2 k\Delta x \le 1$. 

{\em Exercise: Show that when $|\alpha | \le 1$ the magnitude of the
	amplification factor $|S|^2=S^* S=1$, meaning that the scheme is
	stable. However, when $|\alpha | > 1$ the $S_{-}$ solution is
	unstable.}

This is an improvement on the FTCS scheme, which was unconditionally
unstable. This is the single most important result in the numerical
modelling of fluid flows.  It is called the Courant-Friedrichs-Lewy
(CFL) condition.  What it boils down to is that, for a given
grid-spacing $\Delta x$, the time step must satisfy \BEQ \Delta t <
\frac{\Delta x}{u} . \EEQ The physical interpretation of this condition
is that \emph{the scheme is unstable if fluid parcels move more than
	one gridbox in one timestep}.

The solution with +ve real part in (\ref{asoln}) does not change sign
every timestep and is called the {\em physical mode}. However, the
solution with the -ve square root alternates sign every step which is
unphysical behaviour and is called the {\em computational mode}. The
relative amplitudes of the two modes is determined by projection from
the initial conditions. The implication is that although the CTCS
scheme is stable for $|\alpha | \le 1$ we can expect unphysical
oscillations that are worst at the grid-scale (where $\sin k\Delta
x=1$).

\begin{figure}
	%\begin{center}
	\mbox{\scalebox{0.1}{\includegraphics{Figures/adv-ctcs.png}}
		\scalebox{0.1}{\includegraphics{Figures/adv-upstream.png}}}
	%\end{center}
	\caption{\textsl{Comparing numerical solutions for the advection of a top-hat distribution by uniform flow. Left: CTCS solution exhibits unphysical undershoots and overshoots. Right: Upstream scheme remains monotonic.}}
	\label{fig:ctcs}
\end{figure}


\subsubsection{Numerical dispersion}

The phase shift in one time-step for a wave moving with speed $c$
would be $\phi=-kc\Delta t$. By comparing the imaginary part of
$e^{i\phi}$ with (\ref{asoln}) we obtain the phase speed as
represented by the numerical scheme: 
\BEQ kc\Delta t=\sin^{-1}\left\{
\alpha \sin k\Delta x \right\} \EEQ

While for the exact solution the tracer pattern must move with the
uniform velocity $u$, irrespective of its shape or lengthscale, the
numerical solution moves with a phase speed dependent on the
wavenumber of the tracer pattern, $k$. The ratio of numerical to exact speed, $\alpha^*/\alpha$ is plotted in Figure \ref{fig:disperse}, which shows that long waves ($k\Delta x \ll 1$) move almost at the flow speed, but short waves move much more slowly.\\
Waves in the CTCS numerical model are {\em
	dispersive}, even though they should not be. Note also that the
computational mode moves with the same phase speed as the physical
mode but in the opposite direction to the flow. The net effect is
severe for sharp features where small scale waves propagate out behind
the leading edge of tracer.

%\begin{figure}
\begin{figure}
	%\centering
	\scalebox{0.19} {\includegraphics{Figures/alphastar-vs-kdx.png}}
	\caption{\textsl{Ratio of numerical speed to flow speed as a function of wavenumber, $k\Delta x$, for the CTCS numerical scheme.}}
	\label{fig:disperse}
\end{figure}
%\end{figure}


\subsubsection{Numerical dissipation}

The CTCS scheme maintained wave amplitude for all
wavenumbers. However, many schemes result in a decay of amplitude with
time which is faster at short wavelengths. One example is the {\em
	first order upstream scheme}:
\BEQ
\frac{q_i^{n+1}-q_i^n}{\Delta t}+u\frac{q_{i}^n-q_{i-1}^n}{\Delta x}=0
\EEQ

which is ``upstream'' for $u>0$ since the interval $x_{i-1}\rightarrow
x_i$ is upstream of point $x_i$. 

{\em Exercise: Show using von Neumann stability analysis that the
	upstream scheme results in a wave amplitude $|S|$ that decays with
	time.}

In the model, short waves decay more rapidly than long waves, with the
net result that the tracer distribution becomes smoother with
time. This effect is called {\em numerical dissipation} or {\em
	numerical diffusion}. The advantage is that unphysical overshoots are
eliminated (Fig.~\ref{fig:ctcs}).

\subsection{The Diffusion Equation}

\textsl{Diffusion, partial differential equations, 
	von Neumann stability analysis, boundary conditions.}

A second major transport process operating in gases, liquids and
solids is diffusion. As an example, think of the layers of soil near
the surface of the Earth. In the top layers soil temperature varies
with air temperature, but deep down below the surface, the temperature
is more-or-less constant. Over time, heat is transferred between the
surface and lower levels.

There is a {\em flux} of heat (denoted by $F$) through the surface
of the Earth defined so that $F>0$ implies a flow of heat in the
direction of increasing depth ($z$). Each layer of soil will have a
flux of heat in from above and a flux of heat out from below. If
the flux in and the flux out are equal, the temperature of the layer
will remain constant. The temperature will only change if there is a
{\em flux gradient}. We can define $F$ so that the rate of change of
temperature $T$ is equal to minus the gradient of $F$,
\begin{equation}
	\partd{T}{t} = -\partd{F}{z}.
	\label{eq:flux-tend}
\end{equation}
{\em i.e.}, if more heat flux enters than leaves a slab of soil,
the temperature of the slab must increase.

When an object is placed into contact with an environment with lower
temperature, heat can flow from hot to cold, as with the cup of tea
example. In a solid, molecules in hot regions jiggle more than in cold
regions and the heat is transferred as the jiggling molecules push and
pull their neighbours through intermolecular forces. In a fluid,
molecules can move and collide with each other resulting in a {\em
	random walk}. If a molecule travels from a hot region and collides
with a molecule in a neighbouring colder region it will transfer some
energy from hot to cold. Here we
assume that all these physical processes result in a flux that is {\em
	proportional} to the temperature gradient locally, {\em i.e.,} giving
the {\em flux-gradient relation}:
\begin{equation}
	F = -D(z) \partd{T}{z}.
	\label{eq:flux-val}
\end{equation}
The {\em diffusion coefficient} $D(z)$ is positive and the `--' sign
shows us that the flux is {\em down-gradient} from hot to cold.

Plugging eqn~(\ref{eq:flux-val}) into eqn~(\ref{eq:flux-tend})
gives:
\begin{eqnarray}
	\partd{T}{t} & = & \partd{}{z} \left(
	D(z) \partd{T}{z}
	\right)		\\
	& = & D \partd{^{2}T}{z^2}
	~~\mbox{if $D$ is constant}
	\label{diffeqn}
\end{eqnarray}
Physically, the diffusion equation can be interpreted as describing
the way heat travels from warm areas to cold areas. The diffusion
equation will take an initial temperature distribution $T(z)$ and
`smear it out' over time, eventually leaving a constant temperature
gradient $\partial T/\partial z$.

This is a \emph{second-order} partial differential
equation, because $T$ is differentiated twice with respect to $z$ on the 
right-hand side.  Compare this with the advection equation from the previous 
chapter, which was a \emph{first-order} partial differential equation.
The diffusion equation is therefore slightly more complicated.

{\em Exercise: Show that $T(z,t) = \mathrm{e}^{-D k^2 t} \cos kz$ is a
	solution of the diffusion eqn~(\ref{diffeqn}).}

\subsubsection{Numerical schemes for the diffusion equation}

Let us look at how we might solve the diffusion equation on a
computer. One way forward is again to replace the derivatives with
finite differences. As before, we can consider a series of {\em
	snapshots} of the temperature $T$ each separated by the time-step
$\Delta t$. The depth $z$ can be similarly {\em discretised}: we can divide
the distance between the surface and bottom of the soil at $z=H$ into
$J$ slabs of depth $\Delta z=H/J$ and consider the characteristic
temperature of each slab (Fig.~\ref{soil}).

\begin{figure}
	\begin{center}
		\scalebox{0.15}{\includegraphics{Figures/temp-diffuse.png}}
	\end{center}
	\caption{\textsl{Diffusion of heat through soil near the Earth's surface.}}
	\label{soil}
\end{figure}

As in the previous chapter, to shorten the equations we define
\BEQ T(z_i,t_n) = T_i^n . \EEQ
To discretize the right-hand side of the diffusion equation, 
consider the following two Taylor expansions:
\BEQ T(z+\Delta z) = T(z) + T'(z)\Delta z + \frac{1}{2} T''(z) (\Delta z)^2 + 
\frac{1}{6} T'''(z) (\Delta z)^3 +O(\Delta z^4) \EEQ
\BEQ T(z-\Delta z) = T(z) - T'(z)\Delta z + \frac{1}{2} T''(z) (\Delta z)^2 - 
\frac{1}{6} T'''(z) (\Delta z)^3 +O(\Delta z^4) \EEQ
Adding these two expansions together we find that
\BEQ T(z+\Delta z) + T(z-\Delta z) = 2T(z) + T''(z) (\Delta z)^2 +O(\Delta z^4), \EEQ
which we can re-arrange to give
\BEQ T''(z) = \frac{T(z+\Delta z) - 2T(z) + T(z-\Delta z)}{(\Delta z)^2}+O(\Delta z^2) .\EEQ
Our discretization of the second derivative $\partial^2 T / \partial z^2$ is therefore a {\em second order approximation} and 
at the $i$th grid-point and the $n$th time-point can be written:
\BEQ \frac{T_{i+1}^n-2T_i^n+T_{i-1}^n}{(\Delta z)^2} . 
\label{d2t}
\EEQ
To discretize the left-hand side of the diffusion equation, we could choose
a forward in time, backward in time, or centred in time scheme.  Here, we will
only study the first of these three possible choices, the FTCS scheme:
\BEQ \frac{T_i^{n+1}-T_i^n}{\Delta t} = D \; \frac{T_{i+1}^n-2T_i^n+T_{i-1}^n}{(\Delta z)^2} 
\label{ftcs:diff}
\EEQ
which can be re-arranged to give
\BEQ T_i^{n+1} = T_i^n + \gamma ( T_{i+1}^n -2T_i^n + T_{i-1}^n ) \label{diffusionscheme}\EEQ
where we have defined a dimensionless model parameter
\BEQ \gamma = \frac{D\Delta t}{(\Delta z)^2} . \EEQ

\subsubsection{Stability analysis}

As for advection, the stability of our numerical scheme can
be assessed using von Neumann stability analysis.  Remember that an
exact solution of the continuous diffusion equation (\ref{diffeqn}) is
$T(z,t) = \mathrm{e}^{-D k^2 t} \cos kz$.  As $t$ increases, the phase
of the cosine wave does not change but its amplitude is reduced.  We
must determine if our numerical scheme also has these properties.
Once again we begin with the {\em trial solution}:
\BEQ T_i^n = \cos kz_i , \EEQ
and examine the outcome from the numerical scheme after one time step:
\BEQ T_i^{n+1} = A\cos (kz_i-\phi),\EEQ
where $A$ and $\phi$ are the unknown amplitude and phase that we want
to work out.  When we substitute these expressions for $T_i^n$ and
$T_i^{n+1}$ into the numerical scheme (\ref{diffusionscheme}) we find:
\BEQ A\cos (kz_i-\phi) = \cos kz_i + \gamma ( \cos kz_{i+1} - 2\cos kz_i + \cos kz_{i-1} ). \EEQ
Remembering that $z_{i+1}=z_i+\Delta z$ and $z_{i-1}=z_i-\Delta z$,
we find:
\BEQ A (\cos kz_i\cos\phi +\sin kz_i\sin\phi) = \cos kz_i + 2\gamma (\cos k\Delta z -1) \cos kz_i . \EEQ
Since this equation must be valid for all $z_i$, we can equate
the terms in $\cos kz_i$ and $\sin kz_i$ to find, respectively:

\begin{eqnarray}
	A \cos \phi & = & 1 + 2\gamma (\cos k\Delta z -1) \\
	A \sin \phi & = & 0 .
\end{eqnarray}

Since we are not interested in the case when $A=0$, the second of
these equations tells us that $\phi=0$.  In other words, the phase of
the cosine wave is not changed by our numerical scheme.  This is good
news because it agrees with the true solution.  Substituting $\phi=0$
into the first equation, and remembering that $\cos 0=1$,
we obtain a formula for the amplification factor:
\BEQ A = 1 + 2\gamma (\cos k\Delta z -1) \EEQ
If our numerical scheme is to be stable for a general diffusion
problem, then it must be stable for all possible values of $k$.  There
will always be a value of $k$ for which $\cos k\Delta z = -1$, and it
is this extreme value of $k$ that will determine the stability.  For
this value of $k$ the amplification factor becomes
\BEQ A = 1 -4 \gamma .\EEQ
The stability of the numerical solution depends solely on the value of
$\gamma$.  There are three cases we need to consider:

\begin{tabbing}
	$0<\gamma < \frac{1}{4}$~~~~ \= Implies $0< A<1$: the solution decays monotonically like the exact solution \\
	$\frac{1}{4}<\gamma < \frac{1}{2}$ \> Implies $-1< A<0$: the solution decays but oscillates unphysically \\
	$\gamma > \frac{1}{2}$ \> Implies $A<-1$: the solution is unstable and oscillates with growing amplitude
\end{tabbing}

We conclude that the FTCS numerical scheme for the diffusion equation is
\emph{conditionally stable}.  The condition for realistic behaviour is
$\gamma<1/4$ or equivalently:  
\BEQ \Delta t < \frac{(\Delta z)^2}{4D} . \EEQ

\subsection{Boundary conditions}

We need to consider \emph{boundary conditions} when solving the
diffusion equation or the advection equation.  For example, to apply
the spatial discretization (\ref{d2t}) to a soil layer we require
model layers above and below it (see Figure~\ref{soil}). We run into
problems with the top model layer, because there is no layer above and
we must employ a {\em boundary condition}.

These are commonly-used boundary conditions:

\paragraph{No flux boundary conditions} 

assume that the fluxes across the top and bottom of the domain are
zero. We can discretise eqn~(\ref{eq:flux-tend}) as follows:

\begin{equation}
	\frac{T^{n+1}_i-T^n_i}{\Delta t} \approx 
	-\frac{F_{i+1/2}-F_{i-1/2}}{\Delta z} 
	\label{fluxdiff}
\end{equation}

where the flux gradient has been estimated by a finite difference
between the top and bottom interface of each layer. The flux at the
interface $z_{i+1/2}$ midway between levels $z_i$ and $z_{i+1}$ is
also defined by centred difference of eqn~(\ref{eq:flux-val}):
\BEQ F_{i+1/2}\approx -D
\frac{(T^n_{i+1}-T^n_i)}{\Delta z}.
\label{flux-cen}  
\EEQ 
The no flux boundary condition is implemented in the top layer ($i=1$) using
$F_{1/2}=F(0)=0$ in eqn.(\ref{fluxdiff}) giving:
\begin{eqnarray}
	\frac{T^{n+1}_1-T^n_1}{\Delta t} & = & 
	-\frac{F_{3/2}}{\Delta z} \\\nonumber
	& = & D\frac{T^n_2-T^n_1}{(\Delta z)^2}
\end{eqnarray}
and similarly $F_{J+1/2}=F(H)=0$ is used in the equation for the
bottom layer. 

{\em Exercise: Show that the numerical scheme for the diffusion
	equation at any level-$i$ (away from the boundaries) obtained by
	substituting eqn~(\ref{flux-cen}) into eqn~(\ref{fluxdiff}) is
	identical to eqn~(\ref{ftcs:diff}).}

\paragraph{Fixed value boundary conditions} 

assume that the temperature at the boundaries is specified. In the
soil example, the top boundary is at air temperature and the lower
boundary is at the underlying rock temperature. In our numerical
scheme the boundary conditions are best included when the fluxes at
the boundaries are estimated using eqn.(\ref{flux-cen}). For example,
at the top boundary:
\BEQ
F_{1/2}=F(0)\approx -D \frac{(T^n_1-T_{AIR})}{\frac{1}{2}\Delta z}
\EEQ
and then this is plugged into eqn.(\ref{fluxdiff}).

The boundary conditions have a profound effect on the physics of the
solution and the equilibrium temperature profile that the soil tends
towards. With no flux boundary conditions, the vertical average of
temperature must stay constant with time because no heat enters or
leaves the soil. However, the temperature at the top and bottom of the
soil layer can change. With fixed value boundary conditions, the
temperature at the boundaries cannot change but the vertical average
can, implying a net warming or cooling of the soil.

\paragraph{Periodic boundary conditions}

\begin{figure}
	\begin{center}
		\scalebox{0.8}{\includegraphics{Figures/adv-grid.png}}
	\end{center}
	\caption{\textsl{Model grid-points around latitude circle.}}
	\label{adv-grid}
\end{figure}

In many advection problems we assume that the domain is {\em
	periodic}. For example, we might consider advection by a jet blowing
around a latitude circle. If the jet does not deviate to the north or
south, then a tracer leaving Reading will return to Reading after
circulating the globe. In Figure~\ref{adv-grid}, a tracer value crossing $x=L$
will re-appear at $x=0$. The discretised {\em periodic boundary
	conditions} can be written:

\begin{equation}
	T(x_1-\Delta x)=T(x_J)~~~ ; ~~~T(x_J+\Delta x)=T(x_1)
\end{equation}

where there are $J$ grid-points across the domain with regular spacing
$\Delta x=L/J$. The two temperature values above, at points outside the
domain $0<x<L$, can be used to find the temperature gradient at the
boundaries.

\paragraph{Kinematic BCs}

The fluid moves with the boundary in the direction normal to the boundary.
\BEQ
{\bf u.n}={\bf u}_b.{\bf n}
\EEQ

\paragraph{Dynamic BCs}

There is a balance of internal fluid forces normal to the boundary. If
the boundary is not rigid and there is fluid on both sides this
amounts to pressures being equal on both sides of the boundary: 
\BEQ
p_1=p_2 
\EEQ

\paragraph{Flow parallel to boundaries}

The {\bf no slip} condition is that the fluid moves with the boundary
\BEQ
{\bf u}={\bf u}_b
\EEQ
in contrast to {\bf free slip} where the fluid is free to move along
the boundary as if no friction were acting. This is unrealistic at
finescales but is used when the fluid away from boundaries
is being modelled without dealing with the {\em boundary layer}. 

\vspace{1em} 

%%%%%%%%%%%%%%%%%%%%%%%%%%%%%%%%%%%%%%%%%%%%%%%%%%
\newpage

%%%%%%%%%%%%%%%%%%%%%%%%%%%%%%%%%%%%%%%%%%%%%%%%%%
\newpage
\chapter{Complex numbers}

\section{Euler relations}
\href{https://github.com/pierluigividale/MTMW14/blob/main/Notebooks/Euler%20equations.ipynb}{Link to Euler formulae Notebook}

\section{Representation of wave functions in the complex plane}

\section{What are the benefits of using such wave functions}
Link to iPython Notebook

\vspace{1em} 


\chapter{Vorticity equation derivation}

\section{What is PV?}

\subsection{Derivation of an important conservative quantity: PV}

We saw in Lecture 4 how we can derive the Shallow Water Equations (SWEs). There are three independent variables: $u,v,p$ in that system:
\begin{eqnarray}
	\partd{u}{x}+\partd{v}{y}+\partd{w}{z} & = & 0 \\
	\lagd{u}{t}-fv & = & -\frac{1}{\rho_o}\partd{p}{x} \\
	\lagd{v}{t}+fu & = & -\frac{1}{\rho_o}\partd{p}{y}
\end{eqnarray}

Using the vertically integrated hydrostatic balance: $-\frac{1}{\rho_o} \partd{p}{x} = -g \partd{h}{x}$, we can write: 
\begin{eqnarray}
	\lagd{u}{t} & = & -g\partd{h}{x} +fv\\
	\lagd{v}{t} & = & -g\partd{h}{y} -fu
\end{eqnarray}

It is possible to combine the three governing equations into a single one by cross-differentiation of the two momentum equations and substitution of mass continuity:
\begin{eqnarray}
	&&\frac{\partial}{\partial y} \left( \lagd{u}{t} = -g\partd{h}{x} +fv \right) \\
	&&\frac{\partial}{\partial x} \left( \lagd{v}{t} = -g\partd{h}{y} -fu \right) 
\end{eqnarray}

Subtract the second from the first equation; use the definition of relative vorticity: $\zeta = \frac{\partial v}{\partial x} - \frac{\partial u}{\partial y}$:

\begin{eqnarray}
	&&\frac{D_h}{Dt}\left({\zeta+f}\right) = -(\zeta+f) \left( \frac{\partial u}{\partial x} + \frac{\partial v}{\partial y} \right);\\
	&&\lagdh{h}{t} = -h \left( \frac{\partial u}{\partial x} + \frac{\partial v}{\partial y} \right)
\end{eqnarray}
to yield:
\begin{eqnarray}
	\frac{D_h}{Dt}\left[\frac {\zeta+f}{h} \right]=0
	\label{PVcons}
\end{eqnarray} 

\subsection{Dispersion relation for Rossby waves on $\beta$-plane}

To derive the dispersion relation for Rossby waves we revert to the use of basic state and perturbations: $u=\bar{u}+u'; v=v'; \zeta = \bar{\zeta}+\zeta'$ on a $\beta$-plane: $f=f_0+\beta y$ in the barotropic vorticity equation: $\frac{D_h}{Dt}\left({\zeta+f} \right)=0$
and making use of a streamfunction $\psi$ to define: $u'=-\partd{\psi'}{y}; v'=-\partd{\psi'}{x}$ to yield:
\begin{eqnarray}
	\zeta'&=&\nabla^2\psi'\\
	\left(\partd{}{t}+\bar{u}\partd{}{x}\right)\nabla^2\psi'+\beta \partd{\psi'}{x}&=&0
\end{eqnarray}

and inserting a solution of this type: $\psi'=Re[\Psi exp(i\phi)]$, where $\phi=kx+ly-\omega t$ results in:

\begin{eqnarray}
	(-\omega + k\bar{u})(-k^2-l^2)+k \beta & = & 0\\
	c_x-\bar{u} & = & -\frac{\beta}{K^2}
\end{eqnarray}

where: $\omega=\bar{u}k-\frac{\beta k}{K^2}$ and $K^2=k^2+l^2$.\\


\vspace{1em} 

%%%%%%%%%%%%%%%%%%%%%%%%%%%%%%%%%%%%%%%%%%%%%%%%%%
\newpage
\chapter{Eigenvalues and eigenvectors}

\section{How can we perform stability analysis in 2D?}

\section{What are eigenvalues and eigenvectors, and how can they help?}
Link to iPython Notebook

\vspace{1em} 

%%%%%%%%%%%%%%%%%%%%%%%%%%%%%%%%%%%%%%%%%%%%%%%%%%
\newpage
\chapter{The Lorenz Model}
%%%%%%%%%%%%%%%%%%%%%%%%%%%%%%%%%%%%%%%%%%%%%%%%%%%%

\section{What is chaos?}

\section{Resource: the Lorenz model}
\href{https://www.dropbox.com/scl/fi/rw3p2aig729zogynv6r0l/run_Lorenz_example.ipynb?rlkey=dw8byerasoreiumat80r0frea&dl=0}{Link to iPython Notebook on Lorenz's 1963 model}

\vspace{1em} 

% ----------------------------------------------------------------------------------------
% 	BIBLIOGRAPHY
% ----------------------------------------------------------------------------------------

\begin{thebibliography}{9}
	
	\printbibliography[%
	heading=bibempty
	]
	
\end{thebibliography}

