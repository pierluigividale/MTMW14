%%%%%%%%%%%%%%%%%%%%%%%%%%%%%%%%%%%%%%%%%%%%%%%%%%%%%%%%%%%%%%%%%%%%	
\chapterimage{2613-1477-max.jpg} % Chapter heading image	
\chapter{Courant-Friedrichs-Levy (CFL) condition}
%%%%%%%%%%%%%%%%%%%%%%%%%%%%%%%%%%%%%%%%%%%%%%%%%%%%

\section{Re-visiting advection and time step limitations}

In this chapter we are going to re-visit some of the ideas about numerical stability in 1D, then moving on to 2D flows, and relative stability, finally to consider aspects of coupling between equations.

\subsection{Horizontal advection}

In atmosphere or ocean science we often consider quantities that are
carried with the flow without changing. Examples include the mixing
ratio of an unreactive trace gas or potential temperature. Such
quantities can be used to mark air parcels and follow them and
therefore they are called {\em tracers}. If the wind field ${\bf
	u}({\bf x}, t)$ is known the {\em trajectories} of air parcels can be
found by numerically integrating the ODE:
\begin{equation}
	\frac{D{\bf x}}{Dt}={\bf u}({\bf x}, t)
\end{equation}
where ${\bf x}$ is the position of an air parcel and the capital
$D/Dt$ is used to mean rate of change following an air parcel. 
The tracer value, $\theta_j$, does not change following parcel $j$ so
if we know the values $\theta_j$ for all parcels at time $t_i$ and
their initial positions, we can use their trajectories
to infer the new distribution of tracer at a later time $t_f$. A model
based on following fluid parcels is called a {\em Lagrangian model}.

However, many atmospheric processes occur relative to a fixed
position. Examples include a pollution source from a city and
temperature fluxes from the land or ocean. Sometimes it is better
devise a model which considers atmospheric changes relative to fixed
points. These are called {\em Eulerian
	models}. What happens when air masses carrying tracers pass over a
point?

\subsection{The Advection Equation}

\begin{figure}[h!]
	\begin{center}
		\includegraphics[scale=0.35]{/Users/vidale/Projects/Teaching/MTMW14/Figures/advection.pdf}
	\end{center}
	\caption{Temperature at a point increases when wind blows from the warm side.}
	\label{fig:kencfl}
\end{figure}


In Figure~\ref{fig:ken}, the wind is bringing air from the $-x$
direction. If the air `upstream' is warmer than the air `downstream',
the observer will see the temperature increasing. The rate of change of
temperature observed must depend on both the magnitude of this
gradient and the speed at which the air is moving, {\em i.e.,}
\begin{equation}
	\frac{\partial T}{\partial t} =- u \frac{\partial T}{\partial x}.
	\label{advectioncont}
\end{equation}
where the left hand side means the rate of change at a given position
$x$ and the right hand side is called the {\em advection term}. $u$ is
the wind component and $\partial T/\partial x$ means the gradient in
the $x$-direction at a given time. 

We can solve this analytically, but the question is what happens when we solve it numerically. As an example, let us look at CTCS.


\subsection{A graphical description of what happens with finite differences}

	Let us continue with something very familiar in numerical modelling, e.g. simple advection solved with CTCS.
		
	We have our beloved CFL criterion from studying the problem of advection:
	$\alpha = U \frac{\Delta t}{\Delta x} \le \alpha_{max}$
	, where $\Delta x$ is our grid spacing (at times $\Delta x$ will be written as $\Delta h$, else  using the shorthand $d$).\\

\begin{figure}[h!]
	\begin{center}
\includegraphics[trim={0cm .8cm 0cm .5cm},clip,width=0.8\textwidth]{Figures/image_grid_cfl.jpg}
	\end{center}
\caption{CFL in the x direction}
\end{figure}
	
What is the practical meaning of CFL for Eulearian finite differences?

If we think of it as a method for propagating information in space, the CFL criterion is telling us that we are not allowed to transmit information from grid point $i$ to grid point $i+2$ without first making a stop at grid point $i+1$.
If our information "leapfrogs", very bad things will happen.
	
\subsection{A short reminder of the meaning of the CFL condition}
\subsection{The example of advection in 1D}

	Let us start with something familiar, e.g. simple advection solved with CTCS:
	
	\begin{equation}
		\frac {A_j^{n+1}-A_j^{n-1}}{ 2 \Delta t} +c \left(  \frac {A_{j+1}^{n}-A_{j-1}^{n}}{ 2 \Delta x}  \right) = 0
		\label{adv-CTCS-1D}
	\end{equation}
	
	We can substitute a (Von Neumann) solution of this type: ${A_j^{n}=B^{n \Delta t} e^{i \mu m \Delta x}}$, where $\mu$ is the horizontal wavenumber and $m \Delta x$ is the distance along the $x$ axis. $B$ can be any complex number. If we substitute into our finite difference (CTCS) scheme above we obtain:
	
	\begin{equation}
		\left ( B^{(n+1)\Delta t}-B^{(n-1) \Delta t} \right ) e^{i \mu m \Delta x} = - \frac {c \Delta t}{\Delta x} B^{n \Delta t} \left( e^{i \mu (m+1) \Delta x}  - e^{i \mu (m-1) \Delta x}  \right)
		\label{adv-CTCS-1D-solution}
	\end{equation}
	
	If we remember Euler's formula: $e^{i \theta}=\cos \theta +i \sin \theta$ and we multiply the expression above by $B^{\Delta t}$ we obtain a simple equation, after we cancel out the common term $A_j^{n}$:
	
	\begin{equation}
		B^{2 \Delta t}    + 2 i \sigma B^{\Delta t}  -1 = 0  
		\label{adv-CTCS-1D-stability}
	\end{equation}
	
	where $\sigma = \frac {c \Delta t}{\Delta x} \sin \mu \Delta x$, so that (\ref{adv-CTCS-1D-stability}) has the solution:
	
	\begin{equation}
		B^{\Delta t} = -i \sigma \pm \left( 1 - \sigma^2 \right)^{1/2} 
		\label{adv-CTCS-1D-eigenvalues}
	\end{equation}
	

	Two cases may be considered: \textbf{stable}, with $|\sigma| \leq 1$ and \textbf{unstable}, with $|\sigma| > 1$.
	\medskip
	
	The \underline{stable case} is very interesting from the point of view of phase, group velocity etc., but does not seem to pose major threats to the stability of the simulation (see Haltiner and Williams, pages 112-119).
	
	\medskip
	The \underline{unstable case} concerns us from the point of view of the amplitude of the signal. In fact, if $|\sigma| > 1$, then $\left( 1 - \sigma^2 \right)^{1/2}$ is imaginary and both roots will be pure imaginary:
	
	
	\begin{align}
		B_{+}^{\Delta t}  &= -i \left ( \sigma -S \right)  \mathrm{,where } | \sigma| > S \equiv \left(  \sigma^2 - 1 \right)^{1/2} \nonumber \\
		B_{-}^{\Delta t}  &= -i \left ( \sigma +S \right)  \nonumber
		\label{adv-CTCS-1D-unstable-solution}
	\end{align}
	
	If $\sigma$ is positive, the magnitude $R=|B_{-}^{\Delta t}| > 1$ and the solution $ B_{-}^{\Delta t} = R e^{-i \pi /2 }$ will thus grow exponentially when raised to the power of $n$ across the time steps. If $\sigma$ is positive, the other root has a magnitude exceeding $1$. \\
	In either case, the solution: $A_j^{n} = \left( M B_{+}^{n \Delta t}+ E B_{-}^{n \Delta t} \right)e^{i \mu m \Delta x}$  (with two arbitrary constants, $M,E$, to be determined later), will amplify with increasing time, which is not a desired property, since it does not correspond to the true solution of the differential equation. This phenomenon of \emph{exponential amplification of the solution} is known as \emph{computational instability} and must be avoided at all costs. 
	
\subsection{The Courant-Friedrichs-Levy (CFL) condition in 1D}

	So, we are left with having to impose a \emph{condition for a stable solution} $|\sigma| \leq 1$, that is:
	
	\begin{equation}
		\left | \frac {c \Delta t}{\Delta x} \sin \mu \Delta x \right | \leq 1
	\end{equation}	
	
	If this condition is to hold for all admissible values of the wavelength $\mu$, then:
	
	
	\begin{theorem}[Courant-Friedrichs-Levy (CFL) condition for computational stability in 1D]		
	The maximum value of $\sin \mu \Delta x$ will happen for the highest resolved wavenumber (that is a wavelenght $L=4 \Delta x$), which requires that:
	\begin{equation}
		\left | \frac {c \Delta t}{\Delta x}  \right | \leq 1       \textrm{  ~~  or, with the notation used later ($\Delta x = d$): ~~ } \left | \frac {V_s \Delta t}{d}  \right | \leq 1 
	\end{equation}	
	\end{theorem}

	which is commonly referred to as the \emph{\textbf{Courant-Friedrichs-Levy} condition for computational stability} (in one dimension).

\subsection{The CFL condition in 1D: plugging in some numbers}

What happens in practice? What typical time step are we contending with, in the horizontal and vertical directions?

	{\bf Horizontal example:} $\Delta x = 10km$, $c = O(100) ms^{-1}$ what $\Delta t$ can we afford?
		
	But what will happen for a contemporary grid, like the one in Figure  \ref{horizontal-grid-stretch}?	
	\begin{figure}[h!]
			\begin{center}
		\includegraphics[trim={0cm 0.cm 0cm 0cm},clip,width=0.8\textwidth]{Figures/horizontal-irregular-grid}
		\end{center}
		\caption{Variable grid length in the horizontal direction: example from the MPAS GCM, used for weather and climate applications}
			\label{horizontal-grid-stretch}
	\end{figure}

{\bf Vertical example:} $\Delta z = 500m$, $c = O(1) ms^{-1}$ what $\Delta t$ can we afford?

... and what about a typical vertical atmospheric grid, telescoped, like the one in Figure \ref{veftical-grid-stretch}?
\begin{figure}[h!]
	\begin{center}
	\includegraphics[trim={0cm 0.cm 0cm 0cm},clip,width=0.5\textwidth]{Figures/vertical-telescoped-grid}
	\end{center}
	\caption{Variable grid length in the vertical direction: example from a typical GCM used for weather and climate}
	\label{vertical-grid-stretch}
\end{figure}

\clearpage

\section{The CFL condition in 2 dimensions}

Waves in 2D can propagate in \underline{any} direction; they are not bound to travel along the zonal or meridional direction.

\begin{center}
	\includegraphics[width=0.7\textwidth]{Figures/2DGaussian.png}
\end{center}

\vspace{2em}
	\fcolorbox{ocre}{lightgray}{\parbox{\dimexpr \linewidth-2\fboxsep-2\fboxrule}{
		\textbf{Question: }
		what does this 2D propagation mean for our CFL criterion? Is the time step requirement going to be stricter? Or is it going to be less strict?
}}
\vspace{2em}
	
{\bf How to solve this puzzle:} draw a simple circular wave front on a 2D cartesian grid and think of the distance covered by the wave front over a sequence of 3-4 time steps.
What happens to the signal along the $x$, $y$ axes and what happens along the diagonal?

\subsection{The Courant-Friedrichs-Levy (CFL) condition in 2D}

	Having explored these ideas graphically, we can approach the advection problem once again, albeit in two dimensions (2D) in a more formal way:
	
	\begin{equation}
		\frac{\partial A}{\partial t} + \mathbf{V}_s \cdot \grad{A} = 0 \textrm{, where } \mathbf{V}_s=U\mathbf{i}+V\mathbf{j} = \textrm{const.}
	\end{equation}
	
	Let $\Delta x = \Delta y =d$; $x=jd$; $y=kd$; $t=n \Delta t$ and apply CTCS as before:
	
	\begin{equation}
		{A_{j,k}^{n+1}-A_{j,k}^{n-1}} = - \frac{ U \Delta t}{d} \left( A_{j+1,k}^{n} - A_{j-1,k}^{n}  \right) - \frac{ V \Delta t}{d} \left( A_{j,k+1}^{n} - A_{j,k-1}^{n}  \right) 
		\label{adv-CTCS-2D}
	\end{equation}
	
	Just as before, we apply a solution following the Von Neumann method: ${A_{j,k}^{n}=B^{n \Delta t} e^{i pjd + qkd }}$ and we simplify by cancelling common terms, just as we did in the 1D case, leading to:
	
	\begin{equation}
		B^{n+1}  =  B^{n-1}  - \frac{2 i \Delta t}{d}  \left ( U \sin pd + V \sin qd \right) B_n
		\label{adv-CTCS-2D-solution}
	\end{equation}
	
	We can solve by defining $D=B^{n-1}$, re-writing (\ref{adv-CTCS-2D-solution}) in matrix form and finding its eigenvalues. This is quite similar to what is done on page 127 of \cite{Haltiner-Williams-book} to solve the 1D case we treated before, which results in a quadratic equation ((5-55), page 127 of \cite{Haltiner-Williams-book}) with coefficients identical to what we found previously (equation (\ref{adv-CTCS-1D-eigenvalues}), else see equation (5-16) in \cite{Haltiner-Williams-book}, page 112).

\subsection{The 2D eigenvalues for stability and the 2D CFL}

	The eigenvalues are:
	
	\begin{equation}
		\lambda = -i \frac{ \Delta t}{d}  \left ( U \sin pd + V \sin qd \right) \pm \sqrt { 1-  (\frac{ \Delta t}{d})^2  \left ( U \sin pd + V \sin qd \right) ^2 }
	\end{equation}
	
	which will have (the desired) magnitude of $1$ provided that: 
	\begin{equation}
		\frac{ \Delta t}{d}  \left | U \sin pd + V \sin qd \right | \leq 1
	\end{equation}
	
	At this time we consider $U$ and $V$ as projections of the vector $\mathbf{V}_s$ onto the $x$ and $y$ directions, making use of the "angle of the wind", or \emph{direction of the wave front} $\theta$: $U=V_s \cos \theta$ and $V=V_s \sin \theta$, which gives us:
	
	
	\begin{equation}
		\frac{ V_s\Delta t}{d}  \left | \cos \theta \sin pd + \sin \theta \sin qd \right | \leq 1
	\end{equation}
	
	Which is the wind direction $\theta$ that is most likely to violate CFL? Since the wave numbers $p,q$ are independent, we can choose the maximum value, $1$, for each of the two terms: $\sin pd$ and $\sin qd$, and we are left with the task of maximising the remaining sum: $\cos \theta + \sin \theta$, which is $\sqrt{2}=0.707$, corresponding to $\theta=\pi/4=45^o$.


\begin{theorem}[CFL condition in 2D]
	For 2D CTCS, the CFL condition that will allow us to avoid computational instability under all circumstances (even with wind blowing at an angle of $45^o$) is: 
	\begin{equation}
		V_s \Delta t \sqrt{2} / d \leq 1 \textrm{   ~~~~~      or     ~~~~~  } V_s \Delta t \leq 0.707 d 
	\end{equation}
\end{theorem}


\begin{figure}[h!]
	\begin{center}
		\includegraphics[width=0.4\textwidth]{Figures/GameOver.jpg}
		\caption{CFL in the 2D: what happens at the time of "BOOM!"}
	\end{center}
\end{figure}	

	
\subsection{2D CFL: graphical interpretation}

	In summary, we have seen that in 2D the maximum value of $\Delta t$ required for computational stability is almost 30\% smaller than for the 1D case, other things being equal. In order to understand this, consider the two figures below: on the left is the grid as seen normally, while on the right is the grid as seen when aligned with the progressing wavefront, at an angle of $45^o$.
	A wave propagating at that angle (e.g from the SW to the NE) encounters an effective distance of $d/\sqrt{2}$ between gridpoints. So, the CFL stability criterion says that the wave cannot move more than the effective distance between gridpoints ($d/\sqrt{2}$ at its shortest) during time $\Delta t$, without incurring computatational instability.
	
	\begin{figure}[h!]
	\mbox{\scalebox{0.5}{\includegraphics[trim={5.25cm 15.cm 5.25cm 2cm},clip,width=1.1\textwidth]{Figures/2D_CFL_drawing}}
	            \scalebox{0.5}{\includegraphics[trim={4.25cm 15.cm 7.25cm 2.cm},clip,width=.9\textwidth]{Figures/2D_CFL_drawing_rotated}}}
			\caption{Wavefront propagation across a 2D checkerboard: as seen with the original grid orientation (left panel), and as seen by aligning with the wavefront, at 45 degrees (right panel).}
	\end{figure}
	

\section{CFL and coupled equations: where does $\Delta t$ matter?}

The goal of the second project is to solve the Stommel model on the finite difference C-grid with a FBT time integration scheme. The Stommel model is the simplest geophysical flow model able to represent a wind-driven circulation in a closed basin with a western boundary layer. The model equations read:
\begin{eqnarray}
	&&\frac{\partial \mathbf{u}}{\partial t} + (f_0+\beta y) \mathbf{e}_z \times
	\mathbf{u} = -g \bnabla \eta - \gamma \mathbf{u} + \frac{\btau^{\eta}}{\rho h},\\
	&&\frac{\partial \eta}{\partial t} + h \bnabla \cdot \mathbf{u} = 0,
\end{eqnarray}
where $f_0$ is the reference value of the Coriolis parameter, $\beta$ is the reference value of the Coriolis parameter first derivative in the $y$-direction, $\gamma$ is a linear friction coefficient, $\rho$ is the homogeneous density of the fluid and $\btau^{\eta}$ is the wind stress acting on the surface of the fluid. The model solutions look like these:

\begin{figure}[h!]
	\begin{center}
		\includegraphics[width=0.9\textwidth]{Figures/sol_project2.eps}
	\end{center}
\end{figure}


\subsection{Practical considerations: stability analysis in 1D and 2D for Project 2}

When solving this problem numerically, and considering 2D CFL, let us not confuse the advection process with the other processes, described in parametrizations, or with the coupling between the equations. 
As an example, take a look at the numerical solutions proposed for the SWEs that will be the focus of Practical 2:

\begin{eqnarray*}
	\frac{u^{n+1}-u^{n-1}}{2\Delta t}-fv^n+\frac{g}{2}
	\left( \partd{h}{x}^{n+1} +\partd{h}{x}^{n-1} \right) & = & 0 \\
	\frac{v^{n+1}-v^{n-1}}{2\Delta t}+fu^n+\frac{g}{2}
	\left( \partd{h}{y}^{n+1} +\partd{h}{y}^{n-1} \right) & = & 0 \\
	\frac{h^{n+1}-h^{n-1}}{2\Delta t}+\frac{H}{2}
	\left( \partd{u}{x}^{n+1} +\partd{u}{x}^{n-1}
	+\partd{v}{y}^{n+1} +\partd{v}{y}^{n-1} 
	\right) & = & 0. 
\end{eqnarray*}

\emph{Questions}: can you identify the advection terms, and how CFL impacts the stability of the solutions? What processes are coupled at every time step?

Also, for a real NWP or climate model, remember that the time step may be set by the 2D CFL, but the vertical grid spacing is very likely far smaller, meaning that we must use a different numerical scheme in the vertical.

\subsection{How do we decide on grid, grid spacing, time step?}

Remember that we started with the CFL criterion in 1D:

\begin{equation}
	\mu = U \frac {\Delta t}{\Delta x} \le 1
\end{equation}

There are now several ways in which we can meet the stability criteria for our particular 2D SWE problem. Thinking a bit harder about these decisions seems to lead to some circular reasoning: where do we start? \textbf{Are there tradeoffs?} 
\medskip

Questions that we must consider when we choose the numerical method to solve our problem:
\begin{enumerate}
	\item What is the "worst case scenario" distance relevant to CFL for SWEs in 2D?
	\item How many grid points are the minimum required to resolve anything?
	\item What is the size of our domain?
	\item What is the size of the phenomenon we are trying to simulate?
	\item What do all these decisions end up causing in terms of the physical realism of the solutions? For instance, compromising the accuracy of solution for one wave type.
	
\end{enumerate}

\section{More considerations in choosing $\Delta t$ and $\Delta x $}
\subsection{Numerical analysis of the 1D "upstream" advection equation}

It is not a given that going for a small $\mu$ is always the best possible decision.
For instance, let us start with the advection equation and let us choose the \textbf{upstream scheme}, which we know is \textbf{dissipative}. That means, each time step we lose a bit of the signal.


If we increase the resolution of our model, making the time step smaller, we may think that we will obtain a better solution. However, this is not necessairly true for every $\mu$.

\medskip
Consider a domain of size $D$, resolved by a number $J$ of $\Delta x$ points, in which we are solving the advection equation:

\begin{equation}
	\frac {A_j^{n+1}-A_j^{n}}{\Delta t} +c \left(  \frac {A_j^{n}-A_{j-1}^{n}}{\Delta x}  \right) = 0
	\label{adv-upstream}
\end{equation}

$D=J \Delta x$ and {\bf every time that we decrease $\Delta x$ we increase $J$, so that the domain size D does not change}. For a signal that has wavenumber $k$ in the $x$ direction:

\begin{equation}
	k \Delta x = \frac {k D}{J}
\end{equation}


\subsection{Stability analysis of the upstream advection equation}


If you remember your Von Neuman analysis, \textbf{the amplification factor, $\lambda$} for upstream advection (equation \ref{adv-upstream})	has this form:

\begin{equation}
	|\lambda|^2 = 1+2\mu(\mu-1)(1-\cos k \Delta x) = 1+2\mu(\mu-1)[1-\cos ( \frac {k D} {J})]
\end{equation}

which tells us that $\lambda$ depends on both wavenumber ($k$) and our Courant number ($\mu$). 
In order to maintain computational stability, \emph{we keep $\mu$ fixed as $\Delta x$ decreases}, and that decision limits our time step:

\begin{equation}
	\Delta t = \frac {\mu \Delta x}{c} = \frac {\mu D}{cJ}
\end{equation}

Since we know the velocity of the fluid, $c$, we also know how long it takes for it to cross the entire domain: $T=\frac {D}{c}$. If we choose to carry out the time integration in $N$ steps:

\begin{equation}
	N= \frac {T}{\Delta t} = \frac {D}{c \Delta t} = \frac {D}{\mu \Delta x} = \frac {J}{\mu}
\end{equation}



\subsection{How damped is the signal after N time steps?}
The total amount of damping that "accumulates" throughout the time integration ($N$ time steps) is given by:

\begin{equation}
	|\lambda|^N=(|\lambda^2|)^{N/2} = \left \{ 1 -2 \mu (1-\mu) \left[ 1 - \cos ( \frac {k D} {J}) \right] \right \} ^{\frac {J}{2 \mu}}
\end{equation}

This relationship says that $\lambda$ depends on $J$ (the resolution) in two different ways, so that making $\Delta x$ smaller (increasing the resolution) can be good or bad. 

\medskip
Which is it?
Let us pick a wavelength half the domain size, so that $kD=4 \pi$. This causes the $\cos$ factor in the equation to approach 1, which \textbf{weakens the damping}; on the other hand it also causes the exponent to increase, which \textbf{strenghtens the damping}. This is shown in the figure, for two values of $\mu$.

	\begin{figure}[h!]
		\includegraphics[trim={0cm 0cm 0cm 0cm},clip,width=15.cm]{/Users/vidale/Projects/Teaching/MTMW14/Figures/Resolution_damping.png}
		\caption{The compromise between simulation stability and fidelity}
	\end{figure}

	Overall, \textbf{increasing the resolution, $J$, is good}: $\lambda$ tends to 1 on the right hand side, even though we take more time steps, $N$, to complete the integration. However, if we fix $J$ and decrease $\mu$, by decreasing the time step $\Delta t$, the damping increases and \textbf{the solution will be less accurate}. 

\subsection{Recommendations on choosing $\Delta t$ and $\Delta x $}

	\begin{itemize}
		\item perform numerical analysis, speficically for each scheme
		\item \textbf{for the upstream scheme}, the amplitude error can be minimised by using the largest stable value of $\mu$.
		\item that is, do not exaggerate in making $\mu$ much less than 1: this decision impacts both cost and quality of the solutions
		\item in other words, \underline{it pays off to live dangerously}!
	\end{itemize}

\section{Real world applications: CFL in current GCMs}

In lectures 4,5 we will learn about Arakawa-sensei's staggered grids, and I showed you my current C-grid configuration, for the UM's EndGame dynamical core. \\

In the last row near the poles, my $\Delta x \approx 5$km GCM has a grid spacing of 3.1m. Given a "worst case scenario" for wave propagation, what should my time step be, according to the 1D ("optimistic!") CFL criterion? Please compute that now, and raise your hand as soon you have the answer.\\

Here is my real time step: 

\begin{figure}[h!]
	\includegraphics[width=1.\textwidth]{Figures/Stevens_Table5}
	\caption{DYAMOND GCMs, with typical $\Delta x \approx 5$km, from Stevens et al., 2019}
\end{figure}

How can such magic be possible?

\subsection{Hurricane and Typhoon simulation in climate GCMs: frequency}

\begin{figure}[h!]
\begin{center}	
	\includegraphics[width=0.8\textwidth]{Figures/Zarz1}
\end{center}
\end{figure}


\subsection{Hurricane and Typhoon simulation in climate GCMs: dynamics and physics}

\begin{figure}[h!]
\begin{center}	
	\includegraphics[width=0.8\textwidth]{Figures/Zarz2}
\end{center}
\end{figure}

